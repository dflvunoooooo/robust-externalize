\documentclass[a4paper,doc2]{ltxdoc} % doc2 is needed to force the old version, or links get colored in a weird red way even with hidelinks. https://github.com/latex3/latex2e/issues/822

%%%%%%%%%%%%%%%%%%%%%%%%%%%%%%
%%% Packages
%%%%%%%%%%%%%%%%%%%%%%%%%%%%%%
%% Warning: if you compile and get:
%% ERROR: Argument of \tikz@lib@matrix@with@options has an extra }.
%% make sure to fix catcodes around it as | is given a different meaning in ltxdoc.

\usepackage{amsmath}
\usepackage[margin=3cm]{geometry}
\usepackage{calc}
\usepackage{tikz}
\usetikzlibrary{shadows,fit}
% \usetikzlibrary fails because file is not in current directory, lazy to setup TEXINPUTS
\makeatletter
  \ProvidesPackage{robust-externalize}[2.2+unstable Cache anything (tikz, latex, python) in a robust, efficient and pure way.]
% todo: 
% change order argument replace from list, it is hard to read this way

\RequirePackage{pgfkeys} % We use the /robExt/... path to store our keys.
\RequirePackage{pgffor} % For the .list keys
\RequirePackage{graphicx} % For the includegraphics command
\RequirePackage{verbatim} % For the \verbatim command, useful for debugging purpose for instance
\RequirePackage{xsimverb} % To easily write verbatim code to files
\RequirePackage{etoolbox} % To easily write verbatim code to files
\RequirePackage{iftex} % computing md5 sum differs on lualatex/pdflatex/xelatex
\RequirePackage{xparse} % solve https://github.com/leo-colisson/robust-externalize/issues/6
\RequirePackage{atveryend} % to compile in parallel all images
%\usepackage{etl} % For auto forward https://tex.stackexchange.com/a/700844/116348

%% TODO list:
% - provide an easy way to use cross-ref, bibtex etc (we just need to add them when writing the file) without recompiling the whole document (we don't want to lose the cache everytime a new bib entry is added) but while preserving.

%%%%%%%%%%%%%%%%%%%%%%%%%%%%%%
%%%%%%%%%%% Utils %%%%%%%%%%%%
%%%%%%%%%%%%%%%%%%%%%%%%%%%%%% 

\ExplSyntaxOn
\NewDocumentCommand{\robExtIfWindowsTF}{mm}{%
  \sys_if_platform_windows:TF {#1}{#2}%
}
\ExplSyntaxOff

% Debug function: by default, do nothing
\def\robExtDebugMessage#1{}
\def\robExtDebugInfo#1{\message{#1}}
\def\robExtDebugWarning#1{\message{^^J[robExt warning] #1}}

\NewDocumentCommand{\robExtConfigure}{m}{%
  \pgfqkeys{/robExt}{#1}% faster than \pgfkeys{/robExt/.cd,#1}
}

% https://tex.stackexchange.com/questions/690700/latex3-elegant-way-to-forward-a-variable-outside-of-the-group
% modified to deal with csname instead
\def\robExtKeepaftergroup#1{%
   %\global \expandafter \expandafter \let \csname x:#1\endcsname =\csname #1\endcsname
   \global \expanded{\noexpand \let \expandafter\noexpand\csname x:#1\endcsname =\expandafter\noexpand\csname #1\endcsname}%
   \aftergroup\let%
   \expandafter\aftergroup\csname #1\endcsname%
   \expandafter\aftergroup \csname x:\string#1\endcsname%
}

\ExplSyntaxOn

%%%
%%% Expl3 variants
%%%

\cs_generate_variant:Nn \str_replace_all:Nnn { Nnx, Nnv, cnx, cxx, NnV, Nnx, Nnv }

\cs_generate_variant:Nn \str_set:Nn { NV }

\cs_generate_variant:Nn \tl_rescan:nn { nv }
\cs_generate_variant:Nn \tl_set_rescan:Nnn { Nnv, NnV }
\cs_generate_variant:Nn \tl_gset_rescan:Nnn { cnv }

\cs_generate_variant:Nn \seq_remove_all:Nn { NV }

\cs_generate_variant:Nn \iow_now:Nn { NV,Nx }
\cs_generate_variant:Nn \iow_open:Nn { Nx }
\cs_generate_variant:Nn \ior_str_get:NN { Nc }

\cs_generate_variant:Nn \file_if_exist:nTF { xTF }

\cs_generate_variant:Nn \cs_replacement_spec:N { c }
\cs_generate_variant:Nn \cs_argument_spec:N { c }

\cs_generate_variant:Nn \regex_extract_all:nnN { VVN, nVN, VVN }
\cs_generate_variant:Nn \regex_extract_all:NnN { NVN }


\cs_generate_variant:Nn \regex_match:nnTF { nVTF }

%%%
%%% String manipulation
%%%

\cs_set:Nn \robExt_set_hash_robust:Nn {
  \str_set:Nn {#1} {#2}
  \str_replace_all:Nnx {#1} { ## } { \c_hash_str }
}
\cs_generate_variant:Nn \robExt_set_hash_robust:Nn { cn }
\cs_generate_variant:Nn \robExt_set_hash_robust:Nn { cx }

\NewDocumentCommand{\robExt@set@hash@robust}{mm}{\robExt_set_hash_robust:Nn #1 {#2}}

% Double the number of hashes... quite dirty but cannot find any solution or the user need to double it itself
%% See also https://tex.stackexchange.com/questions/695432/latex3-latex-doubles-the-number-of-hashes-when-storing-them-in-string/695460#695460
\NewDocumentCommand{\robExtStrSetDoubleHash}{mm}{
  \str_set:Nn {#1} {#2}
  \str_replace_all:Nnx {#1} { ## } { \c_hash_str \c_hash_str }
}

\NewDocumentCommand{\robExtStrSetHashRobust}{mm}{
  \robExt_set_hash_robust:Nn {#1} {#2}
}

\NewDocumentCommand{\robExtRescanHashRobust}{m}{
  \robExt_set_hash_robust:Nn \l__robExt_tmp_str {#1}
  \tl_rescan:nv {}{ l__robExt_tmp_str }
}

%%%%%%%%%%%%%%%%%%%%%%%%%%%%%%%%%%%%%%%
%%%%%%%%%%% Error messages %%%%%%%%%%%%
%%%%%%%%%%%%%%%%%%%%%%%%%%%%%%%%%%%%%%%
%% In LaTeX3 error messages are defined here, this allows people to translate them, filter them, change them,
%% make them more verbose etc.
\msg_new:nnn {robExt}{underscore in name}{Error:~all~the~placeholders~should~contain~two~consecutive~underscores~in~their~name.}
\msg_new:nnn {robExt}{placeholder not existing}{[robExtGetPlaceholderInResultFromSinglePlaceholder] ~ The ~ placeholder ~ #3 ~ does ~ not ~ exists.}
\msg_new:nnn {robExt}{forgot template}{You ~ are ~ writing ~ __ROBEXT_TEMPLATE__ ~ to ~ your ~ file: ~ seems ~ like ~ you ~ forgot ~ to ~ define ~ your ~ template ~ or ~ set ~ a ~ preset}
\msg_new:nnn {robExt}{need shell escape}{You ~ need ~ to ~ compile ~ with ~ shell-escape ~ as ~ in: ~ "pdflatex ~ -shell-escape ~ yourfile.tex" ~ to ~ be ~ able ~ to ~ compile ~ automatically ~ the ~ figures}
\msg_new:nnn {robExt}{need shell escape parallel}{You ~ need ~ to ~ compile ~ with ~ shell-escape ~ as ~ in: ~ "pdflatex ~ -shell-escape ~ yourfile.tex" ~ to ~ be ~ able ~ to ~ compile ~ automatically ~ the ~ figures ~ in ~ parallel}
\msg_new:nnn {robExt}{missing compiled pdf}{The ~ pdf ~ file ~ #1.pdf ~ is ~ not ~ present. ~ The ~ compilation ~ command ~ certainly ~ failed, ~ see ~ logs ~ above ~ or ~ in ~ #1.log.}
\msg_new:nnn {robExt}{missing compiled pdf parallel}{The ~ pdf ~ file ~ #1.pdf ~ is ~ not ~ present. ~ The ~ compilation ~ command ~ "#3"~used~to~compile~the~environment~on~line~#2~ certainly ~ failed, ~ see ~ logs ~ above ~ or ~ in ~ #1.log.}
\msg_new:nnn {robExt}{auto forward not in cachemecode}{Auto~forward~is~less~efficient~in~cacheMeCode.}

% warnings
\msg_new:nnn {robExt}{recommend two underscores in placeholders}{We~recommend~to~use~only~placeholders~containing~two~consecutive~underscores~in~their~name}
\msg_new:nnn {robExt}{enabled parallel no shell escape}{Warning:~you~enabled~parallel~compilation~but~shell-escape~is~disabled.}
\msg_new:nnn {robExt}{rerun because parallel}{Warning:~Compiling~all~missing~figures~in~parallel~with~"#1".~You~need~to~rerun~LaTeX~to~include~them.}

% dummy placeholders added if image is not present
\def\robExtImagePlaceholderIfManualMode{
  \framebox[\linewidth]
  {
    \begin{minipage}{\linewidth}
      \textbf{Manual ~ mode: ~ either ~ compile ~ with ~ \texttt{-shell-escape} ~ or ~ compile:\newline \texttt{\robExtAddCachePathAndName{\robExtFinalHash.tex}}\newline via ~ \newline \texttt{\l__robExt_current_compilation_command_str}\newline or ~ call ~ \newline\texttt{bash ~ \jobname-\robExtAddPrefixName{compile-missing-figures.sh}}\newline to ~ compile ~ all ~ missing ~ figures.}
    \end{minipage}
  }
}

\def\robExtImagePlaceholderIfFallbackMode{
  \framebox[\linewidth]
  {
    \begin{minipage}{\linewidth}
      \textbf{Falling ~ back ~ to ~ draft ~ mode: ~ either ~ compile ~ with ~ \texttt{-shell-escape} ~ or ~ compile:\newline \texttt{\robExtAddCachePathAndName{\robExtFinalHash.tex}}\newline via ~ \newline \texttt{\l__robExt_current_compilation_command_str}\newline or ~ call ~ \newline\texttt{bash ~ \jobname-\robExtAddPrefixName{compile-missing-figures.sh}}\newline to ~ compile ~ all ~ missing ~ figures.}
    \end{minipage}
  }
}

\def\robExtImagePlaceholderIfParallelCompilation{
  \framebox[\linewidth]
  {
    \begin{minipage}{\linewidth}
      \textbf{The~file:\\
        \texttt{\robExtAddCachePathAndName{\robExtFinalHash.tex}}\\
        is~compiled~in~parallel~mode.~You~need~to~recompile~your~document~to~include~it.}
    \end{minipage}
  }
}


\ExplSyntaxOff

%%%%%%%%%%%%%%%%%%%%%%%%%%%%%%
%%%%%%%%%% SCRIPTS %%%%%%%%%%%
%%%%%%%%%%%%%%%%%%%%%%%%%%%%%% 

%%% Important: this script is NOT executed by this file, it is only created so that users can remove unused previously cached files by running this script
%%% outside of LaTeX.
%%% Create scripts to remove useless files:
%%% Note that we don't override the script if it exists on purpose (the user might have changed it to fits their needs)
\begin{filecontents}[noheader]{robExt-remove-old-figures.py}
#!/usr/bin/env python3
import os
import re
import glob
# Just run this script in order to remove all old figures not listed in robExt-all-figures.txt.

# Note that this part is not extracted from the pdf file since it might be different on a previous run. You can however hardcode
# it here, your updated script will not be overriden unless you remove it yourself.
prefixes = [ "robExt-" ]
folders  = [ "robustExternalize" ]

def main():
    imagesToKeep = dict()
    list_all_figures_file = glob.glob('*robExt-all-figures.txt')
    for filename in list_all_figures_file:
        with open(filename) as f:
            for line in f:
                line = line.strip()
                if line.endswith('.tex'):
                    imagesToKeep[line[:-4]] = True # The exact value is not important, we mostly use dict to get ~O(1) access

    listOfFilesToRemove = []
    # We are looking for images in the folders
    for folder in folders:
        for root, dirs, files in os.walk(folder):
            for f in files:
                for prefix in prefixes: # Not the most efficient, but anyway we typically have a single prefix
                    # In case prefix contains weird caracters that collide with regexps:
                    prefixEsc = re.escape(prefix)
                    # result_search = re.search(rf"^({prefixEsc}[A-F0-1]{32}).*", f)
                    result_search = re.search(rf"^(.*[A-F0-9]{{32}}).*", f)
                    if result_search:
                        if result_search.group(1) not in imagesToKeep:
                            listOfFilesToRemove.append(os.path.join(root,f))
    for f in listOfFilesToRemove:
        print(f"-- {f}")
    print(f"Above are the files to remove, are you sure you want to proceed? [y/N] (based on prefixes {prefixes})")
    x = input().strip()
    if x not in ["y", "Y"]:
        print("All right, we abort.")
        exit(1)
    for f in listOfFilesToRemove:
        os.remove(f)
        print(f"Removed {f}")
        
if __name__ == '__main__':
    main()
\end{filecontents}

\ExplSyntaxOn

%%%%%%%%%%%%%%%%%%%%%%%%%%%%%%%%%%%%%%
%%%%%%%%%%% Compatibility %%%%%%%%%%%%
%%%%%%%%%%%%%%%%%%%%%%%%%%%%%%%%%%%%%%
% These commands exist on really recent kernels (~june 2023), so we provide them in case they do not already exist

\ProvideDocumentCommand \NewEnvironmentCopy {mm} {
    \expandafter \NewCommandCopy \csname#1\expandafter\endcsname \csname#2\endcsname
    \expandafter \NewCommandCopy \csname end#1\expandafter\endcsname \csname end#2\endcsname
}

\ProvideDocumentCommand \RenewEnvironmentCopy {mm} {
    \expandafter \RenewCommandCopy \csname#1\expandafter\endcsname \csname#2\endcsname
    \expandafter \RenewCommandCopy \csname end#1\expandafter\endcsname \csname end#2\endcsname
}

\ProvideDocumentCommand \DeclareEnvironmentCopy {mm} {
    \expandafter \DeclareCommandCopy \csname#1\expandafter\endcsname \csname#2\endcsname
    \expandafter \DeclareCommandCopy \csname end#1\expandafter\endcsname \csname end#2\endcsname
}

\ProvideDocumentCommand \RenewEnvironmentCopy {mm} {
    \expandafter \RenewCommandCopy \csname#1\expandafter\endcsname \csname#2\endcsname
    \expandafter \RenewCommandCopy \csname end#1\expandafter\endcsname \csname end#2\endcsname
}

%% Depending on the engine, computing md5 is done in different ways
\sys_if_engine_luatex:TF{
  \def\robExt@pdfmdfivesum#1{%
    \directlua{tex.print(string.upper(md5.sumhexa("\luaescapestring{#1}")))}%
  }%
}{
  \sys_if_engine_xetex:TF {%
    \def\robExt@pdfmdfivesum{\mdfivesum}%
  }{%
    \def\robExt@pdfmdfivesum{\pdfmdfivesum}%
  }%
}

%%%%%%%%%%%%%%%%%%%%%%%%%%%%%%
%%%%%%%%%%% Paths %%%%%%%%%%%%
%%%%%%%%%%%%%%%%%%%%%%%%%%%%%%

\def\robExtPrefixFilename{robExt-}

\NewExpandableDocumentCommand{\robExtAddCachePathAndName}{m}{%
  \ifdefined\robExtCacheFolder%
    \robExtCacheFolder%
  \fi\robExtPrefixFilename#1%
}

\NewExpandableDocumentCommand{\robExtAddCachePath}{m}{%
  \ifdefined\robExtCacheFolder%
    \robExtCacheFolder%
  \fi#1%
}

\NewDocumentCommand{\robExtCheckIfPrefixFolderExists}{}{
  % Check if the output directory exists
  \ifdefined\robExtCacheFolder
    \bool_if:nTF { \sys_if_shell_unrestricted_p: || \cs_if_exist_p:N \robExtForceCompilation}
    {
      \ifdefined\robExtDoNotMkdirFolder\else
        \ifdefined\robExtManualMode
          \message{If ~ you ~ get~ an~ error,~ make ~ sure ~ to ~ enable ~ pdflatex ~ -shell-escape ~ or ~ to ~ MANUALLY ~ CREATE ~ THE ~ FOLDER ~ \robExtCacheFolder.}
        \else
          \sys_shell_now:x {mkdir ~ -p ~ \robExtCacheFolder}
        \fi
      \fi
    }{
      \message{If ~ you ~ get~ an~ error,~ make ~ sure ~ to ~ enable ~ pdflatex ~ -shell-escape ~ or ~ to ~ manually ~ CREATE ~ THE ~ FOLDER ~ \robExtCacheFolder.}
    }
  \fi
}


\NewExpandableDocumentCommand{\robExtGetPrefixPath}{}{%
  \ifdefined\robExtCacheFolder%
    \robExtCacheFolder%
  \fi%
}


\NewExpandableDocumentCommand{\robExtAddPrefixName}{m}{%
  \robExtPrefixFilename#1%
}

%% Todo: not sure if I should use \seq_push:Nx \l_file_search_path_seq {subfolder}
%% to find the subfolder (seems to work for input/includegraphics/...), or if it's
%% better to hardcode the subfolder.

%%%%%%%%%%%%%%%%%%%%%%%%%%%%%%%%%%%%%%%%%%%%%%%%%%%%%%%%%
%%%%%%%%%%% Setup new commands and variables %%%%%%%%%%%%
%%%%%%%%%%%%%%%%%%%%%%%%%%%%%%%%%%%%%%%%%%%%%%%%%%%%%%%%%

%% Temporary: used when quickly writing to a file
\iow_new:N \g__robExt_write_iow
\ior_new:N \g__robExt_read_ior
%% This is used to write the full list of figures in a single file (used for instance by Makefile etc...)
%% TODO: create a makefile.
\iow_new:N \g__robExt_write_list_all_figures_iow
%% Create a file robExt-all-figures.txt with the list of .tex files
\iow_open:Nx \g__robExt_write_list_all_figures_iow {\jobname-\robExtAddPrefixName{all-figures.txt}}
\iow_new:N \g__robExt_write_manually_compile_all_missing_figures_iow
\iow_open:Nx \g__robExt_write_manually_compile_all_missing_figures_iow {\jobname-\robExtAddPrefixName{compile-missing-figures.sh}}

% Contains the list of dependency files (useful to compute the final md5sum)
\seq_new:N \l__robExt_dependencies_str
% Contains a string where on each line we have: "md5sum, dependency". The first line has nothing as "dependency" as it is the main fine whose final name is the md5sum of the dependencies.
\str_new:N \l__robExt_dependencies_mdfive_str
% Contains the current compilation command (including the name of the file to compile).
\str_new:N \l__robExt_current_compilation_command_str
\str_new:N \l__robExt_tmp_str


% To have a code compatible even on older versions
% https://tex.stackexchange.com/questions/615010/how-can-i-use-latex-hooks-in-historical-versions-of-tex-live
\providecommand\RobExtIfFormatAtLeastTF{\@ifl@t@r\fmtversion}
\RobExtIfFormatAtLeastTF{2020-10-01}%
{
  % https://ctan.math.illinois.edu/macros/latex/base/lthooks-doc.pdf
  \NewHook{robust-externalize/just-before-writing-files}
  \def\robExtUseHookJustBeforeWritingFiles{\UseHook{robust-externalize/just-before-writing-files}}%
}%
{
  \def\robExtUseHookJustBeforeWritingFiles{}
}

%%%%%%%%%%%%%%%%%%%%%%%%%%%%%%%%%%%%%
%%%%%%%%%%% Placeholders %%%%%%%%%%%%
%%%%%%%%%%%%%%%%%%%%%%%%%%%%%%%%%%%%%

%% Placeholders are strings like "__MYLIBRARY__" or "__MAINCONTENT__" that will be replaced by a given content.
%% In practice:
%% - for every placeholder MYPLACEHOLDER, a macro \l__robExt_placeholder_MYPLACEHOLDER_str is created, containing
%%   the string to use to replace it
%% - a sequence \l__robExt_placeholder_group_main_seq that is a list of string, contains the list of all placeholders,
%%   in a string, like [MYLIBRARY, MAINCONTENT] etc...
%% One issue is that LaTeX does not keep some symbols (e.g. % etc...) when used inside a macro, so we define
%% different commands depending on whether they can be used in a macro or not, whether they should be taken
%% from an external file etc...
\seq_clear_new:N \l__robExt_placeholder_group_main_seq 

%%%
%%% Debug in terminal
%%%

\NewDocumentCommand{\robExtShowPlaceholder}{sm}{
  \cs_if_exist:cTF {l__robExt_placeholder_#2_str} {
    \message{Placeholder ~ #2 ~ contains:^^J~ \use:c{l__robExt_placeholder_#2_str}}
  }{
    \message{Placeholder ~ #2 ~ does ~ not ~ exist.}
  }
  \IfBooleanTF{#1}{}{\show\def}
}

\NewDocumentCommand{\robExtShowPlaceholders}{s}{
  \message{List ~ of ~ placeholders:}
  \seq_map_inline:Nn \l__robExt_placeholder_group_main_seq {\message{##1,}}
  \IfBooleanTF{#1}{}{\show\def}
}

\NewDocumentCommand{\robExtShowPlaceholdersContents}{s}{
  \message{List ~ of ~ placeholders:}
  \seq_map_inline:Nn \l__robExt_placeholder_group_main_seq {\robExtShowPlaceholder*{##1}}
  \IfBooleanTF{#1}{}{\show\def}
}

%%%
%%% Debug in document
%%%

\NewDocumentCommand{\robExtPrintPlaceholderNoReplacement}{sm}{%
  % For some reasons, newlines are displayed as \Omega. We need to replace them with \\
  % https://tex.stackexchange.com/questions/694716/print-latex3-string-verbatim/694717
  \tl_set_eq:Nc \l__robExt_tmp_str { l__robExt_placeholder_#2_str }
  \tl_replace_all:Nnn \l__robExt_tmp_str {^^J} { \par }
  \tl_replace_all:Nnn \l__robExt_tmp_str { ~ } { \  }
  \IfBooleanTF{#1}{\texttt{\use:c{l__robExt_placeholder_#2_str}}}{\begin{flushleft}\ttfamily%
      \l__robExt_tmp_str
    \end{flushleft}%
  }
}
\let\printPlaceholderNoReplacement\robExtPrintPlaceholderNoReplacement

\NewDocumentCommand{\robExtPrintPlaceholder}{sm}{%
  \robExtGetPlaceholderInResult{#2}
  % For some reasons, newlines are displayed as \Omega. We need to replace them with \\
  % https://tex.stackexchange.com/questions/694716/print-latex3-string-verbatim/694717
  \tl_set_eq:NN \l__robExt_tmp_str \l_robExt_result_str
  \tl_replace_all:Nnn \l__robExt_tmp_str {^^J} { \par }
  \tl_replace_all:Nnn \l__robExt_tmp_str { ~ } { \  }
  \IfBooleanTF{#1}{\texttt{\l__robExt_tmp_str}}{\begin{flushleft}\ttfamily%
      \l__robExt_tmp_str
    \end{flushleft}%
  }
}
\let\printPlaceholder\robExtPrintPlaceholder


\NewDocumentCommand{\robExtPrintAllPlaceholdersExceptDefaults}{s}{
  List ~ of ~ placeholders:\\
  \seq_map_inline:Nn \l__robExt_placeholder_group_main_seq {
    % We hide the elements starting with __ROBEXT_
    \str_if_in:nnTF { ##1 } { __ROBEXT_ } {
      \IfBooleanTF {#1} {
        - ~ Placeholder ~ called ~ \texttt{\tl_to_str:n {##1}} ~ defined ~ by ~ default ~ (we ~ hide ~ the ~ definition ~ to ~ save ~ space)\\
      }{}
    }{
      - ~ Placeholder ~ called ~ \texttt{\tl_to_str:n {##1}} ~ contains: \robExtPrintPlaceholderNoReplacement{##1}
    }
  }
}
\let\printAllPlaceholdersExceptDefaults\robExtPrintAllPlaceholdersExceptDefaults
\let\printImportedPlaceholdersExceptDefaults\robExtPrintAllPlaceholdersExceptDefaults

\NewDocumentCommand{\robExtPrintAllPlaceholders}{}{
  List ~ of ~ placeholders:\\
  \seq_map_inline:Nn \l__robExt_placeholder_group_main_seq {- ~ Placeholder ~ called ~ \texttt{\tl_to_str:n {##1}} ~ contains: \robExtPrintPlaceholderNoReplacement{##1}}
}
\let\printAllPlaceholders\robExtPrintAllPlaceholders
\let\printImportedPlaceholders\robExtPrintAllPlaceholders


%%%
%%% Evaluation of placeholders
%%%

\NewDocumentCommand{\robExtEvalPlaceholderNoReplacement}{m}{
  % scantokens add an empty space at the end, so we need to add \empty to avoid it having effects
  % https://tex.stackexchange.com/questions/213659/could-someone-further-elucidate-expansion-catcodes-and-scantokens
  \tl_rescan:nv {}{ l__robExt_placeholder_#1_str }
}
\let\evalPlaceholderNoReplacement\robExtEvalPlaceholderNoReplacement

\NewDocumentCommand{\robExtGetPlaceholderNoReplacement}{m}{
  \str_use:c { l__robExt_placeholder_#1_str }
}
\let\getPlaceholderNoReplacement\robExtGetPlaceholderNoReplacement

\NewDocumentCommand{\robExtRescanPlaceholderInVariableNoReplacement}{mm}{
  \tl_gset_rescan:cnv {#1} {} { l__robExt_placeholder_#2_str }
}
\let\rescanPlaceholderInVariableNoReplacement\robExtRescanPlaceholderInVariableNoReplacement

% Make sure that the placeholder is in the list \l__robExt_placeholder_group_main_seq.
% This should automatically be called by other tools
\NewDocumentCommand{\robExtAddPlaceholderToList}{m}{
  \cs_if_exist:NTF {\robExtCompletelyDisableWholeImportSystem}{}{
    % Make sure we have a string here:
    \robExt_set_hash_robust:Nn \l__robExt_tmp_str { #1 }
    % First we remove existing occurrences (useful to avoid listing the same macro more than once
    % if we redefine a macro):
    \seq_remove_all:NV \l__robExt_placeholder_group_main_seq \l__robExt_tmp_str
    \seq_put_right:NV \l__robExt_placeholder_group_main_seq \l__robExt_tmp_str
  }
}
\let\robExtImportPlaceholder\robExtAddPlaceholderToList
\let\importPlaceholder\robExtAddPlaceholderToList

\NewDocumentCommand{\robExtClearImportedPlaceholders}{}{
  \seq_clear:N \l__robExt_placeholder_group_main_seq
}
\let\clearImportedPlaceholders\robExtClearImportedPlaceholders

\NewDocumentCommand{\robExtRemoveImportedPlaceholder}{m}{
  \robExt_set_hash_robust:Nn \l__robExt_tmp_str { #1 }
  \seq_remove_all:NV \l__robExt_placeholder_group_main_seq \l__robExt_tmp_str
}
\let\removeImportedPlaceholder\robExtRemoveImportedPlaceholder


% add it first in the list
\NewDocumentCommand{\robExtAddPlaceholderToListFirst}{m}{
  % Make sure we have a string here:
  \robExt_set_hash_robust:Nn \l__robExt_tmp_str { #1 }
  % First we remove existing occurrences (useful to avoid listing the same macro more than once
  % if we redefine a macro):
  \seq_remove_all:NV \l__robExt_placeholder_group_main_seq \l__robExt_tmp_str
  \seq_put_left:NV \l__robExt_placeholder_group_main_seq \l__robExt_tmp_str
}
\let\robExtImportPlaceholderFirst\robExtAddPlaceholderToListFirst
\let\importPlaceholderFirst\robExtAddPlaceholderToListFirst


% add it first in the list
\NewDocumentCommand{\robExtAddPlaceholdersToListFirst}{m}{
  \robExt_set_hash_robust:Nn \l__robExt_tmp_str { #1 }
  \seq_set_from_clist:NN \l_tmp_seq \l__robExt_tmp_str
  % First we remove existing occurrences (useful to avoid listing the same macro more than once
  % if we redefine a macro):
  \seq_map_inline:Nn \l_tmp_seq {
    % This seems to be required to ensure catcodes are correct before removing the elements:
    \seq_remove_all:Nn \l__robExt_placeholder_group_main_seq {##1}
  }
  \seq_put_left:NV \l__robExt_placeholder_group_main_seq \l__robExt_tmp_str
}
\let\robExtImportPlaceholdersFirst\robExtAddPlaceholdersToListFirst
\let\importPlaceholdersFirst\robExtAddPlaceholdersToListFirst



\NewDocumentCommand{\robExtRemovePlaceholder}{m}{
  % This seems to be required to ensure catcodes are correct before removing the elements:
  \robExt_set_hash_robust:Nn \l__robExt_tmp_str { #1 }
  % First we check if it exists, no need to loop over everything otherwise
  \str_if_exist:cTF { l__robExt_placeholder_#1_str } {
    \seq_remove_all:NV \l__robExt_placeholder_group_main_seq \l__robExt_tmp_str
    \expandafter \let \csname l__robExt_placeholder_#1_str \endcsname \undefined
  } { }
}
\let\removePlaceholder\robExtRemovePlaceholder

\NewDocumentCommand{\checkIfPlaceholderNameIsLegal}{m}{
  \cs_if_exist:NTF {\robExtPlaceholderOnlyWithUnderscores} {
    \str_if_in:nnTF {#1}{__}{}{
      \msg_error:nn{robExt}{underscore in name}
    }
  } {
    \str_if_in:nnTF {#1}{__}{}{
      \msg_warning:nn{robExt}{recommend two underscores in placeholders}
    }
  }
}

%% Usage: \placeholderFromContent{MYTITLE}{My slide title}
%% MYTITLE will contain at the end "My slide title"
%% Warning: only LaTeX-friendly code should be placed here, as LaTeX does not preserve some symbols and adds spaces
%% Tested!
%% The star version does NOT import the placeholder (useful for efficiency reasons)
\NewDocumentCommand{\robExtPlaceholderFromContent}{sm+m}{
  \robExt_set_hash_robust:cn { l__robExt_placeholder_#2_str } {#3}
  \checkIfPlaceholderNameIsLegal{#2}
  \ifdefined\robExtCompletelyDisableWholeImportSystem\else% Mostly for time optimizations...
    \IfBooleanTF {#1} {} {\robExtAddPlaceholderToList{#2}}%
  \fi
}
\let\placeholderFromContent\robExtPlaceholderFromContent
\let\robExtSetPlaceholder\robExtPlaceholderFromContent
\let\setPlaceholder\robExtPlaceholderFromContent


\NewDocumentCommand{\robExtPlaceholderFromContentFirst}{smm}{
  \robExt_set_hash_robust:cn { l__robExt_placeholder_#2_str } {#3}
  \checkIfPlaceholderNameIsLegal{#2}
  \IfBooleanTF {#1} {} {\robExtAddPlaceholderToListFirst{#2}}
}
\let\placeholderFromContentFirst\robExtPlaceholderFromContentFirst
\let\robExtSetPlaceholderFirst\robExtPlaceholderFromContentFirst
\let\setPlaceholderFirst\robExtPlaceholderFromContentFirst


\NewDocumentCommand{\robExtPlaceholderFromString}{smm}{
  \str_set_eq:cN { l__robExt_placeholder_#2_str } {#3}
  \checkIfPlaceholderNameIsLegal{#2}
  \IfBooleanTF {#1} {} {\robExtAddPlaceholderToList{#2}}
}
\let\placeholderFromString\robExtPlaceholderFromString
\let\setPlaceholderFromString\robExtPlaceholderFromString

\NewDocumentCommand{\robExtPlaceholderFromStringExpanded}{smm}{
  \str_set:cx { l__robExt_placeholder_#2_str } {#3}
  \checkIfPlaceholderNameIsLegal{#2}
  \IfBooleanTF {#1} {} {\robExtAddPlaceholderToList{#2}}
}
\let\placeholderFromStringExpanded\robExtPlaceholderFromStringExpanded
\let\setPlaceholderFromStringExpanded\robExtPlaceholderFromStringExpanded


\NewDocumentCommand{\robExtCopyPlaceholder}{smm}{
  \str_set_eq:cc { l__robExt_placeholder_#2_str } { l__robExt_placeholder_#3_str }
  \IfBooleanTF {#1} {} {\robExtAddPlaceholderToList{#2}}
}
\let\copyPlaceholder\robExtCopyPlaceholder


%% Add something to the right of the placeholder
%% By default it adds a space in between, unless we use the star version
\NewDocumentCommand{\robExtAddToPlaceholder}{smm}{
  \str_if_exist:cTF { l__robExt_placeholder_#2_str } {
    \robExt_set_hash_robust:Nn \l_tmp_str {#3} %% needed as put_right does not convert to string first
    \IfBooleanTF{#1}{}{\str_put_right:cn { l__robExt_placeholder_#2_str } { ~ } }
    \str_put_right:cV { l__robExt_placeholder_#2_str } \l_tmp_str
  }{
    \robExt_set_hash_robust:cn { l__robExt_placeholder_#2_str } {#3}
    \robExtAddPlaceholderToList{#2}
  }
}
\let\addToPlaceholder\robExtAddToPlaceholder

\NewDocumentCommand{\robExtAddToPlaceholderNoImport}{smm}{
  \str_if_exist:cTF { l__robExt_placeholder_#2_str } {
    \robExt_set_hash_robust:Nn \l_tmp_str {#3} %% needed as put_right does not convert to string first
    \IfBooleanTF{#1}{}{\str_put_right:cn { l__robExt_placeholder_#2_str } { ~ } }
    \str_put_right:cV { l__robExt_placeholder_#2_str } \l_tmp_str
  }{
    \robExt_set_hash_robust:cn { l__robExt_placeholder_#2_str } {#3}
  }
}
\let\addToPlaceholderNoImport\robExtAddToPlaceholderNoImport


%% Add something to the right of the placeholder
%% By default it adds a space in between, unless we use the star version
\NewDocumentCommand{\robExtAddBeforePlaceholder}{smm}{
  \str_if_exist:cTF { l__robExt_placeholder_#2_str } {
    \robExt_set_hash_robust:Nn \l_tmp_str {#3} %% needed as put_right does not convert to string first
    \IfBooleanTF{#1}{}{\str_put_left:cn { l__robExt_placeholder_#2_str } { ~ } }
    \str_put_left:cV { l__robExt_placeholder_#2_str } \l_tmp_str
  }{
    \robExt_set_hash_robust:cn { l__robExt_placeholder_#2_str } {#3}
    \robExtAddPlaceholderToList{#2}
  }
}
\let\addBeforePlaceholder\robExtAddBeforePlaceholder


\NewDocumentCommand{\robExtAddBeforePlaceholderNoImport}{smm}{
  \str_if_exist:cTF { l__robExt_placeholder_#2_str } {
    \robExt_set_hash_robust:Nn \l_tmp_str {#3} %% needed as put_right does not convert to string first
    \IfBooleanTF{#1}{}{\str_put_left:cn { l__robExt_placeholder_#2_str } { ~ } }
    \str_put_left:cV { l__robExt_placeholder_#2_str } \l_tmp_str
  }{
    \robExt_set_hash_robust:cn { l__robExt_placeholder_#2_str } {#3}
  }
}
\let\addBeforePlaceholderNoImport\robExtAddBeforePlaceholderNoImport


% Usage:
% \begin{placeholderFromCode}{HELPERFUNCTION}
%   def my_helper_function(bla):
%   return bla + 1
% \end{placeholderFromCode}
% HELPERFUNCTION will contain at the end "def ..."
% This environment cannot be placed inside any other macro/align/...
\NewDocumentEnvironment{RobExtPlaceholderFromCode}{sm}{%
  \checkIfPlaceholderNameIsLegal{#2}%
  % % debug part
  % \str_set:Nn \l_test_str {#1}
  % \show\l_test_str
  %% Environments can't use verbatim yet: https://github.com/latex3/latex3/issues/591
  % Might be related: https://tex.stackexchange.com/questions/633523/saving-the-body-of-an-environment-to-a-file-verbatim-using-xparse
  %% Here is the first part:
  %% https://tex.stackexchange.com/a/680259/116348 might work and be more efficient, but it might be less reliable
  %% and definitely more complicated and error prone. Instead, we write to a file and read the result.
  %% TODO: try to do it using verbatim, might be trivial or complicated, not sure, maybe see https://tex.stackexchange.com/questions/555359/reading-lines-verbatim-into-a-sequence-variable
  \XSIMfilewritestart{\jobname-robExt-tmp-file-you-can-remove.tmp}%
}{%
  \XSIMfilewritestop%
  \ior_open:Nn \g__robExt_read_ior {\jobname-robExt-tmp-file-you-can-remove.tmp}%
  %% Put the file in l__robExt_tmp_contain_file_str
  \str_clear_new:N \l__robExt_tmp_str%
  \ior_str_map_inline:Nn \g__robExt_read_ior {%
    \str_gput_right:Nx \l__robExt_tmp_str {\tl_to_str:n{##1}^^J}%
  }%
  \str_set_eq:cN {l__robExt_placeholder_#2_str} \l__robExt_tmp_str%
  \IfBooleanTF {#1} {} {
    \robExtAddPlaceholderToList{#2}
    %% Otherwise they will be lost when the environment ends
    \robExtKeepaftergroup{l__robExt_placeholder_group_main_seq}%
  }
  %% for other variable
  \robExtKeepaftergroup{l__robExt_placeholder_#2_str}%
}%
\let\PlaceholderFromCode\RobExtPlaceholderFromCode
\let\endPlaceholderFromCode\endRobExtPlaceholderFromCode
\let\SetPlaceholderCode\RobExtPlaceholderFromCode
\let\endSetPlaceholderCode\endRobExtPlaceholderFromCode

\NewDocumentCommand{\robExtKeepPlaceholderAfterGroup}{m}{
  \robExtKeepaftergroup{l__robExt_placeholder_#1_str}%
}
\let\keepPlaceholderAfterGroup\robExtKeepPlaceholderAfterGroup

%% Usage:
%% \placeholderPathFromFilename{MYLIBPATH}{mylib.py}
%% This will copy mylib.py in the cache, and set MYLIBPATH to the name of the file in the cache like
%% MYLIBPATH = robExt-abcmylib.py
%% Tested!
\NewDocumentCommand{\robExtPlaceholderPathFromFilename}{smm}{
  \ior_open:Nn \g__robExt_read_ior {#3}
  %% Put the file in l__robExt_tmp_contain_file_str
  \str_clear_new:N \l__robExt_tmp_contain_file_str
  \ior_str_map_inline:Nn \g__robExt_read_ior {
    \str_put_right:Nx \l__robExt_tmp_contain_file_str {\tl_to_str:n{##1}^^J}
  }
  %% computes the new filename based on the md5
  \str_set:Nx \l__robExt_tmp_filename_no_prefix_str {\robExt@pdfmdfivesum{\l__robExt_tmp_contain_file_str}#3}
  %% Writes the content to the file
  \robExtCheckIfPrefixFolderExists
  \iow_open:Nx \g__robExt_write_iow {\robExtAddCachePathAndName{\l__robExt_tmp_filename_no_prefix_str}}
  \iow_now:NV \g__robExt_write_iow \l__robExt_tmp_contain_file_str
  \iow_close:N \g__robExt_write_iow
  %% sets the template name to the relative path to the file
  \str_set:cx { l__robExt_placeholder_#2_str } {\robExtPrefixFilename\l__robExt_tmp_filename_no_prefix_str}
  \IfBooleanTF {#1} {} {\robExtAddPlaceholderToList{#2}}
}
\let\placeholderPathFromFilename\robExtPlaceholderPathFromFilename

%% Usage:
%% \placeholderFromFileContent{MYLIB}{mylib.py}
%% This will set MYLIB to the content of the file mylib.py
%% Tested!
\NewDocumentCommand{\robExtPlaceholderFromFileContent}{smm}{
  \checkIfPlaceholderNameIsLegal{#2}
  \ior_open:Nn \g__robExt_read_ior {#3}
  %% Put the file in l__robExt_tmp_contain_file_str
  \str_clear_new:N \l__robExt_tmp_contain_file_str
  \ior_str_map_inline:Nn \g__robExt_read_ior {
    \str_put_right:Nx \l__robExt_tmp_contain_file_str {\tl_to_str:n{##1}^^J}
  }
  %% sets the template name to the relative path to the file
  \str_set_eq:cN { l__robExt_placeholder_#2_str } \l__robExt_tmp_contain_file_str
  \IfBooleanTF {#1} {} {\robExtAddPlaceholderToList{#2}}
}
\let\placeholderFromFileContent\robExtPlaceholderFromFileContent


%% Usage:
%% \placeholderPathFromContent{MYLIBPATH}{some code}
%% This will copy "some code" in the cache, and set MYLIBPATH to the name of the file in the cache like
%% MYLIBPATH = robExt-abc.py
%% Tested!
\NewDocumentCommand{\robExtPlaceholderPathFromContent}{smO{.tex}m}{
  \robExt_set_hash_robust:Nn \l__robExt_tmp_contain_file_str {#4}
  %% computes the new filename based on the md5
  \str_set:Nx \l__robExt_tmp_filename_no_prefix_str {\robExt@pdfmdfivesum{\l__robExt_tmp_contain_file_str}#3}
  %% Writes the content to the file
  \robExtCheckIfPrefixFolderExists
  \iow_open:Nx \g__robExt_write_iow {\robExtAddCachePathAndName{\l__robExt_tmp_filename_no_prefix_str}}
  \iow_now:NV \g__robExt_write_iow \l__robExt_tmp_contain_file_str
  \iow_close:N \g__robExt_write_iow
  %% sets the template name to the relative path to the file
  \str_set:cx { l__robExt_placeholder_#2_str } {\robExtPrefixFilename\l__robExt_tmp_filename_no_prefix_str}
  \IfBooleanTF {#1} {} {\robExtAddPlaceholderToList{#2}}
}
\let\placeholderPathFromContent\robExtPlaceholderPathFromContent


%% Usage:
%% \begin{PlaceholderPathFromCode}{mylibpath}
%% def blabla():
%% \end{PlaceholderPathFromCode}
%% This will copy "some code" in the cache, and set MYLIBPATH to the name of the file in the cache like
%% MYLIBPATH = robExt-abc.py
\NewDocumentEnvironment{RobExtPlaceholderPathFromCode}{sO{}m}{
  \XSIMfilewritestart*{\jobname-robExt-tmp-file-you-can-remove.tmp}
}{
  \XSIMfilewritestop
  \ior_open:Nn \g__robExt_read_ior {\jobname-robExt-tmp-file-you-can-remove.tmp}
  %% Put the file in \l__robExt_tmp_contain_file_str
  \str_clear_new:N \l__robExt_tmp_contain_file_str
  \ior_str_map_inline:Nn \g__robExt_read_ior {
    \str_gput_right:Nx \l__robExt_tmp_contain_file_str {\tl_to_str:n{##1}^^J}
  }
  %% computes the new filename based on the md5
  \str_set:Nx \l__robExt_tmp_filename_no_prefix_str {\robExt@pdfmdfivesum{\l__robExt_tmp_contain_file_str}#2}
  %% Writes the content to the file
  \robExtCheckIfPrefixFolderExists
  \iow_open:Nx \g__robExt_write_iow {\robExtAddCachePathAndName{\l__robExt_tmp_filename_no_prefix_str}}
  \iow_now:NV \g__robExt_write_iow \l__robExt_tmp_contain_file_str
  \iow_close:N \g__robExt_write_iow
  %% sets the template name to the relative path to the file
  \str_set:cx { l__robExt_placeholder_#3_str } {\robExtPrefixFilename\l__robExt_tmp_filename_no_prefix_str}
  \IfBooleanTF {#1} {} {
    \robExtAddPlaceholderToList{#3}
    \robExtKeepaftergroup{l__robExt_placeholder_group_main_seq}
  }
  %% Otherwise they will be lost when the environment ends
  \robExtKeepaftergroup{l__robExt_placeholder_#3_str}
}
\let\PlaceholderPathFromCode\RobExtPlaceholderPathFromCode
\let\endPlaceholderPathFromCode\endRobExtPlaceholderPathFromCode

% %%% Evaluate a string by replacing the placeholders until there is none left
% %%% the result will be in \robExtResult
\cs_set:Nn \__robExt_replace_until_impossible: {
  \robExtDebugMessage{We~needed~to~run~replace_until_impossible}
  % Try to replace directly if single placeholder
  \str_if_exist:cTF { l__robExt_placeholder_ \l_robExt_result_str _str }{
    \str_set_eq:Nc {\l_robExt_result_str}{ l__robExt_placeholder_ \l_robExt_result_str _str }
    \str_set:Nn \l_robext_tmp_before {}
  } {
    \str_set_eq:NN \l_robext_tmp_before \l_robExt_result_str
    \seq_map_inline:Nn \l__robExt_placeholder_group_main_seq {
      \robExtDebugMessage{Trying ~ to ~ replace ##1^^J}
      \str_if_exist:cTF { l__robExt_placeholder_##1_str } {
        \str_replace_all:Nnv \l_robExt_result_str { ##1 } { l__robExt_placeholder_##1_str }
      }{
        \robExtDebugWarning{WARNING: ~ the ~ placeholder ~ ##1 ~ does ~ not ~ exist.}
      }    
    }
  }
  % In quick mode, we first try to check if the string contains __ since by default all placeholders start with __
  % If not, no need to do one more run
  \cs_if_exist:NTF {\robExtPlaceholderOnlyWithUnderscores} {
    \str_if_in:NnTF { \l_robExt_result_str } { __ } {
      \str_compare:eNeTF \l_robext_tmp_before = \l_robExt_result_str {
        % We converged
      }{
        % The strings are different: we restart
        %\message{The strings are different, we restart: It ~ used ~ to ~ be ~\l_robext_tmp_before^^J^^J now it is^^J^^J \l_robExt_result_str}
        \__robExt_replace_until_impossible:
      }
    }{
      % We found no __ so we stop before even if a change was made
    }
  }{
    % We compare the result
    \str_compare:eNeTF \l_robext_tmp_before = \l_robExt_result_str {
      % \str_set_eq:cN { l__robExt_placeholder_#1_str } \l_robExt_result_str
      % \robExtAddPlaceholderToList{#1}
    }{
      % The strings are different: we restart
      \__robExt_replace_until_impossible:
    }
  }
}

\NewDocumentCommand{\robExtGetPlaceholderInResult}{sO{}m}{
  \robExt_set_hash_robust:Nn \l_robExt_result_str { #3 }
  \robExtDebugMessage{Begin~to~evaluate~placeholder^^J~\l_robExt_result_str}
  \__robExt_replace_until_impossible:
  \robExtDebugMessage{Ended~the~replacement^^J~\l_robExt_result_str}
  \tl_if_blank:nTF {#2} {} {
    % To avoid infinite recursion later and allow concatenation to a placeholder that does not exists
    % we remove the name of the placeholder at the end
    \str_remove_all:Nn \l_robExt_result_str { #2 }
    \str_set_eq:cN { l__robExt_placeholder_#2_str } \l_robExt_result_str
    \IfBooleanTF {#1} {} {\robExtAddPlaceholderToList{#2}}
  }
}
\let\getPlaceholderInResult\robExtGetPlaceholderInResult

%% like robExtGetPlaceholderInResult put assumes that there is a single placeholder name.
%% It is used mainly for efficiency as it avoids a loop on all placeholder.
\NewDocumentCommand{\robExtGetPlaceholderInResultFromSinglePlaceholder}{sO{}m}{
  \cs_if_exist:cTF { l__robExt_placeholder_#3_str }{
    \str_set_eq:Nc { \l_robExt_result_str }{ l__robExt_placeholder_#3_str }
  }{
    \msg_error:nn{robExt}{placeholder not existing}
  }
  \robExtDebugMessage{Starting~to~replace~placeholder~#3~to~get~\l_robExt_result_str}
  \__robExt_replace_until_impossible:
  \robExtDebugMessage{Ended~up~with~\l_robExt_result_str}
  \tl_if_blank:nTF {#2} {} {
    % To avoid infinite recursion later and allow concatenation to a placeholder that does not exists
    % we remove the name of the placeholder at the end
    \str_remove_all:Nn \l_robExt_result_str { #2 }
    \str_set_eq:cN { l__robExt_placeholder_#2_str } \l_robExt_result_str
    \IfBooleanTF {#1} {} {\robExtAddPlaceholderToList{#2}}
  }
}

\cs_set:Nn \__replace_from_list:N {
  \str_set_eq:NN \l_robext_tmp_before \l_robExt_result_str
  \clist_map_inline:Nn {#1} {
    \str_if_exist:cTF { l__robExt_placeholder_##1_str }{
      \str_replace_all:Nnv \l_robExt_result_str { ##1 } { l__robExt_placeholder_##1_str }
    } { }
  }
  % We compare the result
  \str_compare:eNeTF \l_robext_tmp_before = \l_robExt_result_str {
    % \str_set_eq:cN { l__robExt_placeholder_#1_str } \l_robExt_result_str
    % \robExtAddPlaceholderToList{#1}
  }{
    % The strings are different: we restart
    \__replace_from_list:N { #1 }
  }
}

\NewDocumentCommand{\robExtGetPlaceholderInResultReplaceFromList}{smO{}m}{
  \robExt_set_hash_robust:Nn \l_robExt_result_str { #4 }
  \clist_set:Nn \l__robExt_list_placeholder_replace_clist {#2}
  \__replace_from_list:N \l__robExt_list_placeholder_replace_clist
  \tl_if_blank:nTF {#3} {} {
    \str_remove_all:Nn \l_robExt_result_str { #3 }
    \str_set_eq:cN { l__robExt_placeholder_#3_str } \l_robExt_result_str
    \IfBooleanTF {#1} {} {\robExtAddPlaceholderToList{#3}}
  }
}
\let\getPlaceholderInResultReplaceFromList\robExtGetPlaceholderInResultReplaceFromList


% Otherwise we need latex3 
\def\robExtResult{\l_robExt_result_str}

%%%% Same version, but replaces only one (useful sometimes)
\cs_set:Nn \__replace_n_times:n {
  \str_set_eq:NN \l_robext_tmp_before \l_robExt_result_str
  \seq_map_inline:Nn \l__robExt_placeholder_group_main_seq {
    \str_replace_all:Nnv \l_robExt_result_str { ##1 } { l__robExt_placeholder_##1_str }
  }
  \int_compare:nTF { #1 > 1}{
    \__replace_n_times:n {\int_eval:n {#1-1}}
  }{}
}


\NewDocumentCommand{\robExtSetPlaceholderRec}{smm}{
  \IfBooleanTF {#1} {
    \robExtGetPlaceholderInResult*[#2]{#3}
  } {
    \robExtGetPlaceholderInResult[#2]{#3}
  }
}
\let\setPlaceholderRec\robExtSetPlaceholderRec

\NewDocumentCommand{\robExtSetPlaceholderRecReplaceFromList}{smmm}{
  \IfBooleanTF {#1} {
    \robExtGetPlaceholderInResultReplaceFromList*{#2}[#3]{#4}
  }{
    \robExtGetPlaceholderInResultReplaceFromList{#2}[#3]{#4}
  }
}
\let\setPlaceholderRecReplaceFromList\robExtSetPlaceholderRecReplaceFromList


\NewDocumentCommand{\robExtGetPlaceholder}{O{}m}{
  \robExtGetPlaceholderInResult[#1]{#2}
  \l_robExt_result_str
}
\let\getPlaceholder\robExtGetPlaceholder

%%% Evaluate a string by replacing the placeholders until there is none left
%%% the result will be in \robExtResult
\NewDocumentCommand{\robExtEvalPlaceholder}{m}{
  \robExt_set_hash_robust:Nn \l_test_str {#1}
  \robExtGetPlaceholderInResult{#1}
  \tl_rescan:nv {} { l_robExt_result_str }
}
\let\evalPlaceholder\robExtEvalPlaceholder

%%% Evaluate a string by replacing the placeholders until there is none left
%%% the result will be in \robExtResult
\NewDocumentCommand{\robExtEvalPlaceholderReplaceFromList}{mm}{
  \robExt_set_hash_robust:Nn \l_test_str {#2}
  \robExtGetPlaceholderInResultReplaceFromList{#1}{#2}
  \tl_rescan:nv {} { l_robExt_result_str }
}
\let\evalPlaceholderReplaceFromList\robExtEvalPlaceholderReplaceFromList

%%% Evaluate a placeholder by first trying some strings, then checking if a template is still present (if the
%%% mode is enabled), and then it continues normally. Mostly for performance reasons.
\NewDocumentCommand{\robExtEvalPlaceholderStartFromList}{mm}{
  \robExt_set_hash_robust:Nn \l_test_str {#2}
  \robExtGetPlaceholderInResultReplaceFromList{#1}{#2}
  \cs_if_exist:NTF {\robExtPlaceholderOnlyWithUnderscores} {
    %% there might be non-important placeholders that are replaced later:
    \str_set_eq:NN \l__robExt_result_minus_str \l_robExt_result_str
    \str_remove_all:Nn \l__robExt_result_minus_str {__ROBEXT_OUTPUT_PREFIX__}
    \str_remove_all:Nn \l__robExt_result_minus_str {__ROBEXT_SOURCE_FILE__}
    \str_remove_all:Nn \l__robExt_result_minus_str {__ROBEXT_OUTPUT_PDF__}
    \str_remove_all:Nn \l__robExt_result_minus_str {__ROBEXT_WAY_BACK__}
    \str_remove_all:Nn \l__robExt_result_minus_str {__ROBEXT_CACHE_FOLDER__}
    \str_if_in:NnTF { \l__robExt_result_minus_str } { __ } {
      \__robExt_replace_until_impossible:
    }{
    }
  }{ \__robExt_replace_until_impossible: }
  \tl_rescan:nv {} { l_robExt_result_str }
}
\let\evalPlaceholderStartFromList\robExtEvalPlaceholderStartFromList


\NewDocumentCommand{\robExtEvalPlaceholderInplace}{m}{
  \robExtGetPlaceholderInResult{#1}
  \tl_set_rescan:Nnx \l__robExt_tmp_tl  {} { \l_robExt_result_str }
  \robExt_set_hash_robust:cx { l__robExt_placeholder_#1_str } { \l__robExt_tmp_tl }
}
\let\evalPlaceholderInplace\robExtEvalPlaceholderInplace

\NewDocumentCommand{\robExtEvalPlaceholderInplaceFromSinglePlaceholder}{m}{
  \robExtGetPlaceholderInResultFromSinglePlaceholder{#1}
  \tl_set_rescan:Nnx \l__robExt_tmp_tl  {} { \l_robExt_result_str }
  \robExt_set_hash_robust:cx { l__robExt_placeholder_#1_str } { \l__robExt_tmp_tl }
}
\let\evalPlaceholderInplaceFromSinglePlaceholder\robExtEvalPlaceholderInplaceFromSinglePlaceholder


% Sometimes two symbols ## (with a different catcode than a normal hash) are added instead of a single #
\NewDocumentCommand{\robExtPlaceholderDoubleNumberHashesInplace}{m}{
  \str_replace_all:cxx { l__robExt_placeholder_#1_str } { \c_hash_str } { \c_hash_str \c_hash_str }
}
\let\placeholderDoubleNumberHashesInplace\robExtPlaceholderDoubleNumberHashesInplace

\NewDocumentCommand{\robExtPlaceholderHalveNumberHashesInplace}{m}{
  \str_replace_all:cxx { l__robExt_placeholder_#1_str } { \c_hash_str \c_hash_str } { \c_hash_str }
}
\let\placeholderHalveNumberHashesInplace\robExtPlaceholderHalveNumberHashesInplace

\NewDocumentCommand{\robExtPlaceholderReplaceInplace}{mmm}{
  \str_replace_all:cnn { l__robExt_placeholder_#1_str } { #2 } { #3 }
}
\let\placeholderReplaceInplace\robExtPlaceholderReplaceInplace

\NewDocumentCommand{\robExtPlaceholderReplaceInplaceEval}{mmm}{
  \str_replace_all:cxx { l__robExt_placeholder_#1_str } { #2 } { #3 }
}
\let\placeholderReplaceInplaceEval\robExtPlaceholderReplaceInplaceEval

%%%%%%%%%%%%%%%%%%%%%%%%%%%%%%%%%%%%%%%%%%%%
%%%%%%%%%%% Placeholders groups %%%%%%%%%%%%
%%%%%%%%%%%%%%%%%%%%%%%%%%%%%%%%%%%%%%%%%%%%
% I used to put all placeholders into the same group, but it was quite inefficient (latex is really slow)
% Then I tried to hide this group but keep a single list of all items, but still automatically add items to it:
% also quite slow, and the list was not so useful anyway.
% In practice, however, we do not need to put the placeholders created by default in a group: we can just store them
% in a standalone way, and load them when needed. So now I create the concept of group: people can add placeholders
% to a group of placeholders to load them later if needed (e.g. to list all placeholders created by this library).
% The group named "main" will be the main one with imported groups, where stuff like "set placeholder rec" will search for the
% existing placeholders.
% The group named "preexisting" will contain the placeholders created by this library (useful mostly for
% debugging purpose).

\NewDocumentCommand{\robExtAddPlaceholdersToGroup}{mm}{
  % Make sure we have a string here:
  \robExt_set_hash_robust:Nn \l__robExt_tmp_str { #2 }
  \seq_set_from_clist:NN \l__robExt_tmp_seq \l__robExt_tmp_str
  \cs_if_exist:cTF {l__robExt_placeholder_group_#1_seq}{}{ \seq_clear_new:c {l__robExt_placeholder_group_#1_seq}}
  % First we remove existing occurrences (useful to avoid listing the same macro more than once
  % if we redefine a macro):
  \seq_map_inline:Nn \l__robExt_tmp_seq {
    \robExtDebugMessage{Adding ##1 to #1}
    \seq_remove_all:cn {l__robExt_placeholder_group_#1_seq}  {##1}
    \seq_put_right:cn {l__robExt_placeholder_group_#1_seq} {##1}
  }
}
\let\addPlaceholdersToGroup\robExtAddPlaceholdersToGroup

\NewDocumentCommand{\robExtEnsureGroupPlaceholdersExists}{m}{
  \cs_if_exist:cTF {l__robExt_placeholder_group_#1_seq}{}{ \seq_clear_new:c {l__robExt_placeholder_group_#1_seq}}
}
\let\ensureGroupPlaceholdersExists\robExtEnsureGroupPlaceholdersExists

\NewDocumentCommand{\robExtAddPlaceholdersToGroupBefore}{mm}{
  % Make sure we have a string here:
  \robExt_set_hash_robust:Nn \l__robExt_tmp_str { #2 }
  \seq_set_from_clist:NN \l__robExt_tmp_seq \l__robExt_tmp_str
  \seq_reverse:N \l__robExt_tmp_seq
  \cs_if_exist:cTF {l__robExt_placeholder_group_#1_seq}{}{ \seq_clear_new:c {l__robExt_placeholder_group_#1_seq}}
  % First we remove existing occurrences (useful to avoid listing the same macro more than once
  % if we redefine a macro):
  \seq_map_inline:Nn \l__robExt_tmp_seq {
    \seq_remove_all:cn {l__robExt_placeholder_group_#1_seq}  {##1}
    \seq_put_left:cn {l__robExt_placeholder_group_#1_seq} {##1}
  }
}
\let\addPlaceholdersToGroupBefore\robExtAddPlaceholdersToGroupBefore

\NewDocumentCommand{\robExtRemovePlaceholdersFromGroup}{mm}{
  % Make sure we have a string here:
  \robExt_set_hash_robust:Nn \l__robExt_tmp_str { #2 }
  \seq_set_from_clist:NN \l__robExt_tmp_seq \l__robExt_tmp_str
  \seq_reverse:N \l__robExt_tmp_seq
  \cs_if_exist:cTF {l__robExt_placeholder_group_#1_seq}{}{ \seq_clear_new:c {l__robExt_placeholder_group_#1_seq}}
  % First we remove existing occurrences (useful to avoid listing the same macro more than once
  % if we redefine a macro):
  \seq_map_inline:Nn \l__robExt_tmp_seq {
    \seq_remove_all:cn {l__robExt_placeholder_group_#1_seq}  {##1}
  }
}
\let\removePlaceholdersFromGroup\robExtRemovePlaceholdersFromGroup
\let\removePlaceholderFromGroup\robExtRemovePlaceholdersFromGroup

\NewDocumentCommand{\robExtClearGroupPlaceholders}{m}{
  \seq_clear_new:c {l__robExt_placeholder_group_#1_seq}
}
\let\clearGroupPlaceholders\robExtClearGroupPlaceholders

\NewDocumentCommand{\robExtCopyGroupPlaceholders}{mm}{
  \seq_set_eq:cc {l__robExt_placeholder_group_#1_seq} {l__robExt_placeholder_group_#2_seq}
}
\let\copyGroupPlaceholders\robExtCopyGroupPlaceholders

\NewDocumentCommand{\robExtAppendGroupPlaceholders}{mm}{
  \seq_map_inline:cn {l__robExt_placeholder_group_#2_seq} {
    \seq_remove_all:cn {l__robExt_placeholder_group_#1_seq}  {##1}
    \seq_put_right:cn {l__robExt_placeholder_group_#1_seq} {##1}
  } 
}
\let\appendGroupPlaceholders\robExtAppendGroupPlaceholders

\NewDocumentCommand{\robExtAppendBeforeGroupPlaceholders}{mm}{
  \seq_eq:Nc \l__robExt_tmp_seq {l__robExt_placeholder_group_#2_seq}
  \seq_reverse:N \l__robExt_tmp_seq
  \seq_map_inline:Nn \l__robExt_tmp_seq {
    \seq_remove_all:cn {l__robExt_placeholder_group_#1_seq}  {##1}
    \seq_put_left:cn {l__robExt_placeholder_group_#1_seq} {##1}
  }
}
\let\appendBeforeGroupPlaceholders\robExtAppendBeforeGroupPlaceholders

\NewDocumentCommand{\robExtImportPlaceholdersFromGroup}{m}{
  \robExtAppendGroupPlaceholders{main}{#1}
}
\let\importPlaceholdersFromGroup\robExtImportPlaceholdersFromGroup

\NewDocumentCommand{\robExtImportAllPlaceholders}{}{
  \seq_map_inline:Nn \l__robExt_list_groups_seq {
    \robExtImportPlaceholdersFromGroup{##1}
  }
}
\let\importAllPlaceholders\robExtImportAllPlaceholders
%%%%
%%%% List groups, mostly for debugging purpose
%%%%

\seq_new:N \l__robExt_list_groups_seq % contains the list of all groups

\NewDocumentCommand{\robExtRegisterGroupPlaceholders}{m}{
  \seq_if_exist:cTF {l__robExt_placeholder_group_#1_seq}{}{
    \seq_clear_new:c {l__robExt_placeholder_group_#1_seq}
  }
  \seq_put_right:Nn \l__robExt_list_groups_seq {#1}
}
\let\registerGroupPlaceholders\robExtRegisterGroupPlaceholders
\let\newGroupPlaceholders\robExtRegisterGroupPlaceholders

\NewDocumentCommand{\robExtShowAllRegisteredGroupsAndPlaceholders}{s}{%
  \seq_map_inline:Nn \l__robExt_list_groups_seq {%
    \message{^^J- Group ##1: ^^J}%
    \seq_map_inline:cn {l__robExt_placeholder_group_##1_seq} {%
      \message{, ####1 }%
      \IfBooleanTF{#1}{
        \message{Content:\use:c{l__robExt_placeholder_####1_str}}
      }{}
    }%
    \message{^^J}
  }
  \show\def
}
\let\showAllRegisteredGroupsAndPlaceholders\robExtShowAllRegisteredGroupsAndPlaceholders

\NewDocumentCommand{\robExtPrintAllRegisteredGroups}{s}{%
  \seq_map_inline:Nn \l__robExt_list_groups_seq {%
    \texttt{##1}\par
  }
}
\let\printAllRegisteredGroups\robExtPrintAllRegisteredGroups

\NewDocumentCommand{\robExtPrintAllRegisteredGroupsAndPlaceholders}{s}{%
  \seq_map_inline:Nn \l__robExt_list_groups_seq {%
    - ~ Group \texttt{##1}:\par
    \seq_map_inline:cn {l__robExt_placeholder_group_##1_seq} {%
      Placeholder \texttt{####1}%
      \IfBooleanTF {#1}{\robExtPrintPlaceholderNoReplacement{####1}}{\par}
    }%
  }
}
\let\printAllRegisteredGroupsAndPlaceholders\robExtPrintAllRegisteredGroupsAndPlaceholders


% The star version displays the content as well
\NewDocumentCommand{\robExtPrintGroupPlaceholders}{sm}{%
  Content~ of~ group~ \texttt{#2}:\par%
  \seq_map_inline:cn {l__robExt_placeholder_group_#2_seq} {%
    -~Placeholder ~ \texttt{##1}\par \IfBooleanTF {#1}{\robExtPrintPlaceholderNoReplacement{##1}}{}
  }%
}
\let\printGroupPlaceholders\robExtPrintGroupPlaceholders


%%%%%%%%%%%%%%%%%%%%%%%%%%%%%%%%%%%%%
%%%%%%%%%%% Dependencies %%%%%%%%%%%%
%%%%%%%%%%%%%%%%%%%%%%%%%%%%%%%%%%%%%
\NewDocumentCommand{\robExtResetDependencies}{m}{
  \seq_clear:N \l__robExt_dependencies_str
}

\NewDocumentCommand{\robExtAddDependency}{m}{
  \seq_put_left:Nx \l__robExt_dependencies_str {#1}
}

\NewDocumentCommand{\robExtDebugDependency}{}{
  \show\l__robExt_dependencies_str
}

% Useful when compiling commands
\NewDocumentCommand{\robExtSaveDependencies}{m}{
  \seq_gset_eq:cN {#1} \l__robExt_dependencies_str
}
\NewDocumentCommand{\robExtRestoreDependencies}{m}{
  \seq_set_eq:Nc \l__robExt_dependencies_str {#1}
}


%%%%%%%%%%%%%%%%%%%%%%%%%%%%%%%%%%%%%%%%
%%%%%%%%%%% Externalization %%%%%%%%%%%%
%%%%%%%%%%%%%%%%%%%%%%%%%%%%%%%%%%%%%%%%
%% The idea is to populate a placeholder called __ROBEXT_TEMPLATE__ that will contain the file to generate
%% together with a placeholder called __ROBEXT_COMPILATION_COMMAND__ that will contain the command to compile the file

\NewDocumentCommand{\robExtSetCompilationCommand}{m}{
  \robExtSetPlaceholderFirst{__ROBEXT_COMPILATION_COMMAND__} {#1}
}

\NewDocumentCommand{\robExtAddArgumentToCompilationCommand}{m}{
  \robExtSetPlaceholderRec{__ROBEXT_COMPILATION_COMMAND__} {__ROBEXT_COMPILATION_COMMAND__ ~ "#1"}
}


%% Alias of robExtFinalFile to \robExtSourceFile, as I don't like anymore the name I chose
%\def\robExtSourceFile{\robExtFinalFile}

% First is name of sequence for only try (we only replace those and stop), second is name of sequence for first try (we first try with the sequence and
% then do a more normal thing), third is the name of the placeholder
\cs_set:Nn \try_to_evaluate_single_placeholder_without_going_to_replace_until_impossible:NNn {
  %%%%% Take care of the compilation command: might seem a bit weird and convoluted, but this is needed to get really good efficiency
  %%%%% since \__robExt_replace_until_impossible: is quite costly, we try to avoid it
  \cs_if_exist:NTF {#1} { % We only replace fixed state of placeholders
    \str_set_eq:Nc \l_robExt_result_str {l__robExt_placeholder_#3_str}
    \seq_map_inline:Nn {#1} {
      \robExtDebugMessage{replacing~##1~in~\l_robExt_result_str}
      \str_replace_all:Nnv \l_robExt_result_str {##1} {l__robExt_placeholder_##1_str}
    }
    \robExtDebugMessage{Cool,~we~avoided~the~costly~replace~(only~try).}
  }{
    \cs_if_exist:NTF {#2} { % We start by replacing a fixed state of placeholders
      \str_set_eq:Nc \l_robExt_result_str {l__robExt_placeholder_#3_str}
      \robExtDebugMessage{We~start~by~replacing~a~fixed~state~of~placeholders}
      \seq_map_inline:Nn {#2} {
        \str_if_exist:cTF {l__robExt_placeholder_##1_str} {
          \robExtDebugMessage{Replacing~##1~into~\use:c{l__robExt_placeholder_##1_str}}%
          \str_replace_all:Nnv \l_robExt_result_str {##1} {l__robExt_placeholder_##1_str}
        } {
          \robExtDebugMessage{Warning: ~ trying~to~replace~##1~that~does~not~exist.}
        }
      }
      %\show\l_robExt_result_str
      %% Check if there is some more stuff to replace (possible only if we consider only templates with underscores
      \cs_if_exist:NTF {\robExtPlaceholderOnlyWithUnderscores} {
        % all templates have underscores
        % first check if some non-important elements are here: for this we first remove them
        \str_set_eq:NN \l__robExt_result_minus_str \l_robExt_result_str
        \str_remove_all:Nn \l__robExt_result_minus_str {__ROBEXT_OUTPUT_PREFIX__}
        \str_remove_all:Nn \l__robExt_result_minus_str {__ROBEXT_SOURCE_FILE__}
        \str_remove_all:Nn \l__robExt_result_minus_str {__ROBEXT_OUTPUT_PDF__}
        \str_remove_all:Nn \l__robExt_result_minus_str {__ROBEXT_WAY_BACK__}
        \str_remove_all:Nn \l__robExt_result_minus_str {__ROBEXT_CACHE_FOLDER__}
        % Now, we check if some __ are left:
        \str_if_in:NnTF { \l__robExt_result_minus_str } { __ } {
          % Some templates are left
          \robExtDebugMessage{Still~some~templates~are~left.}
          \__robExt_replace_until_impossible:
        } {
          % No more template: the end
          \robExtDebugMessage{Cool,~we~avoided~the~costly~replace: \l_robExt_result_str.}
        }
      }{
        % no way to know, must do it anyway
        \robExtDebugMessage{Seems~like~not~all~templates~contain~underscore.~Too~bad,~it~will~be~slower}
        \__robExt_replace_until_impossible:
      }
    }{
      \robExtDebugMessage{No~advice~on~how~to~start:~doing~it~the~long~way}
      \robExtGetPlaceholderInResultFromSinglePlaceholder{#3}
    }
  }
}

\NewDocumentCommand{\robExtOnlyPlaceholdersInCompilationCommand}{m}{
  \seq_set_from_clist:Nn \robExtOnlyTryThisSequenceOnCompilationCommand {#1}
}
\let\onlyPlaceholdersInCompilationCommand\robExtOnlyPlaceholdersInCompilationCommand

\NewDocumentCommand{\robExtFirstPlaceholdersInCompilationCommand}{m}{
  \seq_set_from_clist:Nn \robExtFirstTryThisSequenceOnCompilationCommand {#1}
}
\let\firstPlaceholdersInCompilationCommand\robExtFirstPlaceholdersInCompilationCommand

\NewDocumentCommand{\robExtOnlyPlaceholdersInTemplate}{m}{
  \seq_set_from_clist:Nn \robExtOnlyTryThisSequenceOnTemplate {#1}
}
\let\onlyPlaceholdersInTemplate\robExtOnlyPlaceholdersInTemplate

\NewDocumentCommand{\robExtFirstPlaceholdersInTemplate}{m}{
  \seq_set_from_clist:Nn \robExtFirstTryThisSequenceOnTemplate {#1}
}
\let\firstPlaceholdersInTemplate\robExtFirstPlaceholdersInTemplate

% for use outside of expl3 (pgf)
\def\robExtCurrentCompilationCommand{\l__robExt_current_compilation_command_str}
\NewDocumentCommand{\robExtGetNearlyFinalValueTemplateAndCompilationCommand}{}{
  %% oututs the compilation in \l__robExt_current_compilation_command_str and the template in \l_robExt_result_str
  %% This is for efficiency reasons: we only allow a few placeholders:
  %% In compilation command:
  %% 
  %% In template:
  %% - __ROBEXT_MAIN_CONTENT__
  %% - __ROBEXT_MAIN_CONTENT_ORIG__
  %% - __ROBEXT_LATEX_PREAMBLE__
  %% - __ROBEXT_OUTPUT_PREFIX__
  %% - __ROBEXT_LATEX_WRITE_DEPTH_TO_OUT_FILE__
  %% - __ROBEXT_LATEX_CREATE_OUT_FILE__
  %% Others
  %% - __ROBEXT_SOURCE_FILE__
  %% - __ROBEXT_OUTPUT_PDF__
  %% - __ROBEXT_WAY_BACK__
  %% - __ROBEXT_CACHE_FOLDER__
  \ifdefined\robExtDisablePlaceholders%
    \robExtDebugMessage{You~have~disabled~placeholders~(this~is~likely~a~compiled~template)}%
    \str_set_eq:NN \l__robExt_current_compilation_command_str \l__robExt_placeholder___ROBEXT_COMPILATION_COMMAND___str 
    \str_set_eq:NN \l_robExt_result_str \l__robExt_placeholder___ROBEXT_TEMPLATE___str%
    \str_replace_all:Nnv \l_robExt_result_str {__ROBEXT_MAIN_CONTENT__} {l__robExt_placeholder___ROBEXT_MAIN_CONTENT___str}
    \str_replace_all:Nnv \l_robExt_result_str {__ROBEXT_LATEX_PREAMBLE__} {l__robExt_placeholder___ROBEXT_LATEX_PREAMBLE___str}
    \str_replace_all:Nnv \l_robExt_result_str {__ROBEXT_MAIN_CONTENT_ORIG__} {l__robExt_placeholder___ROBEXT_MAIN_CONTENT_ORIG___str}
  \else
    \ifdefined\robExtEnableImportMechanism\else\robExtImportAllPlaceholders\fi%
    %%%%% Take care of the compilation command: might seem a bit weird and convoluted, but this is needed to get really good efficiency
    %%%%% since \__robExt_replace_until_impossible: is quite costly, we try to avoid it
    \try_to_evaluate_single_placeholder_without_going_to_replace_until_impossible:NNn \robExtOnlyTryThisSequenceOnCompilationCommand \robExtFirstTryThisSequenceOnCompilationCommand {__ROBEXT_COMPILATION_COMMAND__}
    \str_set_eq:NN \l__robExt_current_compilation_command_str \l_robExt_result_str
    %%%%% Take care of 
    \try_to_evaluate_single_placeholder_without_going_to_replace_until_impossible:NNn \robExtOnlyTryThisSequenceOnTemplate \robExtFirstTryThisSequenceOnTemplate {__ROBEXT_TEMPLATE__}
  \fi
}

\NewDocumentCommand{\robExtSetBackCompilationCommandAndTemplate}{}{
  \robExtPlaceholderFromString{__ROBEXT_COMPILATION_COMMAND__}{\l__robExt_current_compilation_command_str}%
  \robExtPlaceholderFromString{__ROBEXT_TEMPLATE__}{\l_robExt_result_str}%
}

\NewDocumentCommand{\robExtWriteFile}{m}{
  %%% First we get all dependencies stored in \l__robExt_dependencies_str to create a csv-like file:
  \str_clear:N \l__robExt_dependencies_mdfive_str
  \robExtGetNearlyFinalValueTemplateAndCompilationCommand
  % We first add on the first line the compilation command, and on the second line the template file.
  \str_set:Nx \l__robExt_dependencies_mdfive_str {command,\l__robExt_current_compilation_command_str^^J\robExt@pdfmdfivesum{\l_robExt_result_str ^^J},^^J} %% ^^J is a newline: LaTeX will automatically add a new line when writing the file
  \seq_map_inline:Nn \l__robExt_dependencies_str {
    \str_put_right:Nx \l__robExt_dependencies_mdfive_str {\file_mdfive_hash:n{##1},##1^^J} %% ^^J is a newline
  }
  %%
  %% Compute the final hash (the hash of all dependencies, including the current picture that is on the first line):
  %% The last newline is needed as the write operation automatically adds a newline.
  \tl_set:Nx \robExtFinalHash {\robExt@pdfmdfivesum{\l__robExt_dependencies_mdfive_str^^J}}
  % Might be useful to clean the previously compiled files, or change the MD5 sum, for instance if we do
  % not want to recompile when the file changes
  \robExtUseHookJustBeforeWritingFiles%
  %% We add the figure in the list of files.
  \iow_now:Nx \g__robExt_write_list_all_figures_iow {\robExtAddPrefixName{\robExtFinalHash.tex}}
  %% We can now set the placeholders, and recompute the final value of the file:
  %% I was doing before \robExtPlaceholderFromContent + \robExtEvalPlaceholderInplace, but it is too slow
  \str_set:Nx \l_tmp_output_prefix_str {\robExtAddPrefixName{\robExtFinalHash}}
  %%% No need to re-evaluate from scratch the string, 
  \str_replace_all:Nnx \l_robExt_result_str {__ROBEXT_OUTPUT_PREFIX__}{\l_tmp_output_prefix_str}
  \str_replace_all:Nnx \l_robExt_result_str {__ROBEXT_SOURCE_FILE__}{\l_tmp_output_prefix_str .tex}
  \str_replace_all:Nnx \l_robExt_result_str {__ROBEXT_OUTPUT_PDF__}{\l_tmp_output_prefix_str .pdf}
  \str_replace_all:Nnx \l_robExt_result_str {__ROBEXT_WAY_BACK__}{\robExtCacheFolderWayBack}
  \str_replace_all:Nnx \l_robExt_result_str {__ROBEXT_CACHE_FOLDER__}{\robExtCacheFolder}
  \file_if_exist:xTF{\robExtAddCachePathAndName{\robExtFinalHash.tex}}{
    \robExtDebugInfo{The\space file\space \robExtAddCachePathAndName{\robExtFinalHash.tex} \space already\space exists.^^J}
  }{
    \str_if_eq:VnTF { \l_robExt_result_str }{__ROBEXT_TEMPLATE__} {
      \msg_error:nn{robExt}{forgot template}
    }{
      % Check if the output directory exists
      \robExtCheckIfPrefixFolderExists
      \iow_open:Nx \g__robExt_write_iow {\robExtAddCachePathAndName{\robExtFinalHash.deps}}
      \iow_now:NV \g__robExt_write_iow \l__robExt_dependencies_mdfive_str
      \iow_close:N \g__robExt_write_iow
      %% Save the final file:
      \iow_open:Nx \g__robExt_write_iow {\robExtAddCachePathAndName{\robExtFinalHash.tex}}
      \iow_now:NV \g__robExt_write_iow \l_robExt_result_str
      \iow_close:N \g__robExt_write_iow
      \robExtDebugInfo{Source ~ saved ~ in ~ \robExtAddCachePathAndName{\robExtFinalHash.tex}.}
    }
  }
}

% https://tex.stackexchange.com/questions/133324/shell-escape-with-latex-3
% We need shell escape to work (but it's enabled by default on overleaf!)
% Think about the number of compilations.
\NewDocumentCommand{\robExtCompileFile}{m}{
  \file_if_exist:xTF{\robExtAddCachePathAndName{\robExtFinalHash.pdf}}{
    \cs_if_exist:NTF {\robExtForceRecompilation}{
      % We cannot do an OR with file_if_exist since it is not expandable, hence this trick
      \let\l__robExt_do_not_compile\undefined
    }{
      \def\l__robExt_do_not_compile{}
    }
  }{
    \let\l__robExt_do_not_compile\undefined
  }
  \cs_if_exist:NTF {\l__robExt_do_not_compile}{
    \robExtDebugInfo{No ~ need ~ to ~ recompile ~ \robExtAddCachePathAndName{\robExtFinalHash.pdf}^^J}
  }{
    % Used to know later if we have run a compilation command at that step or not
    \cs_undefine:N \l__robExt_I_just_ran_a_compilation_command:
    % No need to re-evaluate it
    % \robExtGetPlaceholderInResultFromSinglePlaceholder{__ROBEXT_COMPILATION_COMMAND__}
    \str_replace_all:Nnx \l__robExt_current_compilation_command_str {__ROBEXT_OUTPUT_PREFIX__}{\l_tmp_output_prefix_str}
    \str_replace_all:Nnx \l__robExt_current_compilation_command_str {__ROBEXT_SOURCE_FILE__}{\l_tmp_output_prefix_str .tex}
    \str_replace_all:Nnx \l__robExt_current_compilation_command_str {__ROBEXT_OUTPUT_PDF__}{\l_tmp_output_prefix_str .pdf}
    \str_replace_all:Nnx \l__robExt_current_compilation_command_str {__ROBEXT_WAY_BACK__}{\robExtCacheFolderWayBack}
    \str_replace_all:Nnx \l__robExt_current_compilation_command_str {__ROBEXT_CACHE_FOLDER__}{\robExtCacheFolder}
    % Make sure this command is run from the cache folder
    \ifdefined\robExtCacheFolder
      \str_put_left:Nx \l__robExt_current_compilation_command_str {cd ~ \robExtCacheFolder \space && ~ }
    \fi%
    %%% We enable manual mode if we enabled "compile in parallel after=N" and we compiled more than N elements
    % Check if we want to run stuff in parallel
    \ifdefined\robExt@compile@parallel@after
      % TODO: ****
      \int_gset:Nn \g__robExt_number_figures_just_compiled_or_to_compile {\g__robExt_number_figures_just_compiled_or_to_compile + 1}
      % We check if the number of figures that we already compiled is large enough to start parallel compilation
      \int_compare:nNnTF {
        \g__robExt_number_figures_just_compiled_or_to_compile} > {\robExt@compile@parallel@after
      } {
        \def\robExtManualMode
      } {}
    \fi
    \ifdefined\robExtManualMode
      \message{[robExt] Manual mode enabled: please, manually compile the images using \l__robExt_current_compilation_command_str or run 'bash \jobname-\robExtAddPrefixName{compile-missing-figures.sh}'.}
      % Avoid writing the same command multiple times (might occur in environments that execute the
      % same command multiple times like align etc)
      \cs_if_exist:cTF {l__robExt_missing_figure_\l__robExt_current_compilation_command_str :}{}{
        \iow_now:Nx \g__robExt_write_manually_compile_all_missing_figures_iow {\l__robExt_current_compilation_command_str}% a new line is automatically added
        \cs_gset:cn {l__robExt_missing_figure_\l__robExt_current_compilation_command_str :} {}
        \ifdefined\robExt@compile@parallel@after
          \gdef\robExt@compile@parallel@must@compile{}
          \seq_gput_right:Nx \g__robExt_files_compiled_in_parallel {\robExtAddCachePathAndName{\robExtFinalHash}}
          % Gets the line with the error for easier debugging
          \cs_gset:cx {g__robExt_lines_error_\robExtAddCachePathAndName{\robExtFinalHash}:} {\msg_line_number:}
          \cs_gset:cx {g__robExt_compilation_command_\robExtAddCachePathAndName{\robExtFinalHash}:} {\l__robExt_current_compilation_command_str}
        \fi
      }
    \else
      \bool_if:nTF { \sys_if_shell_unrestricted_p: || \cs_if_exist_p:N \robExtForceCompilation }
      {
        \message{[robExt] We ~ will ~ start ~ the ~ compilation using: ~ \l__robExt_current_compilation_command_str.}
        \sys_shell_now:x {\l__robExt_current_compilation_command_str} % The ~ are used in ExplSyntaxOn to add space
        %% We notify that we are running the compilation command, this way it is possible to check later
        %% if the file is not present because of some compilation error or because we are in manual mode
        %% or fallback
        \cs_set:Nn \l__robExt_I_just_ran_a_compilation_command: {}
      }{
        \ifdefined\robExtFallbackManualMode
          % Avoid writing the same command multiple times (might occur in environments that execute the
          % same command multiple times like align etc)
          \cs_if_exist:cTF {l__robExt_missing_figure_\l__robExt_current_compilation_command_str :}{}{
            \message{[robExt] Fallback to manual mode: please, manually compile the images using \l__robExt_current_compilation_command_str or run 'bash \jobname-\robExtAddPrefixName{compile-missing-figures.sh}'.}
            \iow_now:Nx \g__robExt_write_manually_compile_all_missing_figures_iow {\l__robExt_current_compilation_command_str}% a new line is automatically added
            \cs_gset:cn {l__robExt_missing_figure_\l__robExt_current_compilation_command_str :} {}
            \ifdefined\robExt@compile@parallel@after
              \gdef\robExt@compile@parallel@must@compile{}
              \seq_gput_right:Nx \g__robExt_files_compiled_in_parallel {\robExtAddCachePathAndName{\robExtFinalHash}}
            \fi
          }
        \else
          \msg_error:nn{robExt}{need shell escape}
        \fi
      }
    \fi
  }
}

\robExtSetPlaceholder*{__ROBEXT_INCLUDE_COMMAND__}{%
  \ifdefined\robExtDepth%
    \raisebox{-\robExtDepth}{%
      \includegraphics[__ROBEXT_INCLUDEGRAPHICS_OPTIONS__]{%
        __ROBEXT_INCLUDEGRAPHICS_FILE__%
      }}%
  \else%
    \includegraphics[__ROBEXT_INCLUDEGRAPHICS_OPTIONS__]{%
      %\robExtAddCachePathAndName{\robExtFinalHash.pdf}%
      __ROBEXT_INCLUDEGRAPHICS_FILE__%
    }%
  \fi
}

% These special placeholders are not loaded before for efficiency reasons.
\NewDocumentCommand{\robExtLoadSpecialPlaceholders}{}{
  \robExtPlaceholderFromString {__ROBEXT_OUTPUT_PREFIX__}{\l_tmp_output_prefix_str}
  \robExtPlaceholderFromStringExpanded {__ROBEXT_SOURCE_FILE__}{\l_tmp_output_prefix_str .tex}
  \robExtPlaceholderFromStringExpanded {__ROBEXT_OUTPUT_PDF__}{\l_tmp_output_prefix_str .pdf}
  \robExtPlaceholderFromStringExpanded {__ROBEXT_WAY_BACK__}{\robExtCacheFolderWayBack}
  \robExtPlaceholderFromStringExpanded {__ROBEXT_CACHE_FOLDER__}{\robExtCacheFolder}
}

\def\robExtIncludeGraphicsArgs{}
%%% This command is not meant to be called by the end user. It will be called after the compilation to include
%%% the compiled file back into the original file.
\NewDocumentCommand{\robExtIncludeFile}{m}{%
  % We can disable this loading for efficiency reasons
  \ifdefined\robExtDoNotLoadSpecialPlaceholders\else\robExtLoadSpecialPlaceholders\fi%
  \ifdefined\robExtIncludeCommandAdvanced%
    \robExtIncludeCommandAdvanced%
  \else%
    {%
      \file_if_exist:xTF{\robExtAddCachePathAndName{\robExtFinalHash.pdf}}{%
        \file_if_exist:xTF{\robExtAddCachePathAndName{\robExtFinalHash-out.tex}}{%
          \kern0pt%Without the kern, the next unskip would eat spaces before... and we don't want that. See also
          % https://tex.stackexchange.com/questions/104034/when-is-it-good-practice-to-use-unskip
          \input{\robExtAddCachePathAndName{\robExtFinalHash-out.tex}}\unskip% Otherwise if the file contains space it will be added here.
        }{}%
        %\robExtConfigure{run ~ before ~ include ~ command}%
        \ifdefined\robExtIncludeCommand%
          \robExtIncludeCommand%
        \else%
\robExtEvalPlaceholderStartFromList{__ROBEXT_INCLUDE_COMMAND__,__ROBEXT_INCLUDEGRAPHICS_OPTIONS__,__ROBEXT_INCLUDEGRAPHICS_FILE__,__ROBEXT_LATEX_TRIM_LENGTH__}{__ROBEXT_INCLUDE_COMMAND__}%
        \fi%
        %\robExtConfigure{run ~ after ~ include ~ command}%
      }{
        % Check if we tried to run a compilation command at the previous step:
        \cs_if_exist:NTF \l__robExt_I_just_ran_a_compilation_command: {
          % We ran a compilation command but it failed: we print an error message
          \msg_error:nnxxx{robExt}{missing compiled pdf parallel}{\robExtAddCachePathAndName{\robExtFinalHash}}{\msg_line_number:}{\l__robExt_current_compilation_command_str}
        } { % We ran no compilation command: we print instead a placeholder depending on the context:
          \cs_if_exist:NTF \robExt@compile@parallel@must@compile {
            % we are supposed to compile the current picture in parallel, let'us check if it will be possible:
            \bool_if:nTF { \sys_if_shell_unrestricted_p: || \cs_if_exist_p:N \robExtForceCompilation} {
              \robExtImagePlaceholderIfParallelCompilation
            }{
              \robExtImagePlaceholderIfManualMode
            }
          }{
            \cs_if_exist:NTF \robExtManualMode {
              \robExtImagePlaceholderIfManualMode
              \message{[robExt] ~ You ~ are ~ in ~ manual ~ mode: ~ please ~ compile ~ yourself ~ \robExtAddCachePathAndName{\robExtFinalHash.tex} ~ or ~ use ~ the ~ bash ~ \jobname-\robExtAddPrefixName{compile-missing-figures.sh}}
            }{
              \robExtImagePlaceholderIfFallbackMode
              \message{[robExt] ~ You ~ are ~ falling ~ back ~ to ~ manual ~ mode: ~ please ~ compile ~ yourself ~ \robExtAddCachePathAndName{\robExtFinalHash.tex} ~ or ~ use ~ the ~ bash ~ \jobname-\robExtAddPrefixName{compile-missing-figures.sh}}
            }
          }
        }
      }%
    }%
  \fi%
}

%%%%%%%%%%%%%%%%%%%%%%%%%%%%%%%%%%
%%% Disabling externalization
%%%%%%%%%%%%%%%%%%%%%%%%%%%%%%%%%%

% It seems that when we disable externalization on \tikz, \tikz internally call \tikzpicture in a
% weird way (certainly some tikz magic), and as a result it does not manage to grab the end of the CacheMe
% environment. For this reason, by default, |disable externalization| will disable **all** commands

\seq_clear_new:N \l_commands_to_reset_seq % List of commands to reset by default when disable externalization is set

\NewDocumentCommand{\robExtAddToCommandResetList}{m}{
  \seq_put_right:Nn \l_commands_to_reset_seq {#1}
}

\NewDocumentCommand{\robExtSetCommandResetList}{m}{
  \seq_set_from_clist:Nn \l_commands_to_reset_seq {#1}
}

\seq_clear_new:N \l_environments_to_reset_seq % List of environments to reset by default when disable externalization is set

\NewDocumentCommand{\robExtAddToEnvironmentResetList}{m}{
  \seq_put_right:Nn \l_environments_to_reset_seq {#1}
}

\NewDocumentCommand{\robExtSetEnvironmentResetList}{m}{
  \seq_set_from_clist:Nn \l_environments_to_reset_seq {#1}
}

\NewDocumentCommand{\robExtDisableTikzpictureOverwrite}{}{%
  \ifdefined\robExtTikzPictureOrig%
    \let\tikzpicture\robExtTikzPictureOrig%
    \let\endtikzpicture\endrobExtTikzPictureOrig%
  \fi%
  \ifdefined\robExtEnvironmentOrigName%
    \expanded{\noexpand\DeclareEnvironmentCopy{\robExtEnvironmentOrigName}{robExtEnvironmentOrig\robExtEnvironmentOrigName}}%
  \fi%
  \ifdefined\robExtCommandOrigName%
    \expandafter\DeclareCommandCopy\csname \robExtCommandOrigName\expandafter\endcsname\csname robExtCommandOrig\robExtCommandOrigName\endcsname%
  \fi%
  \ifdefined\robExtDoNotResetAllCommands\else%
    \seq_map_inline:Nn \l_commands_to_reset_seq {
      \cs_if_exist:cTF { robExtCommandOrig##1 } {
        \expandafter\DeclareCommandCopy\csname ##1\expandafter\endcsname\csname robExtCommandOrig##1\endcsname%
      } {}
    }
    \seq_map_inline:Nn \l_environments_to_reset_seq {
      \cs_if_exist:cTF { robExtEnvironmentOrig##1 } {
        \expanded{\noexpand\DeclareEnvironmentCopy{##1}{robExtEnvironmentOrig##1}}
      } {}
    }
  \fi%
}

%%%%%%%%%%%%%%%%%%%%%%%%%%%%%%%%%%
%%% Automatically forward elements
%%%%%%%%%%%%%%%%%%%%%%%%%%%%%%%%%%

% \robExtSetCodeToRunIfMacroPresent{\hello}{some code} will execute "some code" if \hello is present in
% the main content to build and if "auto forward" is enabled.
\NewDocumentCommand{\robExtSetCodeToRunIfMacroPresent}{mm}{
  % we need to remove the leading space
  % \str_set:Nx \l__robExt_tmp_str {\cs_to_str:N #1}
  \expandafter\def \csname l__robExt_execute_if_macro_present\string#1\endcsname {#2}%
}

% \robExtRunCodeToRunForMacroPresent{\hello} will run the code that was configured when calling
% \robExtSetCodeToRunIfMacroPresent{\hello}{some code}
\NewDocumentCommand{\robExtRunCodeToRunForMacroPresent}{m}{
  % \str_set:Nx \l__robExt_tmp_str {\cs_to_str:N ##1}
  \cs_if_exist_use:c {l__robExt_execute_if_macro_present\string#1}
}

% \autoForwardMacro{\hello}[\firstMacroDeps,\secondMacroDeps] will be used to tell to "auto forward" to forward \hello, \firstMacroDeps, and \secondMacroDeps if \hello is present in the main content to build.
\NewDocumentCommand{\robExtConfigIfMacroPresent}{mm}{%
  \robExtSetCodeToRunIfMacroPresent{#1}{%
    \pgfkeysalso{#2}%
  }%
}
\let\configIfMacroPresent\robExtConfigIfMacroPresent

\NewDocumentCommand{\robExtAutoForwardMacro}{mO{}}{%
  \robExtConfigIfMacroPresent{#1}{
    forward=#1,#2
  }%a
}
\let\autoForwardMacro\robExtAutoForwardMacro

\NewDocumentCommand{\robExtLoadAutoForwardMacroConfig}{+m}{
  %\str_set:Nx \l__robExt_tmp_str {\cs_to_str:N #1}
  \cs_if_exist:cTF {l__robExt_execute_if_macro_present\string#1}{
    % if the macro is present twice, we want to run \forward only once
    \cs_if_exist:cTF {l__robExt_execute_if_macro_present_already_forwarded\string#1 :}{}{
      \use:c {l__robExt_execute_if_macro_present\string#1}
      % define it so that we do not import twice next time
      \cs_set:cx {l__robExt_execute_if_macro_present_already_forwarded\string#1 :} {}
    }
  }{}
}



%% Old version: too inneficient
% \regex_const:Nn \l__robExt_macro_regex { \\[A-Za-z]+ }
% \NewDocumentCommand{\robExtAutoForward}{}{
%   \seq_clear_new:N \l__robExt_matches_seq
%   \regex_extract_all:NVN \l__robExt_macro_regex \l__robExt_placeholder___ROBEXT_MAIN_CONTENT_ORIG___str \l__robExt_matches_seq%
%   \seq_map_inline:Nn \l__robExt_matches_seq {
%     \robExtLoadAutoForwardMacroConfig{##1}
%   }
% }

%% "auto forward"
\NewDocumentCommand{\robExtAutoForward}{}{
  % some code will automatically add auto forward, but we don't want to run it twice.
  \robExtDebugInfo{Running auto forward.}
  \ifdefined\robExt@already@ran@autoforward\robExtDebugInfo{Auto forward already ran}\else
    \ifdefined\robExtUserInputCacheMe\else
      \msg_warning:nn{robExt}{auto forward not in cachemecode}
      \tl_set_rescan:NnV \robExtUserInputCacheMe {} \l__robExt_placeholder___ROBEXT_MAIN_CONTENT_ORIG___str
    \fi
    \robExt@normalBraces\robExtUserInputCacheMe
    \expandafter\robExt@getfromstringA\robExtUserInputCacheMe\robExt@getfromstringA%\robExt@getfromstringA is added to know when to stop
    \def\robExt@already@ran@autoforward{}%
  \fi
}

% https://tex.stackexchange.com/questions/700834/efficiently-program-search-of-macro-in-string/700844
\long\def\robExt@getfromstringA#1{\ifx#1\robExt@getfromstringA \else
  %\ifcsname L:\string#1\endcsname \addto\listmacros{#1}\fi
  \robExtLoadAutoForwardMacroConfig{#1}%
  \expandafter\robExt@getfromstringA\fi
}

\def\robExt@normalBraces#1{{\catcode`{=12 \catcode`}=12 \everyeof={\endfile}%
   \expandafter\robExt@normalBracesA\expandafter#1\scantokens\expandafter{#1}}}
\long\def\robExt@normalBracesA#1#2\endfile{\gdef#1{#2}}

%%% To automatically define and forward
\NewDocumentCommand{\newcommandAutoForward}{moomO{}}{
  \IfNoValueTF{#2}{
    \IfNoValueTF{#3}{
      \newcommand{#1}{#4}
    }{
      \newcommand{#1}[#3]{#4}
    }
  }{
    \IfNoValueTF{#3}{
      \newcommand{#1}[#2]{#4}
    }{
      \newcommand{#1}[#2][#3]{#4}
    }
  }
  \robExtAutoForwardMacro{#1}[#5]
}

%%% To automatically define and forward
\NewDocumentCommand{\renewcommandAutoForward}{moomO{}}{
  \IfNoValueTF{#2}{
    \IfNoValueTF{#3}{
      \renewcommand{#1}{#4}
    }{
      \renewcommand{#1}[#3]{#4}
    }
  }{
    \IfNoValueTF{#3}{
      \renewcommand{#1}[#2]{#4}
    }{
      \renewcommand{#1}[#2][#3]{#4}
    }
  }
  \robExtAutoForwardMacro{#1}[#5]
}

%%% To automatically define and forward
\NewDocumentCommand{\providecommandAutoForward}{moomO{}}{
  \IfNoValueTF{#2}{
    \IfNoValueTF{#3}{
      \providecommand{#1}{#4}
    }{
      \providecommand{#1}[#3]{#4}
    }
  }{
    \IfNoValueTF{#3}{
      \providecommand{#1}[#2]{#4}
    }{
      \providecommand{#1}[#2][#3]{#4}
    }
  }
  \robExtAutoForwardMacro{#1}[#5]
}

\NewDocumentCommand{\NewDocumentCommandAutoForward}{mmO{}m}{
  \NewDocumentCommand{#1}{#2}{#4}
  \robExtAutoForwardMacro{#1}[#3]
}

\NewDocumentCommand{\RenewDocumentCommandAutoForward}{mmO{}m}{
  \RenewDocumentCommand{#1}{#2}{#4}
  \robExtAutoForwardMacro{#1}[#3]
}

\NewDocumentCommand{\ProvideDocumentCommandAutoForward}{mmO{}m}{
  \ProvideDocumentCommand{#1}{#2}{#4}
  \robExtAutoForwardMacro{#1}[#3]
}

\NewDocumentCommand{\DeclareDocumentCommandAutoForward}{mmO{}m}{
  \DeclareDocumentCommand{#1}{#2}{#4}
  \robExtAutoForwardMacro{#1}[#3]
}

\NewDocumentCommand{\NewExpandableDocumentCommandAutoForward}{mmO{}m}{
  \NewExpandableDocumentCommand{#1}{#2}{#4}
  \robExtAutoForwardMacro{#1}[#3]
}

\NewDocumentCommand{\RenewExpandableDocumentCommandAutoForward}{mmO{}m}{
  \RenewExpandableDocumentCommand{#1}{#2}{#4}
  \robExtAutoForwardMacro{#1}[#3]
}

\NewDocumentCommand{\ProvideExpandableDocumentCommandAutoForward}{mmO{}m}{
  \ProvideExpandableDocumentCommand{#1}{#2}{#4}
  \robExtAutoForwardMacro{#1}[#3]
}

\NewDocumentCommand{\DeclareExpandableDocumentCommandAutoForward}{mmO{}m}{
  \DeclareExpandableDocumentCommand{#1}{#2}{#4}
  \robExtAutoForwardMacro{#1}[#3]
}

\NewDocumentCommand{\defAutoForward}{mO{}mO{}}{
  \def#1#2{#3}%
  \robExtAutoForwardMacro{#1}[#4]%
}

% \robExtGenericAutoForward[tikz][namespace]{string to match}[additional style to run]{
%    code to run in cached file if string matches, and to always run in current folder if not using
%    the star version
%  }
\ExplSyntaxOff

\NewDocumentCommand{\definecolorAutoForward}{mmmO{}}{%
  \definecolor{#1}{#2}{#3}%
  \robExtConfigure{if matches word={#1}{forward color=#1,#4}}%
}

\NewDocumentCommand{\colorletAutoForward}{mmO{}}{%
  \colorlet{#1}{#2}%
  \robExtConfigure{if matches word={#1}{forward color=#1,#3}}%
}

\NewDocumentCommand{\robExtGenericAutoForward}{sO{latex}O{}mO{}m}{%
  \IfBooleanTF{#1}{}{#6}% We run the code right now
  \robExt@set@hash@robust\zxTmpMacro{#6}%
  \expanded{%
    \noexpand\robExtConfigure{%
      register word with namespace={#3}{#4}{%
        /utils/exec={%
          \noexpand\robExtAddBeforePlaceholder*{__ROBEXT_MAIN_CONTENT__}{%
            \expandonce{\zxTmpMacro}%
          }%
        },%
        #5%
      },%
      add to preset={#2}{%
        auto forward words namespace={#3},%
      }%
    }%
  }%  
}

\let\genericAutoForward\robExtGenericAutoForward

% uses "if matches" instead of "if matches word"
\NewDocumentCommand{\robExtGenericAutoForwardStringMatch}{sO{latex}mO{}m}{%
  \IfBooleanTF{#1}{}{#5}% We run the code right now
  \robExt@set@hash@robust\zxTmpMacro{#5}%
  \robExtConfigure{%
    add to preset={#2}{%
      if matches={#3}{
        run command if externalization={
          \expanded{%
            \noexpand\robExtAddBeforePlaceholder*{__ROBEXT_MAIN_CONTENT__}{%
              \expandonce{\zxTmpMacro}%
              % #1%%
            }%
          }%
        },#4%
      }%
    }%
  }%
}
\let\genericAutoForwardStringMatch\robExtGenericAutoForwardStringMatch
\ExplSyntaxOn

%%%%%%%%%%%%%%%%%%%%%%%%%%%%%%%%%%%%%%%
%%% Automatically forward (regex-based)
%%%%%%%%%%%%%%%%%%%%%%%%%%%%%%%%%%%%%%%

% called by style "if matches".
% \robExtRegexMatches{my regex}{my style} will apply "my style" if "__ROBEXT_MAIN_CONTENT_ORIG__" matches
% "my regex".
\NewDocumentCommand{\robExtIfMatchesRegex}{mm}{
  \regex_match:nVTF {#1} \l__robExt_placeholder___ROBEXT_MAIN_CONTENT_ORIG___str {\pgfkeysalso{#2}} {}
}

\NewDocumentCommand{\robExtIfMatchesString}{mm}{
  \str_if_in:NnTF \l__robExt_placeholder___ROBEXT_MAIN_CONTENT_ORIG___str {#1} {\pgfkeysalso{#2}} {}
}

% \robExtRegisterWord {namespace} {word} {style}
\NewDocumentCommand{\robExtRegisterWord}{mmm}{
  % \robExt_register_match_word:nnn {#1} {#2} {\pgfkeysalso{#3}}
  \robExtRegisterWordCode{#1}{#2}{\pgfkeysalso{#3}}%
}

%%%%%%%%%%%%%%%%%%%%%%%%%%%%%%%%%%%%%%
%%% Automatically forward (word-based)
%%%%%%%%%%%%%%%%%%%%%%%%%%%%%%%%%%%%%%
% For efficiency reasons, we first extract in a single pass all words in the text (a word is basically [A-Za-z]+)
% and run a function on these words

%%% I tried this, but it is really slow (500x slower than if_string_matches, overall 20% increase of
%%% compilation time)
% % \__robExt_auto_forward_words:N \commandToRunOnEachWord \stringToSearchOn
% \cs_set:Nn \__robExt_auto_forward_words:NN {
%   % \l_tmpa_str will contain the current word read so far
%   \str_set:Nn \l_tmpa_str {}%
%   \str_map_inline:Nn #2 {
%     % \token_case_charcode:NnTF ##1 {} {} {}
%     \__robExt_if_letter:nTF {##1} {
%       \str_put_right:Nn \l_tmpa_str {##1}
%     }{
%       \str_if_empty:NTF \l_tmpa_str { } {
%         % if the string is empty, we run the command on the string
%         #1 \l_tmpa_str%
%         \str_set:Nn \l_tmpa_str {}% we reset its value
%       }
%     }
%   }
% }

% %% \__robExt_if_letter:nTF {char} {true} {false} tests if an element is a letter
% %% https://tex.stackexchange.com/a/700864/116348
% \prg_new_conditional:Npnn \__robExt_if_letter:n #1 { TF }
% {
%   \bool_lazy_or:nnTF
%   {
%     \bool_lazy_and_p:nn
%     { \int_compare_p:nNn { `#1 } > { `a - 1 } }
%     { \int_compare_p:nNn { `#1 } < { `z + 1 } }
%   }
%   {
%     \bool_lazy_and_p:nn
%     { \int_compare_p:nNn { `#1 } > { `A - 1 } }
%     { \int_compare_p:nNn { `#1 } < { `Z + 1 } }
%   }
%   \prg_return_true:
%   \prg_return_false:
% }

% % \robExt_register_match_word {namespace that defaults to empty} {word} {code to run if word is present} 
% \cs_set:Nn \robExt_register_match_word:nnn {
%   \cs_set:cn {l__robExt_execute_if_word_present_#1_#2:} {#3}
% }

% % \robExt_try_to_execute_if_match_word:nn {namespace} {word}
% \cs_set:Nn \robExt_try_to_execute_if_match_word:nn {
%   \cs_if_exist:cTF {l__robExt_execute_if_word_present_#1_#2:} {%
%     \cs_if_exist:cTF {l__robExt_execute_if_word_present_#1_#2__already_forwarded:}{\message{Already forwarded}}{
%       \use:c {l__robExt_execute_if_word_present_#1_#2:}%
%       % define it so that we do not import twice next time
%       \cs_set:cx {l__robExt_execute_if_word_present_#1_#2__already_forwarded:} {}
%     }
%   } { }
% }
% \cs_generate_variant:Nn \robExt_try_to_execute_if_match_word:nn { nV }

% \NewDocumentCommand{\robExtAutoForwardWords}{O{}}{
%   \cs_set:Nn \__robExt_tmp_fct:N {
%     \robExt_try_to_execute_if_match_word:nV {#1} ##1
%   }
%   \__robExt_auto_forward_words:NN \__robExt_tmp_fct:N \l__robExt_placeholder___ROBEXT_MAIN_CONTENT_ORIG___str
% }

% \robExtAutoForwardWords[namespace]
\NewDocumentCommand{\robExtAutoForwardWords}{m}{%
  % needed or the original string will be changed
  \let\robExt@tmpString\l__robExt_placeholder___ROBEXT_MAIN_CONTENT_ORIG___str%
  \robExt@scanmacro@find@word{#1}\robExt@tmpString%
}

\ExplSyntaxOff

% Thanks a lot wipet for this optimized version!!
% https://tex.stackexchange.com/questions/701351/more-efficient-string-extraction-of-words/701441#701441
\def\robExtWordSeparators{{ };:."?!@+-/*,=\{\}[]\\'()&|~_^<>}

% Not for user, use \robExtAutoForwardWords
% \robExt@scanmacro@find@word{namespace}\stringToParse
\def\robExt@scanmacro@find@word#1#2{%
  \def\robExt@namespace{#1}%
  % we will change the lccode (basically ascii code) of some characters like +- to delimit what a word is
  % supposed to be. Since we do not want to affect other stuff, we put it in a bgroup.
  \bgroup \expandafter\robExt@set@to@comma\robExtWordSeparators\relax%
  \ifdefined\robExtAdditionalCodeInScanMacroFindWord\robExtAdditionalCodeInScanMacroFindWord\fi%
  % This seems to set the string in lower case, so fill or Fill or get triggered. But it is not what we want here.
  % \lowercase\expandafter{\expandafter\gdef\expandafter#2\expandafter{#2}}%
  \lowercase\expandafter{\expandafter\gdef\expandafter#2\expandafter{#2}}%
  \edef#2{\detokenize\expandafter{#2}}%
  % \message{\string#2: \meaning#2} % prints the modified format of the scanned macro
  \expandafter\egroup%
  \expandafter\robExt@wordscan@aux#2,\relax,%
}
\def\robExt@set@to@comma #1{\ifx\relax#1\else \lccode`#1=`, \expandafter\robExt@set@to@comma\fi}
\def\robExt@wordscan@aux#1,{\ifx\relax#1\empty\else%
   %\message{{#1}}  % prints each scanned "word"
   \ifcsname robExt@action@to@run@on@word:\robExt@namespace:#1\endcsname \csname robExt@action@to@run@on@word:\robExt@namespace:#1\endcsname \fi%
   \expandafter\robExt@wordscan@aux\fi%
}
%\robExtRegisterWordCode{namespace}{word}{code}
\def\robExtRegisterWordCode#1#2#3{\expandafter\gdef\csname robExt@action@to@run@on@word:\string#1:\string#2\endcsname{#3}}


%%%%%%%%%%%%%%%%%%%%%%%%%%%%%%%%%%
%%% Interface
%%%%%%%%%%%%%%%%%%%%%%%%%%%%%%%%%%
% We create interface into pgfkeys in order to allow easier creation of content via style
\pgfkeys{%
  /robExt/.cd,%
  % We create a default style that will be loaded (mostly for the user)
  default style/.style={},%
  %%%%%%%%%%%%%%%%%%%%%%%%%%%%%%%%%
  %%% Code to create new styles %%%
  %%%%%%%%%%%%%%%%%%%%%%%%%%%%%%%%%
  % The advantage of this over .append style is that you do not need to double the number of hashes
  % don't know if there is a better solution.
  % https://tex.stackexchange.com/questions/695432/latex3-latex-doubles-the-number-of-hashes-when-storing-them-in-string/695461
  add to preset/.code 2 args={%
    \robExtStrSetDoubleHash{\robExtTmpStr}{#2}%
    % Sadly, \expanded{\noexpand } does not work, as I get extra {} around the def, creating a group
    % so the simpler seems to use this library ^^
    \robExtPlaceholderFromString{__ROBEXT_TMP__}{\robExtTmpStr}%
    \robExtEvalPlaceholderReplaceFromList{__ROBEXT_TMP__}{%
      \pgfkeys{%
        /robExt/.cd,
        #1/.append style={%
          % Some styles run differently if inside a preset or not,
          % like if matches word. This macro helps with detecting it.
          /utils/exec={\def\robExtCurrentlyDefiningPreset{}},%
          __ROBEXT_TMP__%
        },%
      }%
    }%
    \robExtRemovePlaceholder{__ROBEXT_TMP__}% let us clean our variables
    \let\robExtCurrentlyDefiningPreset\undefined%
  },
  new preset/.code 2 args={%
    \robExtStrSetDoubleHash{\robExtTmpStr}{#2}%
    % Sadly, \expanded{\noexpand } does not work, as I get extra {} around the def, creating a group
    % so the simpler seems to use this library ^^
    \robExtPlaceholderFromString{__ROBEXT_TMP__}{\robExtTmpStr}%
    % \robExtShowPlaceholder*{__ROBEXT_TMP__}
    \robExtEvalPlaceholderReplaceFromList{__ROBEXT_TMP__}{%
      \pgfkeys{%
        /robExt/.cd,%
        #1/.style={%
          % Some styles run differently if inside a preset or not,
          % like if matches word. This macro helps with detecting it.
          /utils/exec={\def\robExtCurrentlyDefiningPreset{}},%
          __ROBEXT_TMP__%
        },%
      }%
    }%
    \robExtRemovePlaceholder{__ROBEXT_TMP__}% let us clean our variables
    \let\robExtCurrentlyDefiningPreset\undefined%
  },
  % new compiled preset={latex compiled}{
  %  code to compile in order to produce the expected strings  
  % }{
  %  code to initialize
  % }
  compile latex template/.style={
    % to allow later addition to the preamble, we add a dummy value that we replace in a few lines
    % same for main content and content orig that we do not want to replace.
    add to placeholder={__ROBEXT_LATEX_PREAMBLE__}{__ROBEXT_LATEX_PREAMBLE_DUMMY__},
    set placeholder={__ROBEXT_MAIN_CONTENT__}{__ROBEXT_MAIN_CONTENT_DUMMY__},
    set placeholder={__ROBEXT_MAIN_CONTENT_ORIG__}{__ROBEXT_MAIN_CONTENT_ORIG_DUMMY__},
    /utils/exec={%
      \robExtGetNearlyFinalValueTemplateAndCompilationCommand%
      \robExtSetBackCompilationCommandAndTemplate%
    },
    set placeholder={__ROBEXT_LATEX_PREAMBLE_DUMMY__}{__ROBEXT_LATEX_PREAMBLE__},
    set placeholder={__ROBEXT_MAIN_CONTENT_DUMMY__}{__ROBEXT_MAIN_CONTENT__},
    set placeholder={__ROBEXT_MAIN_CONTENT_ORIG_DUMMY__}{__ROBEXT_MAIN_CONTENT_ORIG__},
    set placeholder rec replace from list={__ROBEXT_TEMPLATE__,__ROBEXT_LATEX_PREAMBLE_DUMMY__,__ROBEXT_MAIN_CONTENT_DUMMY__,__ROBEXT_MAIN_CONTENT_ORIG_DUMMY__}{__ROBEXT_TEMPLATE__}{__ROBEXT_TEMPLATE__},
    % For the include command
    set placeholder rec replace from list={__ROBEXT_INCLUDE_COMMAND__,__ROBEXT_INCLUDEGRAPHICS_OPTIONS__,__ROBEXT_INCLUDEGRAPHICS_FILE__,__ROBEXT_LATEX_TRIM_LENGTH__}{__ROBEXT_INCLUDE_COMMAND__}{__ROBEXT_INCLUDE_COMMAND__},
  },
  keep placeholder after group/.code={\robExtKeepPlaceholderAfterGroup{#1}},
  new compiled preset/.code n args={3}{
    \robExtStrSetDoubleHash{\robExtCodeToCompileStr}{#2}%
    \robExtStrSetDoubleHash{\robExtCodeToRunStr}{#3}%
    % Sadly, \expanded{\noexpand } does not work, as I get extra {} around the def, creating a group
    % so the simpler seems to use this library ^^
    \robExtPlaceholderFromString{__ROBEXT_TMP_CODE_TO_COMPILE__}{\robExtCodeToCompileStr}%
    \robExtPlaceholderFromString{__ROBEXT_TMP_CODE_TO_RUN__}{\robExtCodeToRunStr}%
    \robExtPlaceholderFromString{__ROBEXT_NAME_TEMPLATE_PREFIX__}{\robExtCodeToRunStr}%
    % \robExtShowPlaceholder*{__ROBEXT_TMP__}
    \robExtEvalPlaceholderReplaceFromList{__ROBEXT_TMP_CODE_TO_COMPILE__,__ROBEXT_TMP_CODE_TO_RUN__}{%
      \robExtConfigure{%
        #1-recompile/.code={%
          \begingroup%
            \robExtConfigure{
              __ROBEXT_TMP_CODE_TO_COMPILE__,
            }%
            \robExtCopyPlaceholder*{__ROBEXT_COMPILED_#1_TEMPLATE__}{__ROBEXT_TEMPLATE__}%
            \robExtKeepPlaceholderAfterGroup{__ROBEXT_COMPILED_#1_TEMPLATE__}%
            \robExtCopyPlaceholder*{__ROBEXT_COMPILED_#1_COMPILATION_COMMAND__}{__ROBEXT_COMPILATION_COMMAND__}%
            \robExtKeepPlaceholderAfterGroup{__ROBEXT_COMPILED_#1_COMPILATION_COMMAND__}%
            \robExtCopyPlaceholder*{__ROBEXT_COMPILED_#1_INCLUDE_COMMAND__}{__ROBEXT_INCLUDE_COMMAND__}%
            \robExtKeepPlaceholderAfterGroup{__ROBEXT_COMPILED_#1_INCLUDE_COMMAND__}%
            \robExtRescanPlaceholderInVariableNoReplacement{robExtCompiledIncludeCommand#1}{__ROBEXT_INCLUDE_COMMAND__}%
            \robExtKeepaftergroup{robExtCompiledIncludeCommand#1}%
            \robExtKeepaftergroup{l__robExt_placeholder___ROBEXT_COMPILED_#1_INCLUDE_COMMAND___str}%
            \robExtKeepaftergroup{l__robExt_placeholder___ROBEXT_COMPILED_#1_TEMPLATE___str}%
            \robExtSaveDependencies{robExtCompiledDependencies#1}%
          \endgroup%
        },
        #1-recompile,
        #1/.style={
          disable placeholders,
          set placeholder={__ROBEXT_LATEX_PREAMBLE__}{},
          add to preamble/.style={
            add to placeholder={__ROBEXT_LATEX_PREAMBLE__}{####1},
          },
          /utils/exec={\robExtRestoreDependencies{robExtCompiledDependencies#1}},
          custom include command={\csname robExtCompiledIncludeCommand#1\endcsname},
          copy placeholder no import={__ROBEXT_TEMPLATE__}{__ROBEXT_COMPILED_#1_TEMPLATE__},
          copy placeholder no import={__ROBEXT_COMPILATION_COMMAND__}{__ROBEXT_COMPILED_#1_COMPILATION_COMMAND__},
          __ROBEXT_TMP_CODE_TO_RUN__
        },%
      }%
    }%
    \robExtRemovePlaceholder{__ROBEXT_TMP_CODE_TO_RUN__}% let us clean our variables
    \robExtRemovePlaceholder{__ROBEXT_TMP_CODE_TO_COMPILE__}% let us clean our variables
  },
  in command/.style={
    set placeholder={__ROBEXT_MAIN_CONTENT__}{__ROBEXT_MAIN_CONTENT_ORIG__},
  },
  %%%%%%%%%%%%%%%%%%%%%%%%%%%%%%%%%%%%%%%%
  %%% Interface to change placeholders %%%
  %%%%%%%%%%%%%%%%%%%%%%%%%%%%%%%%%%%%%%%%
  remove placeholder/.code={\robExtRemovePlaceholder{#1}},
  remove placeholders/.style={
    remove placeholder/.list={#1},
  },
    set main content/.style={
    set placeholder={__ROBEXT_MAIN_CONTENT_ORIG__}{#1}
  },
  copy placeholder/.code 2 args={\robExtCopyPlaceholder{#1}{#2}},
  copy placeholder no import/.code 2 args={\robExtCopyPlaceholder*{#1}{#2}},
  set placeholder/.code 2 args={\robExtSetPlaceholder{#1}{#2}},
  % do not import the placeholder (useful for efficiency reasons)
  set placeholder no import/.code 2 args={\robExtSetPlaceholder*{#1}{#2}},
  set placeholder first/.code 2 args={\robExtSetPlaceholderFirst*{#1}{#2}},
  set placeholder rec/.code 2 args={\robExtSetPlaceholderRec{#1}{#2}},
  set placeholder rec keep after group/.code 2 args={\robExtSetPlaceholderRec{#1}{#2}\robExtKeepPlaceholderAfterGroup{#1}},
  set placeholder rec no import/.code 2 args={\robExtSetPlaceholderRec*{#1}{#2}},
  set placeholder rec replace from list/.code n args={3}{\robExtSetPlaceholderRecReplaceFromList{#1}{#2}{#3}},
  set placeholder rec replace from list no import/.code n args={3}{\robExtSetPlaceholderRecReplaceFromList*{#1}{#2}{#3}},
  set placeholder eval/.code 2 args={\robExtSetPlaceholderRec{#1}{#2}\robExtEvalPlaceholderInplace{#1}},
  set placeholder eval no import/.code 2 args={\robExtSetPlaceholderRec*{#1}{#2}\robExtEvalPlaceholderInplace{#1}},
  set placeholder eval replace from list/.code n args={3}{\robExtSetPlaceholderRecReplaceFromList{#1}{#2}{#3}\robExtEvalPlaceholderInplace{#2}},
  set placeholder eval replace from list no import/.code n args={3}{\robExtSetPlaceholderRecReplaceFromList*{#1}{#2}{#3}\robExtEvalPlaceholderInplace{#2}},
  eval placeholder/.code={\robExtEvalPlaceholder{#1}},
  eval placeholder replace from list/.code 2 args={\robExtEvalPlaceholderReplaceFromList{#1}{#2}},
  set placeholder from content/.code 2 args={\robExtPlaceholderFromContent{#1}{#2}},
  set placeholder from content no import/.code 2 args={\robExtPlaceholderFromContent*{#1}{#2}},
  add to placeholder/.code 2 args={\robExtAddToPlaceholder{#1}{#2}},
  add to placeholder no import/.code 2 args={\robExtAddToPlaceholderNoImport{#1}{#2}},
  add to placeholder no space/.code 2 args={\robExtAddToPlaceholder*{#1}{#2}},
  add to placeholder no space no import/.code 2 args={\robExtAddToPlaceholderNoImport*{#1}{#2}},
  add before placeholder/.code 2 args={\robExtAddBeforePlaceholder{#1}{#2}},
  add before placeholder no space/.code 2 args={\robExtAddBeforePlaceholder*{#1}{#2}},
  add before placeholder no import/.code 2 args={\robExtAddBeforePlaceholderNoImport{#1}{#2}},
  add before placeholder no space no import/.code 2 args={\robExtAddBeforePlaceholderNoImport*{#1}{#2}},
  set placeholder path from filename/.code 2 args={\robExtPlaceholderPathFromFilename{#1}{#2}},
  set placeholder path from filename no import/.code 2 args={\robExtPlaceholderPathFromFilename*{#1}{#2}},
  set placeholder from file content/.code 2 args={\robExtPlaceholderFromFileContent{#1}{#2}},
  set placeholder from file content no import/.code 2 args={\robExtPlaceholderFromFileContent*{#1}{#2}},
  set placeholder path from content/.code n args={3}{\robExtPlaceholderPathFromContent{#1}[#3]{#2}},
  set placeholder path from content no import/.code n args={3}{\robExtPlaceholderPathFromContent*{#1}[#3]{#2}},
  eval placeholder in place/.code={\robExtEvalPlaceholderInplace{#1}},
  placeholder halve number hashes in place/.code={\robExtPlaceholderHalveNumberHashesInplace{#1}},
  placeholder double number hashes in place/.code={\robExtPlaceholderDoubleNumberHashesInplace{#1}},
  placeholder replace in place/.code n args={3}{\robExtPlaceholderReplaceInplace{#1}{#2}{#3}},
  placeholder replace in place eval/.code n args={3}{\robExtPlaceholderReplaceInplaceEval{#1}{#2}{#3}},
  % Interface to set template
  set template/.style={
    set placeholder first={__ROBEXT_TEMPLATE__}{#1},
  },
  %%%%%%%%%%%%%%%%%%%%%%%%%%%%%%%%%%
  %%% Interface for optimization %%%
  %%%%%%%%%%%%%%%%%%%%%%%%%%%%%%%%%%
  % It is important to create an efficient code (otherwise the library can becomes pointless)
  % and it can help a lot if we give some advices to the system that evaluates placeholders, notably on
  % the replacement order it should follow:
  %%%% "only placeholders" says "use only these placeholders in this template/compilation command/... in that order
  %%%% while "first placeholders" says "start with these placeholders, if it is not enough, continue as usual
  only placeholders in compilation command/.code={\robExtOnlyPlaceholdersInCompilationCommand{#1}},
  first placeholders in compilation command/.code={\robExtFirstPlaceholdersInCompilationCommand{#1}},
  only placeholders in template/.code={\robExtOnlyPlaceholdersInTemplate{#1}},
  first placeholders in template/.code={\robExtFirstPlaceholdersInTemplate{#1}},
  only placeholders in include command/.code={\robExtOnlyPlaceholdersInIncludeCommand{#1}},
  first placeholders in include command/.code={\robExtFirstPlaceholdersInIncludeCommand{#1}},
  %%%%%%%%%%%%%
  %%% Debug %%%
  %%%%%%%%%%%%%
  more logs/.code={\def\robExtDebugMessage##1{\message{^^J[robExt] ##1}}\def\robExtDebugInfo##1{^^J[robExt] ##1}},
  less logs/.code={\def\robExtDebugMessage##1{}\def\robExtDebugInfo##1{}},
  show placeholder/.code={\robExtShowPlaceholder{#1}},
  show placeholders/.code={\robExtShowPlaceholders},
  show placeholders contents/.code={\robExtShowPlaceholdersContents},
  print imported placeholders except default/.code={\robExtPrintAllPlaceholdersExceptDefaults},
  print imported placeholders/.code={\robExtPrintAllPlaceholders},
  %%%%%%%%%%%%%%%%%%%%%%%%%%%%%
  %%% Optimizations %%%
  %%%%%%%%%%%%%%%%%%%%%%%%%%%%%
  do not load special placeholders/.code={\let\robExtDoNotLoadSpecialPlaceholders\undefined},
  load special placeholders/.code={\def\robExtDoNotLoadSpecialPlaceholders{}},
  do not load special placeholders,
  %% Replace very little placeholders, basically only __ROBEXT_MAIN_CONTENT__, __ROBEXT_MAIN_CONTENT_ORIG__,
  %% and the default  (prefix, source, output, wayback, cache folder)
  disable placeholders/.code={\def\robExtDisablePlaceholders{}},
  enable placeholders/.code={\let\robExtDisablePlaceholders\undefined},
  %% If all placeholders have __, this option will speed up the compilation time
  all placeholders have underscores/.code={\def\robExtPlaceholderOnlyWithUnderscores{}},
  not all placeholders have underscores/.code={\let\robExtPlaceholderOnlyWithUnderscores\undefined},
  %% not sure if I want users to disable import mechanism as they might use dirty things
  enable import mechanism/.code={\def\robExtEnableImportMechanism{}},
  disable import mechanism/.code={\let\robExtEnableImportMechanism\undefined},
  disable optimizations/.style={
    not all placeholders have underscores,
    disable import mechanism,
  },
  enable optimizations/.style={
    all placeholders have underscores,
    enable import mechanism,
  },
  enable optimizations,
  %%%%%%%%%%%%%%
  %%% Groups %%%
  %%%%%%%%%%%%%%
  clear imported placeholders/.code={\robExtClearImportedPlaceholders},
  remove imported placeholder/.code={\robExtRemoveImportedPlaceholder{#1}},
  remove imported placeholders/.style={
    remove imported placeholder/.list={#1},
  },
  print group placeholders/.code={\robExtPrintGroupPlaceholders{#1}},
  new group placeholders/.code={\robExtRegisterGroupPlaceholders{#1}},
  add placeholder to group/.code 2 args={\robExtAddPlaceholdersToGroup{#1}{#2}},
  add placeholders to group/.code 2 args={\robExtAddPlaceholdersToGroup{#1}{#2}},
  remove placeholder from group/.code 2 args={\robExtRemovePlaceholdersFromGroup{#1}{#2}},
  remove placeholders from group/.code 2 args={\robExtRemovePlaceholdersFromGroup{#1}{#2}},
  copy group placeholders/.code 2 args={\robExtCopyGroupPlaceholders{#1}{#2}},
  append group placeholders/.code 2 args={\robExtAppendGroupPlaceholders{#1}{#2}},
  append before group placeholders/.code 2 args={\robExtAppendGroupPlaceholders{#1}{#2}},
  import placeholders from group/.code={\robExtImportPlaceholdersFromGroup{#1}},
  import all placeholders/.code={\robExtImportAllPlaceholders},
  % ok this is a different kind of group, here latex group { }
  import placeholder/.code={\robExtImportPlaceholder{#1}},
  import all placeholders/.code={\robExtImportAllPlaceholders},
  import placeholders/.style={
    import placeholder/.list={#1},
  },
  import placeholder first/.code={\robExtImportPlaceholderFirst{#1}},
  import placeholders first/.code={\robExtAddPlaceholdersToListFirst{#1}},
  import placeholders from group/.code={\robExtImportPlaceholdersFromGroup{#1}},
  print all registered groups/.code={\robExtPrintAllRegisteredGroups},
  print all registered groups and placeholders/.code={\robExtPrintAllRegisteredGroupsAndPlaceholders},
  show all registered groups/.code={\robExtShowAllRegisteredGroupsAndPlaceholders},
  show all registered groups and placeholders/.code={\robExtShowAllRegisteredGroupsAndPlaceholders},
  %%%%%%%%%%%%%%%%%%%%%%%%%%%%%% 
  %%% Configure dependencies %%%
  %%%%%%%%%%%%%%%%%%%%%%%%%%%%%% 
  %%% Auxiliary command:
  dependenciesList/.code={\robExtAddDependency{#1}},
  % Usage like: dependencies={input_externalize.tex,input_b.tex}
  % They should be relative to the main file when using the subfolder option.
  dependencies/.style={
    /utils/exec={\robExtResetDependencies{}},
    dependenciesList/.list={#1}
  },
  add dependencies/.style={
    dependenciesList/.list={#1}
  },
  reset dependencies/.code={\robExtResetDependencies{}},
  %%%%%%%%%%%%%%%%%%%%%%%%%%% 
  %%% Compilation command %%%
  %%%%%%%%%%%%%%%%%%%%%%%%%%%
  % People might want to force the compilation even if shell_unrestricted is false, notably if they
  % allow commands like mkdir/cd/pdflatex to run in restricted mode.
  force compilation/.code={\def\robExtForceCompilation{}},
  do not force compilation/.code={\let\robExtForceCompilation\undefined},
  % Recompile the file even if it has already been compiled (we do NOT clean the file as we do not feel
  % safe to remove files from this code, but this can also be done by the user hooking into
  recompile/.code={\def\robExtForceRecompilation{}},
  do not recompile/.code={\let\robExtForceRecompilation\undefined},
  set compilation command/.code={\robExtSetCompilationCommand{#1}},
  add argument to compilation command/.code={\robExtAddArgumentToCompilationCommand{#1}},
  add arguments to compilation command/.style={
    add argument to compilation command/.list={#1}
  },
  % This adds arguments like add key value to compilation command={mykey=myvalue} will add to the
  % compilation command two arguments: "mykey" "myvalue"
  % This is useful for scripts that are called like myscript key1 arg1 key2 arg2 key3 arg3, which is a
  % simple way to pass multiple arguments to a script like a python script
  add key value argument to compilation command/.code args={#1=#2}{\robExtAddArgumentToCompilationCommand{#1}\robExtAddArgumentToCompilationCommand{#2}},
  add key and file argument to compilation command aux/.style args={#1=#2}{
    add key value argument to compilation command={{#1}={\ifdefined\robExtCacheFolderWayBack\robExtCacheFolderWayBack\fi#2}},
  },
  add key and file argument to compilation command/.style={
    add key and file argument to compilation command aux/.list={#1},
    add dependencies={#1},
  },
  %%%%%%%%%%%%%%%%%%%%%%%%% 
  %%% Inclusion command %%%
  %%%%%%%%%%%%%%%%%%%%%%%%% 
  %%% Configure the command to include the compiled file back into the main file
  % By default, include command does a bit of logic before running the actual command, notably to
  % input the -out.tex file in order to pass information from the compiled file to the current file.
  % If you want to do everything by yourself, use:
  custom include command advanced/.code={\def\robExtIncludeCommandAdvanced{#1}},
  % The default include command includes the pdf, making sure it is raised depending on its depth,
  % but you can override it:
  custom include command/.code={\def\robExtIncludeCommand{#1}},
  %% Use this when we do not want to include anything (e.g. the video will be processed later in the chain):
  do not include pdf/.style={
    custom include command={}%
  },
  %% If you do or do not want to ask latex to run the compilation commands itself (for instance for security
  %% reasons, you can use these commands and run the command manually later):
  enable manual mode/.code={\def\robExtManualMode{}},
  disable manual mode/.code={\let\robExtManualMode\undefined},
  enable fallback to manual mode/.code={\def\robExtFallbackManualMode{}},
  disable fallback to manual mode/.code={\let\robExtFallbackManualMode\undefined},
  %% Arguments to include graphics
  include graphics args/.code={\def\robExtIncludeGraphicsArgs{#1}},
  %% The role of this command is to set \l_robExt_result_str, that will contain the final string.
  %%%%%%%%%%%%%%%%%%%%%%%%%%%%%%%%%% 
  %%% Configuration of the cache %%%
  %%%%%%%%%%%%%%%%%%%%%%%%%%%%%%%%%% 
  %% Configure the prefix (default to "robExt-")
  set filename prefix/.code={\def\robExtPrefixFilename{#1}},
  % first argument is subfolder, second is how to get from subfolder to the folder containing the source:
  % set subfolder and way back={robustExternal/}{../}
  % synonyme, "cache folder" is prefered over ""
  set subfolder and way back/.code 2 args={\def\robExtCacheFolder{#1}\def\robExtCacheFolderWayBack{#2}},
  set cache folder and way back/.code 2 args={\def\robExtCacheFolder{#1}\def\robExtCacheFolderWayBack{#2}},
  no cache folder/.code={\let\robExtCacheFolder\undefined\def\robExtCacheFolderWayBack{}},
  % By default we put everything in robustExternalize
  % Change this before starting to cache any library, and if you change it mid-document, be aware
  % that you will not be able to refer to elements in the old folder.
  set subfolder and way back={robustExternalize/}{../},
  %%%%%%%%%%%%%%%%%%%%%%%%%%%%%%% 
  %%% Disable externalization %%%
  %%%%%%%%%%%%%%%%%%%%%%%%%%%%%%% 
  %% Note: this does not work reliably for now
  %% TODO: fix this!
  disable externalization/.code={\def\robExtDisableExternalization{}},
  de/.style={disable externalization},
  disable externalization now/.code={\robExtDisableTikzpictureOverwrite\def\robExtDisableExternalization{}},
  enable externalization/.code={\let\robExtDisableExternalization\undefined},
  % Useful to wrap, for instance, text
  command if no externalization/.code={},
  command if no externalization/.code={\robExtDisableTikzpictureOverwrite\evalPlaceholder{__ROBEXT_MAIN_CONTENT__}},
  print verbatim if no externalization/.style={
    command if no externalization/.code={%
      \robExtPrintPlaceholder{__ROBEXT_MAIN_CONTENT__}%
    },
  },
  %%%%%%%%%%%%%%%%%%%%%%%%%%%%%%%%%%%%%%%%%%%%%%%%%%%%
  %%% Forward macros if externalization is enabled %%%
  %%%%%%%%%%%%%%%%%%%%%%%%%%%%%%%%%%%%%%%%%%%%%%%%%%%%
  command if externalization/.code={\def\robExt@we@are@already@in@command@if@externalization{}},
  % Doing "command if externalization/.append code={}" is not always good since we don't always know if the style
  % will always be run before running "command if externalization". In particular this arrives when using
  % something like "if matches word={mypink}{forward color=mypink}"
  run command if externalization/.code={
    \ifdefined\robExt@we@are@already@in@command@if@externalization%
      % we are already running command if externalization
      #1%
    \else%
      \pgfkeysalso{command if externalization/.append code={#1}}%
    \fi%
  },
  fw/.style={forward=#1},
  run code before main content if externalization enabled/.code={},
  % run code before main content if externalization enabled/.code={
  %   \message{aaa #1}
  %   \def\zx@tmp{#1}%
  %   \message{xxx}
  %   \show\zx@tmp%
  %   \scantokens{%
  %     \pgfkeysalso{
  %       command if externalization/.append code={%
  %         \expanded{%
  %           \noexpand\robExtAddBeforePlaceholder*{__ROBEXT_MAIN_CONTENT__}{%
  %             % \expandonce{\zx@tmp}%
  %             #1%%
  %           }%
  %         }%
  %       }%
  %     }%
  %   },
  % },
  forward/.style={%
    run command if externalization={%
      \robExtGetCommandDefinitionInMacro{#1}%
      \expanded{%
        \noexpand\robExtAddBeforePlaceholder*{__ROBEXT_MAIN_CONTENT__}{%
          \expandonce{\robExtDefinitionCommand}%
        }%
      }%
    },
  },
  forward at letter/.style={%
    run command if externalization={%
      \robExtGetCommandDefinitionInMacro{#1}%
      \expanded{%
        \noexpand\robExtAddBeforePlaceholder*{__ROBEXT_MAIN_CONTENT__}{%
          \noexpand\makeatletter%
          \expandonce{\robExtDefinitionCommand}%
          \noexpand\makeatother%
        }%
      }%
    },
  },
  forward eval/.style={%
    run command if externalization={%
      \expanded{%
        \noexpand\robExtAddBeforePlaceholder*{__ROBEXT_MAIN_CONTENT__}{%
          \noexpand\def\noexpand#1{#1}%
        }%
      }%
    },%
  },
  forward counter/.style={%
    run command if externalization={%
      \expanded{%
        \noexpand\robExtAddBeforePlaceholder*{__ROBEXT_MAIN_CONTENT__}{%
          \noexpand\makeatletter%
          % Make sure the counter exists:
          \noexpand\ifcsname c@#1\noexpand\endcsname\noexpand\else\noexpand\newcounter{#1}\noexpand\fi%
          \noexpand\setcounter{#1}{\the\value{#1}}%
          \noexpand\makeatother
        }%
      }%
    },%
  },%
  forward color/.style={
    run command if externalization={%
      \extractcolorspecs{#1}{\zx@tmp@model}{\zx@tmp@cmd}%
      \expanded{%
        \noexpand\robExtAddBeforePlaceholder*{__ROBEXT_MAIN_CONTENT__}{%
          \noexpand\definecolor{#1}{\zx@tmp@model}{\zx@tmp@cmd}%
        }%
      }%
    },%
  },%
  %% This will try to automatically forward some macros. For this, you must first define
  %% 
  auto forward only macros/.code={\robExtAutoForward},
  auto forward/.code={\robExtAutoForward\pgfkeysalso{auto forward words}},
  auto forward color/.style 2 args={%
    add to preset={#1}{%
      if matches word={#2}{forward color=#2},
    },
  },
  load auto forward macro config/.code={\robExtLoadAutoForwardMacroConfig{#1}},
  %% Auto forward words
  auto forward words namespace/.code={%
    \ifdefined\robExt@autoforward@enabled\else%
      \pgfkeysalso{command if externalization/.append code={%
          \ifdefined\robExt@autoforward@enabled%
            \expanded{\noexpand\robExtAutoForwardWords{\robExt@autoforward@enabled}}%
          \fi%
        }%
      }%
    \fi%
    \def\robExt@autoforward@enabled{#1}%
  },%%% todo: finish
  auto forward words/.style={auto forward words namespace={}},
  %% This will
  if matches regex/.code 2 args={\robExtIfMatchesRegex{#1}{#2}},
  if matches/.code 2 args={\robExtIfMatchesString{#1}{#2}},
  if matches word/.code 2 args={%
    \robExtRegisterWord{}{#1}{#2}%
    % If ran inside a preset, we want to enable it, otherwise we enable it on the latex preset
    \ifdefined\robExtCurrentlyDefiningPreset%
      \pgfkeysalso{auto forward words}%
    \else%
      \pgfkeysalso{/robExt/latex/.append style={auto forward words}}%
    \fi%
  },
  % more efficient since we can register the words before the preset, but make sure to call `auto forward words namespace`.
  % You can use an empty namespace.
  register word with namespace/.code n args={3}{%
    \robExtRegisterWord{#1}{#2}{#3}%
  },
  register word/.style 2 args={%
    register word with namespace={}{#1}{#2},%
  },
  %%%%%%%%%%%%%%%%%%%%%%%%%%%%%%%%%%%%%%% 
  %%% Run code before/after inclusion %%%
  %%%%%%%%%%%%%%%%%%%%%%%%%%%%%%%%%%%%%%% 
  %%% todo: make sure that commands can be added instead of replaced
  execute before each externalization/.code={\def\robExtExecuteBefore{#1}},
  execute after each externalization/.code={\def\robExtExecuteAfter{#1}},
  %%%%%%%%%%%%%%%%%%%%%%%%%%%%%%%%%%%%%%%%%%%%%%%%%%%%%%% 
  %%% Get the name of the produced file for later use %%%
  %%%%%%%%%%%%%%%%%%%%%%%%%%%%%%%%%%%%%%%%%%%%%%%%%%%%%%% 
  %%% Here, we provide a way to put the prefixed name into a new global macro
  %%% Use like 'name output=VideoA'. This creates a few macros like:
  %%% \blenderpointNamedOutputFilenameVideoA containing thehashthatisusedforthename
  %%% \blenderpointNamedOutputPrVideoA containing thehashthatisusedforthename
  %% See \robExtGetNamedOutputFilename to get them with \robExtGetNamedOutputFullPath
  name output/.code={%
    \def\robExtExecuteNamedOutput{%
      \expandafter\xdef\csname #1\endcsname{\robExtPrefixFilename\robExtFinalHash}%
      \expandafter\xdef\csname #1InCache\endcsname{\robExtAddCachePathAndName{\robExtFinalHash}}%
    }%
  },
  %%%%%%%%%%%%%%%%%%%%%%%%%%%%%%%%%%%%%
  %%% Compile in parallel automatically
  %%%%%%%%%%%%%%%%%%%%%%%%%%%%%%%%%%%%%
  compile in parallel/.style={compile in parallel after=#1},
  compile in parallel/.default=0,
  compile in parallel after/.code={%
    \gdef\robExt@compile@parallel@after{#1}%
  },
  compile in parallel after/.default=0,
  disable compile in parallel/.code={\let\robExt@compile@parallel@after\undefined},
  if windows/.code={%
    \robExtIfWindowsTF{\pgfkeysalso{zx@tmp@key/.estyle={\unexpanded{#1}},zx@tmp@key}}{}%
  },
  if unix/.code={%
    \robExtIfWindowsTF{}{\pgfkeysalso{zx@tmp@key/.estyle={\unexpanded{#1}},zx@tmp@key}}%
  },
  compile in parallel command/.store in=\robExt@compile@parallel@command,
  % Different backends to compile in parallel
  compile in parallel with gnu parallel/.style={
    compile in parallel command={parallel --jobs #1 :::: '\jobname-\robExtAddPrefixName{compile-missing-figures.sh}'},
  },
  compile in parallel with gnu parallel/.default={200\%},
  compile in parallel with xargs/.style={
    % Use " instead of ' as Windows does not consider ' as a valid surrounding for strings
    compile in parallel command={xargs -t -I "{}" -P #1 \robExtParallelShell\space "{}" < "\jobname-\robExtAddPrefixName{compile-missing-figures.sh}"},
  },
  % With 0 it spawns as many shells at possible, i.e. it compiles all 200 pictures at the same time, making the
  % system a laggy, and it is actually *slower* (2:05mn vs 1:15mn) than when using 16 threads.
  compile in parallel with xargs/.default={16},
  % By default, we compile with xargs and 16 processes:
  compile in parallel with xargs,
}

\ExplSyntaxOn
\int_new:N \g__robExt_number_figures_just_compiled_or_to_compile
\int_set:Nn \g__robExt_number_figures_just_compiled_or_to_compile {0}
% Will hold the list of input files to compile in parallel. This is needed to
% check at the end to find the processes that failed to build.
\seq_new:N \g__robExt_files_compiled_in_parallel

\AfterLastShipout{
  % Check if we want to run stuff in parallel
  \ifdefined\robExt@compile@parallel@after
    % Check if we must compile
    \ifdefined\robExt@compile@parallel@must@compile
      % Check if we can compile
      \bool_if:nTF { \sys_if_shell_unrestricted_p: || \cs_if_exist_p:N \robExtForceCompilation}
      {
        % Check if a command was provided
        \ifdefined\robExtParallelShell\else
          \sys_if_platform_windows:TF {
            \def\robExtParallelShell{cmd~/C}
          }{
            \def\robExtParallelShell{sh~-c}
          }
        \fi
        % We close the file or reading it is empty
        \iow_close:N \g__robExt_write_manually_compile_all_missing_figures_iow
        \msg_warning:nnx{robExt}{rerun because parallel}{\robExt@compile@parallel@command}
        \message{\robExt@compile@parallel@command}
        \sys_shell_now:x {\robExt@compile@parallel@command}
        % We check if some files failed to compile
        \seq_map_inline:Nn \g__robExt_files_compiled_in_parallel {
          \file_if_exist:xTF{#1.pdf}{}{
            \msg_error:nnxxx{robExt}{missing compiled pdf parallel}{#1}{\use:c {g__robExt_lines_error_#1:}}{\use:c {g__robExt_compilation_command_#1:}}
          }
        }
      } {
        \ifdefined\robExtFallbackManualMode
          \msg_warning:nn{robExt}{enabled parallel no shell escape}
        \else
          \msg_error:nn{robExt}{need shell escape parallel}
        \fi
      }
    \fi
  \fi
}
\ExplSyntaxOff

% Not really made for the end user
% It assumes that __ROBEXT_COMPILATION_COMMAND__ and __ROBEXT_TEMPLATE__ is set
\NewDocumentCommand{\robExtEvaluateCompileAndInclude}{}{%
  \ifdefined\robExtDisableExternalization%
    \pgfkeys{%
      /robExt/.cd,
      command if no externalization,
    }%
  \else%
    \pgfkeys{%
      /robExt/.cd,
      command if externalization,
    }%
    \ifdefined\robExtExecuteBefore\robExtExecuteBefore\fi%
    \robExtWriteFile{}%
    \robExtCompileFile{}%
    \robExtIncludeFile{}%
    \ifdefined\robExtExecuteNamedOutput\robExtExecuteNamedOutput\fi%
    \ifdefined\robExtExecuteAfter\robExtExecuteAfter\fi%
  \fi%
  \robExtDebugInfo{Finished to include the file.}%
}


%% #1: Arguments, #2: content to externalize
\NewDocumentCommand{\robExtCacheMe}{O{}+m}{%
  {% Group
    %% We store the input in a non-string element for efficiently implementing "auto forward"
    \edef\robExtUserInputCacheMe{\unexpanded{#2}}% \unexpanded is needed if the macro contains a #1
    \pgfkeys{%
      /robExt/.cd,%
      set placeholder={__ROBEXT_MAIN_CONTENT_ORIG__}{#2},%
      default style,%
      #1,
    }%
    \robExtEvaluateCompileAndInclude%
  }%
}
\let\cacheMe\robExtCacheMe

%% #1: Arguments, #2: content to externalize
\NewDocumentEnvironment{RobExtCacheMe}{m+b}{%
  \robExtCacheMe[#1]{#2}%
}{}
\let\CacheMe\RobExtCacheMe
\let\endCacheMe\endRobExtCacheMe

\NewDocumentEnvironment{RobExtCacheMeCode}{m}{%
  \RobExtPlaceholderFromCode{__ROBEXT_MAIN_CONTENT_ORIG__}%
}{%
  \endRobExtPlaceholderFromCode%
  \pgfkeys{%
    /robExt/.cd,%
    #1,%
  }%
  \robExtEvaluateCompileAndInclude%
}
\let\CacheMeCode\RobExtCacheMeCode
\let\endCacheMeCode\endRobExtCacheMeCode

\NewDocumentEnvironment{RobExtCacheMeNoContent}{+b}{%
  \robExtCacheMe[#1]{}%
}{}
\let\CacheMeNoContent\RobExtCacheMeNoContent
\let\endCacheMeNoContent\endRobExtCacheMeNoContent


%%%%%%%%%%%%%%%%%%%%%%%%%%%%%%%%%%%%%%%
%%%%%%%%%%% Forward a macro %%%%%%%%%%%
%%%%%%%%%%%%%%%%%%%%%%%%%%%%%%%%%%%%%%%
%% In this section, we provide commands to forward a macro, like
%% \def\myMacro{stuff}
%% \cacheMe<forward=\myMacro>
%% Note that this will only be forwarded if externalization is enabled (otherwise the macro already exists)
%% I published this code in this answer, thanks to egreg for some pointers
%% https://tex.stackexchange.com/questions/697120/how-to-extract-the-definition-of-a-macro-to-write-in-a-file-for-instance
\ExplSyntaxOn

\cs_new:Nn \__robExt_getCommandDefinitionFromXparse:N
{
  \str_clear_new:N \l__robExt_definition_str
  \GetDocumentCommandArgSpec { #1 }
  \str_set:Nx \l__robExt_definition_str {
    \c_backslash_str DeclareDocumentCommand
    \c_left_brace_str
    \c_backslash_str \cs_to_str:N #1
    \c_right_brace_str
    \c_left_brace_str \ArgumentSpecification \c_right_brace_str
    \c_left_brace_str
    \cs_replacement_spec:c { \cs_to_str:N #1 ~ code }
    \c_right_brace_str
  }%
}

\cs_new:Nn \__robExt_getCommandDefinitionFromDef:N
{
  \str_clear_new:N \l__robExt_definition_str
  \str_set:Nx \l__robExt_definition_str {
    \c_backslash_str def \c_backslash_str \cs_to_str:N #1
    \cs_argument_spec:c { \cs_to_str:N #1 }
    \c_left_brace_str
    \cs_replacement_spec:c { \cs_to_str:N #1 }
    \c_right_brace_str
  }%
}

\def\robExt@extractDefaultNewcommand#1#2#3#4{
  #4%
}%

\cs_new:Nn \__robExt_getCommandDefinitionFromNewcommand:N
{
  \str_clear_new:N \l__robExt_definition_str
  \str_clear_new:N \l__robExt_tmp
  %% \l__robExt_tmp_str will look like [#1]#2#3#4#5#6#7:
  \str_set:Nx \l__robExt_tmp_str {\cs_argument_spec:c { \c_backslash_str \cs_to_str:N #1 }}%
  %% We remove the brackets:
  \str_replace_all:Nnn \l__robExt_tmp_str {[} {}
  \str_replace_all:Nnn \l__robExt_tmp_str {]} {}
  \str_set:Nx \l__robExt_definition_str {
    %% Make sure the command does not exist
    \c_backslash_str providecommand \c_backslash_str \cs_to_str:N #1 \c_left_brace_str \c_right_brace_str
    %% this line will be like "\newdocumentcommand{\mymacro}"
    \c_backslash_str renewcommand \c_left_brace_str \c_backslash_str \cs_to_str:N #1 \c_right_brace_str
    %% It must have at least one argument, since elements without optional arguments are turned into \def
    %% Moreover, macros can't have more than 1 argument, so it will be easier to parse, we just need the last digit
    [\str_range:Nnn \l__robExt_tmp_str {-1} {-1}] %% <- this is the number of mandatory arguments
    [\expandafter \robExt@extractDefaultNewcommand #1 ] %% <- this is the value of the default argument
    \c_left_brace_str
    \cs_replacement_spec:c { \c_backslash_str \cs_to_str:N #1 }
    \c_right_brace_str
  }%
}

% \robExtGetCommandDefinitionInMacro{\myMacro} will populate
% \l__robExt_definition_str and \robExtDefinitionCommand
% with a string to execute to write it properly.
\NewDocumentCommand{\robExtGetCommandDefinitionInMacro}{m}{
  \cs_if_exist:NTF #1 {
    \cs_if_exist:cTF {\cs_to_str:N #1 ~ code}{
      % xparse-based definition
      \__robExt_getCommandDefinitionFromXparse:N #1%
    } {
      \cs_if_exist:cTF {\c_backslash_str \cs_to_str:N #1}{
        % \newcommand-based definition
        \__robExt_getCommandDefinitionFromNewcommand:N #1
      }{
        % \def-based definition
        \__robExt_getCommandDefinitionFromDef:N #1
      }
    }
  }{
    \msg_error:nn{robExt}{The~macro~#1~does~not~exist}
  }
  \let\robExtDefinitionCommand\l__robExt_definition_str
}

\NewDocumentCommand{\robExtGetCommandDefinition}{m}{%
  \robExtGetCommandDefinitionInMacro{#1}%
  \l__robExt_definition_str%
}
\ExplSyntaxOff

%%%%%%%%%%%%%%%%%%%%%%%%%%%%%%%%%%%%%%%%
%%%%%%%%%%% Default presets %%%%%%%%%%%%
%%%%%%%%%%%%%%%%%%%%%%%%%%%%%%%%%%%%%%%%
%% We create here a few presets and placeholders useful later

%%%%%%
%%%%%% Group "special characters"
%%%%%%

%%%% Generic placeholders, practical to escape stuff
\robExtClearGroupPlaceholders{main}
\ExplSyntaxOn
\robExtPlaceholderFromString{__ROBEXT_LEFT_BRACE__}{\c_left_brace_str}
\robExtPlaceholderFromString{__ROBEXT_RIGHT_BRACE__}{\c_right_brace_str}
\robExtPlaceholderFromString{__ROBEXT_BACKSLASH__}{\c_backslash_str}
\robExtPlaceholderFromString{__ROBEXT_HASH__}{\c_hash_str}
\robExtPlaceholderFromString{__ROBEXT_PERCENT__}{\c_percent_str}
\robExtPlaceholderFromString{__ROBEXT_UNDERSCORE__}{\c_underscore_str}
\ExplSyntaxOff
\robExtCopyGroupPlaceholders{special characters}{main}
\robExtRegisterGroupPlaceholders{special characters}

%%%%%%
%%%%%% Group "latex"
%%%%%%

\robExtClearGroupPlaceholders{main}

\begin{RobExtPlaceholderFromCode}{__ROBEXT_LATEX__}
\documentclass[__ROBEXT_LATEX_OPTIONS__]{__ROBEXT_LATEX_DOCUMENT_CLASS__}
__ROBEXT_LATEX_PREAMBLE__
% most packages must be loaded before hyperref
% so we typically want to load hyperref here
__ROBEXT_LATEX_PREAMBLE_HYPERREF__
% some packages must be loaded after hyperref
__ROBEXT_LATEX_PREAMBLE_AFTER_HYPERREF__
\begin{document}%
__ROBEXT_LATEX_MAIN_CONTENT_WRAPPED__
\end{document}
\end{RobExtPlaceholderFromCode}

\robExtSetPlaceholder{__ROBEXT_LATEX_OPTIONS__}{}
\robExtSetPlaceholder{__ROBEXT_LATEX_DOCUMENT_CLASS__}{standalone}
\robExtSetPlaceholder{__ROBEXT_LATEX_PREAMBLE__}{}
\robExtSetPlaceholder{__ROBEXT_LATEX_PREAMBLE_HYPERREF__}{}
\robExtSetPlaceholder{__ROBEXT_LATEX_PREAMBLE_AFTER_HYPERREF__}{}
\robExtSetPlaceholder{__ROBEXT_LATEX_TRIM_LENGTH__}{30cm}

\begin{RobExtPlaceholderFromCode}{__ROBEXT_LATEX_MAIN_CONTENT_WRAPPED__}
__ROBEXT_LATEX_CREATE_OUT_FILE__%
\newsavebox\boxRobExt%
\begin{lrbox}{\boxRobExt}%
  __ROBEXT_MAIN_CONTENT__%
\end{lrbox}%
\usebox{\boxRobExt}%
__ROBEXT_LATEX_WRITE_DEPTH_TO_OUT_FILE__%
\end{RobExtPlaceholderFromCode}

\begin{RobExtPlaceholderFromCode}{__ROBEXT_LATEX_CREATE_OUT_FILE__}
%% We save the height/depth of the content by using a savebox:
\newwrite\writeRobExt%
\immediate\openout\writeRobExt=\jobname-out.tex%
\end{RobExtPlaceholderFromCode}

\begin{RobExtPlaceholderFromCode}{__ROBEXT_LATEX_WRITE_DEPTH_TO_OUT_FILE__}
\immediate\write\writeRobExt{%
  \string\def\string\robExtWidth{\the\wd\boxRobExt}%
  \string\def\string\robExtHeight{\the\ht\boxRobExt}%
  \string\def\string\robExtDepth{\the\dp\boxRobExt}%
}%
\end{RobExtPlaceholderFromCode}

%% Compilation commands
\robExtSetPlaceholder{__ROBEXT_LATEX_COMPILATION_COMMAND__}{__ROBEXT_LATEX_ENGINE__ __ROBEXT_LATEX_COMPILATION_COMMAND_OPTIONS__ "__ROBEXT_SOURCE_FILE__"}
\robExtSetPlaceholder{__ROBEXT_LATEX_COMPILATION_COMMAND_OPTIONS__}{-halt-on-error}
\robExtSetPlaceholder{__ROBEXT_LATEX_ENGINE__}{pdflatex}

\robExtConfigure{
  %%%% Allow users to specify which command to try (or try first) when replacing placeholders. Order matters.
  %%%% it looks ugly, but it is mostly for performance improvements.
  first placeholders latex/.style={
    first placeholders in compilation command={__ROBEXT_LATEX_COMPILATION_COMMAND__,__ROBEXT_LATEX_ENGINE__,__ROBEXT_LATEX_COMPILATION_COMMAND_OPTIONS__},
    first placeholders in template={__ROBEXT_LATEX__,__ROBEXT_LATEX_OPTIONS__,__ROBEXT_LATEX_DOCUMENT_CLASS__,__ROBEXT_LATEX_PREAMBLE__,__ROBEXT_LATEX_PREAMBLE_HYPERREF__,__ROBEXT_LATEX_PREAMBLE_AFTER_HYPERREF__,__ROBEXT_LATEX_MAIN_CONTENT_WRAPPED__,__ROBEXT_LATEX_TRIM_LENGTH__,__ROBEXT_LATEX_CREATE_OUT_FILE__,__ROBEXT_LATEX_WRITE_DEPTH_TO_OUT_FILE__,__ROBEXT_MAIN_CONTENT__,__ROBEXT_MAIN_CONTENT_ORIG__},
  },
  only placeholders latex/.style={
    only placeholders in compilation command={__ROBEXT_LATEX_COMPILATION_COMMAND__,__ROBEXT_LATEX_ENGINE__,__ROBEXT_LATEX_COMPILATION_COMMAND_OPTIONS__},
    only placeholders in template={__ROBEXT_LATEX__,__ROBEXT_LATEX_OPTIONS__,__ROBEXT_LATEX_DOCUMENT_CLASS__,__ROBEXT_LATEX_PREAMBLE__,__ROBEXT_LATEX_PREAMBLE_HYPERREF__,__ROBEXT_LATEX_PREAMBLE_AFTER_HYPERREF__,__ROBEXT_LATEX_MAIN_CONTENT_WRAPPED__,__ROBEXT_LATEX_TRIM_LENGTH__,__ROBEXT_LATEX_CREATE_OUT_FILE__,__ROBEXT_LATEX_WRITE_DEPTH_TO_OUT_FILE__,__ROBEXT_MAIN_CONTENT__,__ROBEXT_MAIN_CONTENT_ORIG__},
  },
  % some useful presets
  latex/.style={
    enable placeholders,
    % This way it appears before orig in the list: more efficient
    import placeholders from group={latex},
    set placeholder={__ROBEXT_MAIN_CONTENT__}{__ROBEXT_MAIN_CONTENT_ORIG__},
    import placeholder={__ROBEXT_MAIN_CONTENT_ORIG__},
    first placeholders latex,
    set template={__ROBEXT_LATEX__},
    set compilation command={__ROBEXT_LATEX_COMPILATION_COMMAND__},
    %% Configure the latex compilation engine
    use latexmk/.style={
      set placeholder={__ROBEXT_LATEX_ENGINE__}{latexmk},
    },
    use lualatex/.style={
      set placeholder={__ROBEXT_LATEX_ENGINE__}{lualatex},
    },
    use xelatex/.style={
      set placeholder={__ROBEXT_LATEX_ENGINE__}{xelatex},
    },
    set latex options/.style={
      set placeholder={__ROBEXT_LATEX_OPTIONS__}{##1},
    },
    add to latex options/.style={
      add to placeholder no space={__ROBEXT_LATEX_OPTIONS__}{,##1},
    },
    set documentclass/.style={
      set placeholder={__ROBEXT_LATEX_DOCUMENT_CLASS__}{##1},
    },
    set preamble/.style={
      set placeholder={__ROBEXT_LATEX_PREAMBLE__}{##1},
    },
    add to preamble/.style={
      add to placeholder={__ROBEXT_LATEX_PREAMBLE__}{##1},
    },
    add before main content/.style={
      add before placeholder no space={__ROBEXT_MAIN_CONTENT__}{##1},
    },
    add before preamble/.style={
      add before placeholder={__ROBEXT_LATEX_PREAMBLE__}{##1},
    },
    set preamble hyperref/.style={
      set placeholder={__ROBEXT_LATEX_PREAMBLE_HYPERREF__}{##1},
    },
    add to preamble hyperref/.style={
      add to placeholder={__ROBEXT_LATEX_PREAMBLE_HYPERREF__}{##1},
    },
    set preamble after hyperref/.style={
      set placeholder={__ROBEXT_LATEX_PREAMBLE_AFTER_HYPERREF__}{##1},
    },
    add to preamble after hyperref/.style={
      add to placeholder={__ROBEXT_LATEX_PREAMBLE_AFTER_HYPERREF__}{##1},
    },
    do not wrap code/.style={
      set placeholder={__ROBEXT_LATEX_MAIN_CONTENT_WRAPPED__}{__ROBEXT_MAIN_CONTENT__},
    },    
    add to includegraphics options={trim=__ROBEXT_LATEX_TRIM_LENGTH__ __ROBEXT_LATEX_TRIM_LENGTH__ __ROBEXT_LATEX_TRIM_LENGTH__ __ROBEXT_LATEX_TRIM_LENGTH__},
    add to latex options={margin=__ROBEXT_LATEX_TRIM_LENGTH__},
    do not add margins/.style={
      set placeholder={__ROBEXT_LATEX_TRIM_LENGTH__}{0cm}
    },
  },
  % U
  tikz/.style={
    latex,
    add to preamble={\usepackage{tikz}},
  },
  tikzpicture/.style={
    tikz,
    set placeholder no import={__ROBEXT_MAIN_CONTENT__}{\begin{tikzpicture}__ROBEXT_MAIN_CONTENT_ORIG__\end{tikzpicture}},
  },
}


\robExtCopyGroupPlaceholders{latex}{main}
\robExtRegisterGroupPlaceholders{latex}


%%%%%% 
%%%%%% Group "python"
%%%%%%

\robExtClearGroupPlaceholders{main}

\begin{RobExtPlaceholderFromCode}{__ROBEXT_PYTHON__}
__ROBEXT_PYTHON_IMPORT__
__ROBEXT_PYTHON_MAIN_CONTENT_WRAPPED__
\end{RobExtPlaceholderFromCode}

\robExtSetPlaceholder{__ROBEXT_PYTHON_IMPORT__}{}

\begin{RobExtPlaceholderFromCode}{__ROBEXT_PYTHON_MAIN_CONTENT_WRAPPED__}
# This file will be loaded in latex. Useful to pass data to the main document
f_out_write = open("__ROBEXT_OUTPUT_PREFIX__-out.tex", "w")

import os
import sys

def write_to_out(text):
    """Write to the -out.tex file that is loaded by default"""
    f_out_write.write(text)

def parse_args():
    args = {}
    if len(sys.argv) % 2 == 0:
        print("Error: the number of arguments must be even, as tuples of name and value")
        exit(1)
    for i in range(0,len(sys.argv)-1,2):
        args[sys.argv[i+1]] = sys.argv[i+2]
    return args

def get_cache_folder():
    '''
    Path of the cache folder. Warning: this works only when the python script
    is located in this cache folder (that should be true when it's called from LaTeX)
    '''
    return os.path.abspath(os.path.dirname(sys.argv[0])) 

def get_file_base():
    '''
    Outputs the base of the files (i.e. something like robExt-somehash, without any extension)
    '''
    return os.path.splitext(os.path.basename(sys.argv[0]))[0] # __file__ does not work as it refers to the library

def get_current_script():
    '''
    Outputs the path of the current script
    '''
    return os.path.abspath(sys.argv[0]) # __file__ does not work as it refers to the library

    
def get_filename_from_extension(extension):
    '''
    If you want to create a file with extension 'extension' (with the appropriate base name), this command
    is for you. For instance get_filename_from_extension(".mp4") would return something like
    robExt-somehash.mp4
    the extension can also be like get_filename_from_extension("-out.tex") etc.
    '''
    return os.path.join(get_cache_folder(), get_file_base() + extension)

def get_verbatim_output():
    '''Returns the path to -out.txt that is read by verbatim output'''    
    return get_filename_from_extension("-out.txt")

def get_pdf_output():
    '''Returns the path to -out.txt that is read by verbatim output'''    
    return get_filename_from_extension(".pdf")

    
def finished_with_no_error():
    '''
    Call this at the end of your script. This creates the path of the final pdf file that should be
    created (otherwise robust-externalize will think that the compilation failed)
    '''
    if not os.path.exists(get_filename_from_extension(".pdf")):
        # we create an empty path
        with open(get_filename_from_extension(".pdf"), 'w') as f:
            pass

### Starting main content
__ROBEXT_MAIN_CONTENT__
### Ending main content
__ROBEXT_PYTHON_FINISHED_WITH_NO_ERROR__
f_out_write.close()
\end{RobExtPlaceholderFromCode}

% It is annoying to manually call finished_with_no_error(), but it is handy to be able to disable it.
\begin{RobExtPlaceholderFromCode}{__ROBEXT_PYTHON_FINISHED_WITH_NO_ERROR__}
finished_with_no_error()
\end{RobExtPlaceholderFromCode}

%% On windows, python3 does not exist, and python points to python3. On linux, it seems to depend, at least on
%% my system it points to python3 as well.
\robExtSetPlaceholder{__ROBEXT_PYTHON_EXEC__}{python}

\robExtConfigure{
  python/.style={
    enable placeholders,
    import placeholders={__ROBEXT_PYTHON__,__ROBEXT_PYTHON_IMPORT__,__ROBEXT_PYTHON_MAIN_CONTENT_WRAPPED__,__ROBEXT_PYTHON_FINISHED_WITH_NO_ERROR__,__ROBEXT_PYTHON_EXEC__},
    set compilation command={__ROBEXT_PYTHON_EXEC__ "__ROBEXT_SOURCE_FILE__"},
    set template={__ROBEXT_PYTHON__},
    print verbatim if no externalization,
    force python3/.style={
      set placeholder={__ROBEXT_PYTHON_EXEC__}{python3}
    },
    add import/.style={
      add to placeholder no space={__ROBEXT_PYTHON_IMPORT__}{##1^^J},
    },
  }
}

\robExtCopyGroupPlaceholders{python}{main}
\robExtRegisterGroupPlaceholders{python}


%%%%%% 
%%%%%% Group "python print code result"
%%%%%%

\robExtClearGroupPlaceholders{main}

%% A style to print both the code and the result:
\begin{RobExtPlaceholderFromCode}{__ROBEXT_PYTHON_PRINT_CODE_RESULT_TEMPLATE_BEFORE__}
# File where print("bla") should be redirected
# get_filename_from_extension("-foo.txt") will give you the path of the file
# in the cache that looks like robExt-somehash-foo.txt
print_file = open(get_filename_from_extension("-print.txt"),  "w")
sys.stdout = print_file
# This code will read the current code, and extract the lines between
# that starts with "### CODESTARTSHERE" and "### CODESTOPSHERE", and will write
# it into the *-code.text (we do not want to print all these functions in
# the final code)
with open(get_filename_from_extension("-code.txt"), "w") as f:
    # The current script has extension .tex
    with open(get_current_script(), "r") as script:
        should_write = False
        for line in script:
            if line.startswith("### CODESTARTSHERE"):
                should_write = True
            elif line.startswith("### CODESTOPSHERE"):
                should_write = False
            elif "HIDEME" in line:
                pass
            else:
                if should_write:
                    f.write(line)
### CODESTARTSHERE
\end{RobExtPlaceholderFromCode}


\begin{RobExtPlaceholderFromCode}{__ROBEXT_PYTHON_PRINT_CODE_RESULT_TEMPLATE_AFTER__}
### CODESTOPSHERE
print_file.close()
\end{RobExtPlaceholderFromCode}

\robExtSetPlaceholder{__ROBEXT_PYTHON_TCOLORBOX_PROPS__}{colback=red!5!white,colframe=red!75!black}
\robExtSetPlaceholder{__ROBEXT_PYTHON_CODE_MESSAGE__}{}
\robExtSetPlaceholder{__ROBEXT_PYTHON_RESULT_MESSAGE__}{Output:}
\robExtSetPlaceholder{__ROBEXT_PYTHON_LSTINPUT_STYLE__}{frame=single, breakindent=.5\textwidth, frame=single, breaklines=true, style=mypython}

\robExtConfigure{
  python print code and result/.style={
    python,
    import placeholders={__ROBEXT_PYTHON_PRINT_CODE_RESULT_TEMPLATE_BEFORE__,__ROBEXT_PYTHON_PRINT_CODE_RESULT_TEMPLATE_AFTER__,__ROBEXT_PYTHON_TCOLORBOX_PROPS__,__ROBEXT_PYTHON_CODE_MESSAGE__,__ROBEXT_PYTHON_RESULT_MESSAGE__,__ROBEXT_PYTHON_LSTINPUT_STYLE__},
    add before placeholder no space={__ROBEXT_MAIN_CONTENT__}{__ROBEXT_PYTHON_PRINT_CODE_RESULT_TEMPLATE_BEFORE__},
    add to placeholder no space={__ROBEXT_MAIN_CONTENT__}{__ROBEXT_PYTHON_PRINT_CODE_RESULT_TEMPLATE_AFTER__},
    set title/.style={
      set placeholder={__MY_TITLE__}{##1},
    },
    set title={Python code},
    custom include command={
      % Useful to replace __MY_TITLE__:
      \evalPlaceholder{
        \begin{tcolorbox}[title=__MY_TITLE__,__ROBEXT_PYTHON_TCOLORBOX_PROPS__]
          __ROBEXT_PYTHON_CODE_MESSAGE__%
          \lstinputlisting[__ROBEXT_PYTHON_LSTINPUT_STYLE__]{\robExtAddCachePathAndName{\robExtFinalHash-code.txt}}
          __ROBEXT_PYTHON_RESULT_MESSAGE__%
          \verbatiminput{\robExtAddCachePathAndName{\robExtFinalHash-print.txt}}
        \end{tcolorbox}
      }
    },
  },
}

\robExtCopyGroupPlaceholders{python print code result}{main}
\robExtRegisterGroupPlaceholders{python print code result}


%%%%%% 
%%%%%% Group "verbatim"
%%%%%%

\robExtClearGroupPlaceholders{main}

\robExtConfigure{
  verbatim text/.style={
    enable placeholders,
    set template={__ROBEXT_MAIN_CONTENT__},
    custom include command={\evalPlaceholder{\verbatiminput{\robExtAddCachePathAndName{\robExtFinalHash.tex}}}},
    %% Apparently this works on windows as well https://stackoverflow.com/questions/1702762/how-can-i-create-an-empty-file-at-the-command-line-in-windows
    set compilation command={echo "" > __ROBEXT_OUTPUT_PDF__},
  },
  verbatim text no include/.style={
    verbatim text,
    custom include command={\evalPlaceholder{%
        \xdef\robExtPathToInput{\robExtAddCachePathAndName{\robExtFinalHash.tex}}%
      }%
    }%
  },
}

\robExtCopyGroupPlaceholders{verbatim}{main}
\robExtRegisterGroupPlaceholders{verbatim}


%%%%%% 
%%%%%% Group "bash"
%%%%%%

\robExtClearGroupPlaceholders{main}

\begin{RobExtPlaceholderFromCode}{__ROBEXT_BASH_TEMPLATE__}
# Quit if there is an error
set -e
outputTxt="__ROBEXT_OUTPUT_PREFIX__-out.txt"
outputTex="__ROBEXT_OUTPUT_PREFIX__-out.tex"
outputPdf="__ROBEXT_OUTPUT_PDF__"
__ROBEXT_MAIN_CONTENT__
# Create the pdf file to certify that no compilation error occured
touch "${outputPdf}"
\end{RobExtPlaceholderFromCode}

\robExtSetPlaceholder{__ROBEXT_BASH_SHELL__}{bash}

\robExtConfigure{
  bash/.style={
    enable placeholders,
    import placeholders={__ROBEXT_BASH_TEMPLATE__,__ROBEXT_BASH_SHELL__},
    set compilation command={__ROBEXT_BASH_SHELL__ "__ROBEXT_SOURCE_FILE__"},
    set template={__ROBEXT_BASH_TEMPLATE__},
    print verbatim if no externalization,
  }
}

\robExtCopyGroupPlaceholders{bash}{main}
\robExtRegisterGroupPlaceholders{bash}

%%%%%% 
%%%%%% Group "default", available in all styles
%%%%%%

\robExtClearGroupPlaceholders{main}

% This additional level of indirection is made to allow an easier wrapping
% of \begin{tikzpicture} ... \end{tikzpicture} for instance.
% The original behavior (modify __ROBEXT_MAIN_CONTENT__ directly)
% was not really practical as if you use both |tikz| and |\cacheCommand|, it would wrap
% the environment twice.
\robExtSetPlaceholder{__ROBEXT_MAIN_CONTENT__}{__ROBEXT_MAIN_CONTENT_ORIG__}
\setPlaceholder{__ROBEXT_INCLUDEGRAPHICS_OPTIONS__}{}
\setPlaceholder{__ROBEXT_INCLUDEGRAPHICS_FILE__}{\robExtAddCachePathAndName{\robExtFinalHash.pdf}}

\robExtConfigure{
  % Allow nice optimizations
  all placeholders have underscores,
  set includegraphics options/.style={
    set placeholder no import={__ROBEXT_INCLUDEGRAPHICS_OPTIONS__}{#1},
  },
  add to includegraphics options/.style={
    add to placeholder no space no import={__ROBEXT_INCLUDEGRAPHICS_OPTIONS__}{,#1},
  },
  set placeholder no import={__ROBEXT_VERBATIM_COMMAND__}{\verbatiminput},
  % We expect the program to write in __ROBEXT_OUTPUT_PREFIX__-out.txt
  verbatim output/.style={
    import placeholders={__ROBEXT_VERBATIM_COMMAND__},
    custom include command={%
      \evalPlaceholder{%
        __ROBEXT_VERBATIM_COMMAND__{__ROBEXT_CACHE_FOLDER____ROBEXT_OUTPUT_PREFIX__-out.txt}%
      }%
    },
  },
  % Mostly for debugging purpose
  print command and source/.style={
    enable manual mode,
    custom include command advanced={%
      \evalPlaceholder{%
        Command: (run in folder \texttt{__ROBEXT_CACHE_FOLDER__})
        \robExtPrintPlaceholder{__ROBEXT_COMPILATION_COMMAND__}
        Dependencies:
        \verbatiminput{__ROBEXT_CACHE_FOLDER____ROBEXT_OUTPUT_PREFIX__.deps}%
        Source (in \texttt{__ROBEXT_CACHE_FOLDER____ROBEXT_OUTPUT_PREFIX__.tex}):
        \verbatiminput{__ROBEXT_CACHE_FOLDER____ROBEXT_OUTPUT_PREFIX__.tex}%
      }%
    },
  },
  debug/.style={
    print command and source
  },
}


\robExtCopyGroupPlaceholders{default}{main} % we do not reset it as it will be the default imported group
\robExtRegisterGroupPlaceholders{default}

%%%%%%%%%%%%%%%%%%%%%%%%%%%%%%%%%%%%%%%%%%%%%%%%%%%%%%%%%%%%%%%%%%%%%%%
%%%%%%%%%% Automatically cache environments, tikzpicture etc %%%%%%%%%%
%%%%%%%%%%%%%%%%%%%%%%%%%%%%%%%%%%%%%%%%%%%%%%%%%%%%%%%%%%%%%%%%%%%%%%%

%%%%%% 
%%%%%% Parse tikz
%%%%%%

%% The parser is mostly taken from
%% https://github.com/sasozivanovic/memoize/blob/master/memoize-tikz.tex

\newtoks\robExt@temptoks
\def\robExt@apptotoks#1#2{\expandafter#1\expandafter{\the#1#2}}

\def\robExt@tikz{%
  \robExt@temptoks={}%
  \robExt@tikz@anim%
}
\def\robExt@tikz@anim{%
  \pgfutil@ifnextchar[{\robExt@tikz@opt}{%
    \pgfutil@ifnextchar:{\robExt@tikz@anim@a}{%
      \robExt@tikz@code}}%]
}%
\def\robExt@tikz@anim@a#1=#2{%
  \robExt@apptotoks\robExt@temptoks{#1={#2}}%
  \robExt@tikz@anim%
}%
\def\robExt@tikz@opt[#1]{%
  \robExt@apptotoks\robExt@temptoks{[#1]}%
  \robExt@tikz@code%
}
\def\robExt@tikz@code{%
  \pgfutil@ifnextchar\bgroup\robExt@tikz@braced\robExt@tikz@single%
}
\def\robExt@tikz@braced#1{\robExt@apptotoks\robExt@temptoks{{#1}}\robExt@tikz@done}
\def\robExt@tikz@single#1;{\robExt@apptotoks\robExt@temptoks{#1;}\robExt@tikz@done}
\def\robExt@tikz@done{%
  \edef\robExt@marshal{%
    \noexpand\robExtRescanHashRobust{\noexpand\robExtCacheMe[\expandonce{\robExt@tmp@orig@args}, \expandonce{\robExt@tmp@args}]{%
        \noexpand\tikz\the\robExt@temptoks%
      }%
    }%
  }%
  \robExt@marshal%
}

\NewDocumentCommand{\robExtExternalizeAllTikzpictures}{O{tikz}O{tikzpicture}O{<>}}{%
  \robExtCacheEnvironment[#3]{tikzpicture}{#2}%
  % Tikz has a more complicated parsing system
  \DeclareCommandCopy{\robExtCommandOrigtikz}{\tikz}% we save the function for later reset
  \robExtAddToCommandResetList{tikz}% we add it to the list of functions to reset
  \DeclareDocumentCommand{\tikz}{D#3{}}{%
    \def\robExtCommandOrigName{tikz}% we specify the name of the current environment
    \edef\robExt@tmp@args{\detokenize{##1}}% we specify the name of the current environment
    \edef\robExt@tmp@orig@args{\detokenize{#1}}% we specify the name of the current environment
    % we start the parsing
    \robExt@tikz%
  }%
}
\let\cacheTikz\robExtExternalizeAllTikzpictures

%% The cached version
\DeclareDocumentEnvironment{tikzpictureC}{D<>{}O{}O{}}{%
  \begin{CacheMe}{tikzpicture,#1}[#2]%
}{\end{CacheMe}}%

%%%%%% 
%%%%%% The generic functions
%%%%%%

%% Usage: \robExtCacheEnvironment{myenv}
\NewDocumentCommand{\robExtCacheEnvironment}{O{<>}mm}{%
  %% We need to save the original environment to avoid infinite recursion if we disable externalization
  %% https://tex.stackexchange.com/questions/695391/why-isnt-my-environment-restored/695398
  \NewEnvironmentCopy{robExtEnvironmentOrig#2}{#2}%
  \robExtAddToEnvironmentResetList{#2}%
  \DeclareDocumentEnvironment{#2}{D#1{}}{%
    \def\robExtEnvironmentOrigName{#2}%
    \CacheMe{%
      #3,%
      set placeholder no import={__ROBEXT_MAIN_CONTENT__}{\begin{#2}__ROBEXT_MAIN_CONTENT_ORIG__\end{#2}},%
      ##1%
    }%
  }{\endCacheMe}%
}
\let\cacheEnvironment\robExtCacheEnvironment

% \robExtShowPlaceholder{__ROBEXT_MAIN_CONTENT__}
% \robExtShowPlaceholdersContents
%%%% Borrowed and adapted from https://github.com/sasozivanovic/memoize/blob/master/xparse-arglist.sty
%%%% see also https://tex.stackexchange.com/questions/695662/automatically-wrap-a-macro/695734
%% The idea of the library is that it builds a string like
%% [#2]<#3>{#4}
%% in order to generate something like
%% \NewDocumentCommand{\myfunction}{D<>{}O{coucou}D<>{yes}m}
%% {
%%  \cacheMe[#1]{\myfunction[#2]<#3>{#4}}
%% }
%%%%   _________________________________________________________________

\def\robExtArgumentList{%
  \expandafter\robExt@arglist\expandafter0\ArgumentSpecification.%
}

\def\robExt@arglist#1#2{%
  \ifcsname robExt@arglist@#2\endcsname
    \csname robExt@arglist@#2\expandafter\expandafter\expandafter\endcsname
  \else
    \expandafter\robExt@arglist@error
  \fi
  \expandafter{\the\numexpr#1+1\relax}%
}

% \robExt@arglist@...: #1 = the argument number
\def\robExt@arglist@m#1{\noexpand\unexpanded{{#####1}}\robExt@arglist{#1}}
\def\robExt@arglist@r#1#2#3{\noexpand\unexpanded{#2#####1#3}\robExt@arglist{#1}}
\def\robExt@arglist@R#1#2#3#4{\noexpand\unexpanded{#2#####1#3}\robExt@arglist{#1}}
\def\robExt@arglist@v#1{{Handled commands with verbatim arguments are not
    supported}\robExt@arglist{#1}} % error
\def\robExt@arglist@b#1{{This is not the way to handle
    environment}\robExt@arglist{#1}} % error
\def\robExt@arglist@o#1{\noexpand\unexpanded{[#####1]}\robExt@arglist{#1}}
\def\robExt@arglist@d#1#2#3{\noexpand\unexpanded{#2#####1#3}\robExt@arglist{#1}}
\def\robExt@arglist@O#1#2{\noexpand\unexpanded{[#####1]}\robExt@arglist{#1}}
% \def\robExt@arglist@O#1#2{\noexpand\unexpanded{[#####1]}\robExt@arglist{#1}}
\def\robExt@arglist@D#1#2#3#4{\noexpand\unexpanded{#2#####1#3}\robExt@arglist{#1}}
\def\robExt@arglist@s#1{\noexpand\IfBooleanT{#####1}{*}\robExt@arglist{#1}}
\def\robExt@arglist@t#1#2{\noexpand\IfBooleanT{#####1}{#2}\robExt@arglist{#1}}
\csdef{robExt@arglist@+}#1{\expandafter\robExt@arglist\expandafter{\the\numexpr#1-1\relax}}%
\csdef{robExt@arglist@.}#1{}
% e,E: Embellishments are not supported.
% > Argument processors are not supported. And how could they be?
\def\robExt@arglist@error#1.{{Unknown argument type}}
%%%% _________________________________________________________________

%% Since the very first occurrence of D is not forwarded to the function
%% we discard it:
%% The first argument of \robExt@arglist@D is the number of the argument
%% so #2#####1#3 reads as #2 = <, #### = #, #1 = > #3 = default value that can be discarded since it is already part of the argument spec.

\def\robExt@arglist@D#1#2#3#4{%
  \noexpand\unexpanded{%
    \ifnum #1=1 % If it is the first argument D, then we do not add anything to the argument string
    \else #2#####1#3\fi}\robExt@arglist{#1}%
}

%% Usage: \robExtCacheCommand[delimiters]{name command}[argument specification]{style}
%% Inspired and modified from
%% https://github.com/sasozivanovic/memoize/blob/81d960aa547148bdb38fea89917eda1476c9bace/memoize.sty#L744
\NewDocumentCommand{\robExtCacheCommand}{O{<>}mom}{%
  %% We get the specification of the command, like "O{}mm"
  \IfNoValueTF{#3}{%
    \expandafter\GetDocumentCommandArgSpec\csname #2\endcsname%
  }{%
    \def\ArgumentSpecification{#3}%
  }%
  \edef\ArgumentSpecification{D#1{}\ArgumentSpecification}%
  \robExtAddToCommandResetList{#2}%
  % We copy the original definition for later (if externalization is disabled)
  \expandafter\DeclareCommandCopy\csname robExtCommandOrig#2\expandafter\endcsname\csname #2\endcsname%
  \edef\robExt@marshal{%
    \noexpand\DeclareDocumentCommand%
      \expandonce{\csname #2\endcsname}%
      {\expandonce{\ArgumentSpecification}}%
      {%
        \noexpand\def\noexpand\robExtCommandOrigName{#2}%
        % todo: add a hook for users setup; prevent user from changing \MemoizeWrapper?
        \edef\noexpand\robExt@marshal{%
        \noexpand\noexpand\noexpand\robExtRescanHashRobust{\noexpand\noexpand\noexpand\robExtCacheMe[\detokenize{#4}, \noexpand\detokenize{####1}]{%
          \noexpand\noexpand\expandonce{\csname #2\endcsname}%
            \robExtArgumentList%
          }%
        }%
      }%
      % For debug
      %\noexpand\show\noexpand\robExt@marshal%
      \noexpand\robExt@marshal%
    }%
  }%
  % for debug
  %\show\robExt@marshal%
  \robExt@marshal%
}
\let\cacheCommand\robExtCacheCommand


\makeatother
\robExtConfigure{
  enable optimizations
}  
\usepackage{zx-calculus}
\usepackage{forest}
\setlength{\marginparwidth}{2cm}
\usepackage{todonotes}
% Loads the great package that produces tikz-like manual (see also tikzcd for examples)
\input{pgfmanual-en-macros.tex} % Is supposed to be included in recent TeX distributions, but I get errors...


%% For verbatim environments:

\NewDocumentEnvironment{codeAndResult}{}{\XSIMfilewritestart{\jobname-codeAndResult-tmp-file-you-can-remove.tmp}}%
{%
  \XSIMfilewritestop%
  \par
  \medskip
  \noindent\colorbox{graphicbackground}{\noindent\begin{minipage}{1.0\linewidth}
  {\input{\jobname-codeAndResult-tmp-file-you-can-remove.tmp}}
  \end{minipage}}
  % I don't know why, but sometimes this prints nothing, while input works:
  %{\codeexample[width=0pt, leave comments, vbox, from file={\jobname-codeAndResult-tmp-file-you-can-remove.tmp}]}%
  {\codeexample[code only, from file={\jobname-codeAndResult-tmp-file-you-can-remove.tmp}]}%
}

\usepackage{makeidx} % Produces an index of commands.
\makeindex % Useful or not index will be created
\usepackage{alertmessage} % For warning, info...
\newcommand{\mylink}[2]{\href{#1}{#2}\footnote{\url{#1}}}
\usepackage{verbatim}
\usepackage{mathtools}
\usepackage{float} %figure inside minipage
\usepackage{pythonhighlight}
\usepackage{tcolorbox}

\usepackage{listings}
\definecolor{codegreen}{rgb}{0,0.6,0}
\definecolor{codegray}{rgb}{0.5,0.5,0.5}
\definecolor{codepurple}{rgb}{0.58,0,0.82}
\definecolor{backcolour}{rgb}{0.95,0.95,0.92}

\lstdefinestyle{mystyle}{
    backgroundcolor=\color{backcolour},   
    commentstyle=\color{codegreen},
    keywordstyle=\color{magenta},
    numberstyle=\tiny\color{codegray},
    stringstyle=\color{codepurple},
    basicstyle=\ttfamily\footnotesize,
    breakatwhitespace=false,         
    breaklines=true,                 
    captionpos=b,                    
    keepspaces=true,                 
    numbers=left,                    
    numbersep=5pt,                  
    showspaces=false,                
    showstringspaces=false,
    showtabs=false,                  
    tabsize=2
}

\lstset{style=mystyle}


\usepackage[hidelinks]{hyperref}
\usepackage{cleveref}

\begin{document}
%%% Title: thanks tikzcd for the styling
\begin{center}
  \vspace*{1em} % Thanks tikzcd
  \tikz\node[scale=1.2]{%
    \color{gray}\Huge\ttfamily \char`\{\raisebox{.09em}{\textcolor{red!75!black}{robust\raisebox{-0.1em}{-}externalize}}\char`\}};

  \vspace{0.5em}
  {\Large\bfseries Cache anything (\tikzname, tikz-cd, python…),\\in a robust, efficient and pure way.}

  \vspace{1em}
  {Léo Colisson \quad Version 2.1}\\[3mm]
  {\href{https://github.com/leo-colisson/robust-externalize}{\texttt{github.com/leo-colisson/robust-externalize}}}
\end{center}

\tableofcontents

\bigskip

\textbf{WARNING: This library is young and might lack some testing (and is an important rewrite of a previous version), but v2.0 is definitely usable. Even if we try to stay backward compatible, the only guaranteed way to be immune to changes is to copy/paste the library in your main project folder. Please report any bug to \url{https://github.com/leo-colisson/robust-externalize}!}

\textbf{WARNING 2: the coming version 2.1 improves significantly the compilation time compared with 2.0 that was surprisingly slow. Now, it is fairly well optimized (but the code needs a bit of cleaning and documentation before doing an official release)}


\section{A taste of this library}

This library allows you to cache any language: not only \LaTeX{} documents and \tikzname{} images, taking into account depth and overlays, but also arbitrary code. To use it, make sure to have v2.0 installed (see instructions below), and load:
\begin{verbatim}
\usepackage{robust-externalize}
\robExtConfigure{
  % To display instructions in the PDF to manually compile pictures if
  % you forgot/do not want to compile with -shell-escape
  enable fallback to manual mode,
}
\end{verbatim}
Then type something like this (note the |C| for |cached| at the end of |tikzpictureC|, see below to override the original command), and compile with |pdflatex -shell-escape yourfile.tex| (or \textcolor{red}{\textbf{read below if you do not want to use }}|-shell-escape|):
\begin{codeexample}[width=0pt,vbox]
  \robExtConfigure{
    add to preset={tikz}{
      % we load some packages that will be loaded by figures based on the tikz preset
      add to preamble={\usepackage{pifont}}
    }
  }
  The next picture is cached %
  \begin{tikzpictureC}[baseline=(A.base)]
    \node[fill=red, rounded corners](A){My node that respects baseline \ding{164}.};
    \node[fill=red, rounded corners, opacity=.3,overlay] at (A.north east){I am an overlay text};
  \end{tikzpictureC} and you can see that overlay and depth works.
\end{codeexample}

You can also cache arbitrary code (e.g.\ python). You can also define arbitrary compilation commands, inclusion commands, and presets to fit you need. For instance, you can create a preset to obtain:

\begin{codeAndResult}
\begin{CacheMeCode}{python print code and result, set title={The for loop}}
for name in ["Alice", "Bob"]:
    print(f"Hello {name}")
\end{CacheMeCode}
\end{codeAndResult}

Actually, we also provide this style by default (and explain how to write it yourself), just make sure to load:
\begin{verbatim}
\usepackage{pythonhighlight}
\usepackage{tcolorbox}
\end{verbatim}

You can also cache any environment and command using something like:
\begin{itemize}
\item For environments:
\begin{codeAndResult}
% We cache the environment forest
\cacheEnvironment{forest}{latex, add to preamble={\usepackage{forest}}}
  
A forest tree automatically cached %
\begin{forest}
  [VP
    [DP[John]]
    [V’
      [V[sent]]
      [DP[Mary]]
      [DP[D[a]][NP[letter]]]
    ]
  ]
\end{forest}\\
And one not cached: %
\begin{forest}<disable externalization>
  [VP
    [DP[John]]
    [V’
      [V[sent]]
      [DP[Mary]]
      [DP[D[a]][NP[letter]]]
    ]
  ]
\end{forest}
\end{codeAndResult}
People interested by |tikz-cd| might like this example:
\begin{codeAndResult}
\cacheEnvironment{tikzcd}{tikz, add to preamble={\usepackage{tikz-cd}}}
\begin{tikzcd}
  A \rar & B
\end{tikzcd}
\end{codeAndResult}
\item For commands:
\begin{codeAndResult}
% We cache the command \zx
% O{} = optional argument, m = mandatory argument.
% Autodetected if command defined with xparse (NewDocumentCommand)
% (in this example it is redundant since zx IS defined with xparse syntax)
\cacheCommand{zx}[O{}O{}O{}m]{latex, add to preamble={\usepackage{zx-calculus}}}
  
A ZX figure %
\zx{
  \zxX{\alpha} \rar \ar[d,C] & \zxZ*{a\pi} \\
  \zxZ{\beta} \rar           & \zxX*{b\pi}
}, we can disable externalization: %
\zx<disable externalization>{
  \zxX{\alpha} \rar \ar[d,C] & \zxZ*{a\pi} \\
  \zxZ{\beta} \rar           & \zxX*{b\pi}
} or add more options: %
\zx<add to preamble={\usepackage{amsmath}\def\hello#1{Hello #1.}}>{
  \zxX{\alpha} \rar \ar[d,C] & \zxZ{\text{\hello{Alice}}} \\
  \zxZ{\beta} \rar           & \zxX*{b\pi}
}
\end{codeAndResult}
\end{itemize}
You can also cache all tikz pictures by default, except those that are run from some commands (as remember picture does not work yet, this is important). For instance, this code will allow you to freely use the |todonotes| package (that uses internally tikz but is not cachable with this library since it uses remember picture) while caching all other elements:
\begin{codeexample}[code only]
  %% this will cache \tikz and tikzpictures by default (using the tikz preset)
  \cacheTikz
  %% possibly add stuff to the tikz preset
  \robExtConfigure{
    add to preset={tikz}{
      add to preamble={
        \def\whereIAm{(I am cached)}
      }
    },
  }%
  %% We say not to cache the todo commands:
  \cacheCommand{todo}[O{}m]{disable externalization}%
  \todo{Check how to use cacheCommand and cacheEnvironment}
  \tikz[baseline=(A.base)] \node(A)[rounded corners,fill=green]{my cached node \whereIAm};
\end{codeexample}

will gives you: {%
  %% this will cache \tikz and tikzpictures by default (using the tikz preset)
  \cacheTikz
  %% possibly add stuff to the tikz preset
  \robExtConfigure{
    add to preset={tikz}{
      add to preamble={
        \def\whereIAm{(I am cached)}
      }
    },
  }%
  %% We say not to cache the todo commands:
  \cacheCommand{todo}[O{}m]{disable externalization}%
  \todo{Check how to use cacheCommand and cacheEnvironment}
  \tikz[baseline=(A.base)] \node(A)[rounded corners,fill=green]{my cached node \whereIAm};
}

Note that most options can either be specified anywhere in the document like
\begin{verbatim}
\robExtConfigure{
  disable externalization
}
\end{verbatim}
, inside a preset (that is nothing more than a pgf style), or in the options of the cached environment (either inside |CacheMe| or inside the first parameter |<...>| for automatically cached environments):
\begin{codeexample}[vbox]
  \robExtConfigure{
    new preset={my preset with xcolor}{
      latex, % load the latex preset, you can also use the tikz preset and more
      add to preamble={
        % load xcolor in the preamble of the cached document
        \usepackage{xcolor}
      },
    },
  }
  \begin{CacheMe}{my preset with xcolor, add to preamble={\def\hello#1{Hello #1!}} }
    \begin{minipage}{5cm}
      \textcolor{red}{\hello{Alice} See, you can cache any data, including minipages,
        define macros, combine presets, override a single picture or a whole part of
        a document etc.}
    \end{minipage}
  \end{CacheMe}
\end{codeexample}
(see that |CacheMe| can be used to cache arbitrary pictures)

Or include images generated in python:

\begin{codeexample}[code only]
\begin{CacheMeCode}{python, set includegraphics options={width=.8\linewidth}}
import matplotlib.pyplot as plt
year = [2014, 2015, 2016, 2017, 2018, 2019]  
tutorial_count = [39, 117, 111, 110, 67, 29]
plt.plot(year, tutorial_count, color="#6c3376", linewidth=3)  
plt.xlabel('Year')  
plt.ylabel('Number of futurestud.io Tutorials')   
plt.savefig("__ROBEXT_OUTPUT_PDF__")  
\end{CacheMeCode}
\end{codeexample}

You can also include files from the root folder, if you want, recompile all pictures if this files changes (nice to ensure maximum purity):
\begin{codeexample}[width=0pt,vbox]
\begin{CacheMe}{latex,
    add dependencies={common_inputs.tex},
    add to preamble={\input{__ROBEXT_WAY_BACK__common_inputs.tex}}}
  The answer is \myValueDefinedInCommonInputs.
\end{CacheMe}
\end{codeexample}

Since v2.1, we also provide easy ways to export counters:
\begin{codeexample}[width=0pt]
\begin{tikzpictureC}<forward counter=page>
  \node[rounded corners, fill=red]{The current page is \thepage.};
\end{tikzpictureC}    
\end{codeexample}
or even macros (we can also export evaluated macros):
\begin{codeexample}[width=0pt]
\cacheTikz
\newcommand{\MyNodeWithNewCommand}[2][draw,thick]{\node[rounded corners,fill=red,#1]{#2};}
\begin{tikzpicture}<forward=\MyNodeWithNewCommand>
  \MyNodeWithNewCommand{Alice}
  \MyNodeWithNewCommand[xshift=2cm]{Bob}
\end{tikzpicture}
\end{codeexample}

Since manually forwarding elements can be quite verbose, it is also possible to automatically apply some styles depending on the content of the element to cache. For instance, we provide a way to automatically forward some macros:
\begin{codeexample}[vbox]
\cacheTikz  
\robExtConfigure{add to preset={tikz}{auto forward}}

\newcommandAutoForward{\MyNode}[2][draw,thick]{\node[rounded corners,fill=red,#1]{#2};}
\newcommandAutoForward{\MyGreenNode}[2][draw,thick]{\node[rounded corners,fill=green,#1]{#2};}

\begin{tikzpicture}
  \MyNode{Recompiled only if MyNode is changed}
  \MyNode[xshift=8cm]{but not if the (unused) MyGreenNode is changed.}
\end{tikzpicture}\\

\begin{tikzpicture}
  \MyGreenNode{Recompiled only if MyGreenNode is changed}
  \MyGreenNode[xshift=8cm]{but not if the (unused) MyNode is changed.}
\end{tikzpicture}
\end{codeexample}

We also provide a number of other functionalities, among others:
\begin{itemize}
\item To achieve a pleasant and configurable interface (e.g. pass options to a preset with default values), we introduced placeholders, that can be of independent interest.
\item To get maximum purity while minimizing the number of rebuild, we provide functions to forward macros defined in the parent document, and the set of macros that must be forwarded can be automatically detected for each element to cache. It is also possible to conditionally import a library if the element contains a given macro.
\item We provide a way to compile a template via:\\
  |new compiled preset={compiled ZX}{templateZX, compile latex template}{}|\\
  in order to make the compilation even faster (but you cannot use presets except |add to preamble|)
\item We give a highly configurable template, compilation and inclusion system, to cache basically anything
\item We show how to compile all images in parallel using |\robExtConfigure{compile in parallel}| or |\robExtConfigure{compile in parallel after=5}| (make sure to have xargs installed, which is available by default on linux but must be installed on Windows via by installing the lightweight GNU On Windows (Gow) \url{https://github.com/bmatzelle/gow} if you use the default command) even during the first run, so that the first run becomes faster to compile than a normal run(!)
\item We furnish commands to automatically clean the cache while keeping cached elements that are still needed.
\item And more\dots
\end{itemize}


\section{Introduction}

\subsection{Why do I need to cache (a.k.a. externalize) parts of my document?}

One often wants to cache (i.e.\ store pre-compiled parts of the document, like figures) operations that are long to do: For instance, TikZ is great, but TikZ figures often take time to compile (it can easily take a few seconds per picture). This can become really annoying with documents containing many pictures, as the compilation can take multiple minutes: for instance my thesis needed roughly 30mn to compile as it contains many tiny figures, and LaTeX needs to compile the document multiple times before converging to the final result. But even on much smaller documents you can easily reach a few minutes of compilation, which is not only high to get a useful feedback in real time, but worse, when using online \LaTeX{} providers (e.g. overleaf), this can be a real pain as you are unable to process your document due to timeouts.

Similarly, you might want to cache the result of some codes, for instance a text or an image generated via python and matplotlib, without manually compiling them externally.

\subsection{Why not using \tikzname{}'s externalize library?}

\tikzname{} has an externalize library to pre-compile these images on the first run. Even if this library is quite simple to use, it has multiple issues:
\begin{itemize}
\item If you add a picture before existing pre-compiled pictures, the pictures that are placed after will be recompiled from scratch. This can be mitigated by manually adding a different prefix to each picture, but it is highly not practical to use.
\item To compile each picture, TikZ's externalize library reads the document's preamble and needs to process (quickly) the whole document. In large documents (or in documents relying on many packages), this can result in a significant loading time, sometimes much bigger than the time to compile the document without the externalize library: for instance, if the document takes 10 seconds to be processed, and if you have 200 pictures that take 1s each to be compiled, the first compilation with the TikZ's externalize library will take roughly half an hour (200 $\times$ 10s = 33mn) instead of 3mn without the library. And if you add a single picture at the beginning of the document… you need to restart everything from scratch. For these reasons, I was not even able to compile my thesis with TikZ's external library in a reasonable time.
\item  If two pictures share the same code, it will be compiled twice
\item Little purity is enforced: if a macro changes before a pre-compiled picture that uses this macro, the figure will not be updated. This can result in different documents depending on whether the cache is cleared or not.
\item As far as I know, it is made for TikZ picture mostly, and is not really made for inserting other stuff, like matplotlib images generated from python etc...
\item According to some maintainers of TikZ, ``\mylink{https://github.com/pgf-tikz/pgf/issues/758}{the code of the externalization library is mostly unreadable gibberish}'', and therefore most of the above issues are unlikely to be solved in a foreseable future.
\end{itemize}

\subsection{FAQ}

\paragraph{What is supported?}

You can cache most things, including tikz pictures, (including ones with overlays (but not with remember picture), with depth etc.), python code etc. We tried to make the library as customizable as possible to be useful in most scenarios. You can also send some data (e.g.\ : the current page) to the compiled pictures, and feed some data back to the main document (say that you want to compute a value that takes time to compute, or compute the number of pages of the produced document in order to increase the number of pages accordingly\dots{}). Since v2.0, you can also cache automatically any environment.

\paragraph{What is not supported?}

We do not yet support remember picture, and you can't use (yet) cross-references inside your images (at least not without further hacks) or links as links are stripped when the pdf is included. I might have some ideas to solve this\footnote{\url{https://tex.stackexchange.com/questions/695277/clickable-includegraphics-for-cross-reference-data}}, meanwhile you can just disable locally the library on problematic figures. Note that this library is quite young, so expect untested things.

\paragraph{What OS are supported?}

I tested the library on Linux and Windows, but the library should work on all OS, including MacOs. Please let me know if it fails for you.

\paragraph{Do I need to compile using -shell-escape?}

Since we need to compile the images via an external command, the simpler option is to add the argument |-shell-escape| to let the library run the  compilation command automatically (this is also the case of \tikzname's externalize library). However, people worried by security issues of |-shell-escape| (that allows arbitrary code execution if you don't trust the \LaTeX{} code) might be interested by these facts:
\begin{itemize}
\item You can choose to display a dummy content explaining how to manually compile the pictures until they are compiled, see |enable fallback to manual mode|.
\item You can compile manually the images: all the commands that are left to be executed are also listed in \texttt{robExt-compile-missing-figures.sh} and you can just inspect and run them, either with \texttt{bash robExt-compile-missing-figures.sh} or by typing them manually (most of the time it's only a matter of running \texttt{pdflatex somefile.tex}).
\item If you allowed:
  \begin{itemize}
  \item |pdflatex| (needed to compile latex documents)
  \item |cd| (not needed when using |no cache folder|)
  \item and |mkdir| (not needed when using |no cache folder| or if the cache folder that defaults to |robustExternalize| is already created) 
  \end{itemize}
  to run in restricted mode (so without enabling |-shell-escape|), then there is no need for |-shell-escape|. In that case, set |force compilation| and this library will compile even if |-shell-escape| is disabled.
\item If images are all already cached, you don't need to enable \texttt{-shell-escape} (this might be interesting e.g. to send a pre-cached document to the arxiv or to a publisher: just make sure to include the cache folder).
\item We use very few commands to compile latex files, basically only pdflatex, mkdir (to create the cache folder if needed) and cd (if the cache folder is not present). You might want to allow them in restricted mode.
\end{itemize}


\paragraph{Is it working on overleaf?}

Yes: overleaf automatically compiles documents with |-shell-escape|, so nothing special needs to be done there (of course, if you use this library to run some code, the programming language might not be available, but I heard that python is installed on overleaf servers for instance, even if this needs to be doubled checked). If the first compilation of the document to cache images times out, you can just repeat this operation multiple times until all images are cached.

\paragraph{Do you have some benchmarks?}

We completely rewrote the \href{https://github.com/leo-colisson/robust-externalize/commit/09a05a86ec9c04eb6687b75831ebb84185786ac5}{original library} to introduce the notion of placeholders (before, the template was fixed forever), allowing for greater flexibility. The original rewrite (1.0) had really poor performance compared to the original library (only x3 improvement compared to no externalization, while the old library could be 20x faster), but we made some changes (v2.0) that correct this issue. Here is a benchmark we made on an article with 159 small pictures (ZX diagrams from the zx-calculus library):
\begin{itemize}
\item without externalization: 1mn25
\item with externalization, v2.0 (first run, not in parallel): 3mn05, so takes twice the time; but (first run in parallel) takes 60s (so faster by a factor of 1.5).
\item with externalization, v2.0 (next runs): 5.77s (it is 15x times faster)
\item with externalization, v2.0, with a ``compiled'' template (less flexible) like in the old version: 4.0s (22x faster)
\item with externalization, old library (next runs): 4.16s (so 21x faster)
\end{itemize}

\paragraph{Can I compile pictures in parallel to speed up notably the first run?}

The first run might be particularly long to compile since many elements are still not cached and re-starting \LaTeX{} on each picture can take quite some time. If you have v2.1 or above, and if you have xargs installed (installed by default on most linux, on Windows you need to install the lightweight GNU On Windows (Gow) \url{https://github.com/bmatzelle/gow}), you can simply compile all images in parallel by typing at the beginning of the file:
\begin{codeexample}[code only]
\robExtConfigure{
  compile in parallel
}
\end{codeexample}
In my benchmark, I observed a compilation 1.5x faster than a normal compilation without any caching involved. We provide more options in \cref{sec:compileParallel}.

\paragraph{Can I use version-control to keep the cached files in my repository?} Sure, each cached figure is stored in a few files (typically one pdf and one \LaTeX{} file, plus the source) having the same prefix (the hash), avoiding collision between runs. Just commit these files and you are good to go.

\paragraph{Can I deal with baseline position and overlays ?} Yes, the depth of the box is automatically computed and used to include the figure by default, and additional margin is added to the image (30cm by default) to allow overlay text.

\paragraph{How is purity enforced?} Purity is the property that if you remove the cached files and recompile your document, you should end-up with the same output. To enforce purity, we compute the hash of the final program, including the compilation command, the dependency files used for instance in |\input{include.tex}| (unless you prefer not to, for instance to keep parts of the process impure for efficiency reasons), the forwarded macros, and put the code in a file named based on this hash. Then we compile it if it has not been used before, and include the output. Changing a single character in the file, the tracked dependencies, or the compilation command will lead to a new hash, and therefore to a new generated picture.

\paragraph{What if I don't want purity for all files?} If you do not want your files to be recompiled if you modify a given file, then just do not add this file to the list of dependencies.

\paragraph{Can I extend it easily?} We tried to take a quite modular approach in order to allow easy extensions. Internally, to support a new cache scheme, we only expect a string containing the program (possibly produced using a template), a list of dependencies, a command to compile this program (e.g. producing a pdf and possibly a tex file with the properties (depth…) of the pdf), and a command to load the result of the compilation into the final document (called after loading the previously mentioned optional tex file). Thanks to pgfkeys, it is then possible to create simple pre-made settings to automatically apply when needed.

\paragraph{How does it compare with \url{https://github.com/sasozivanovic/memoize}?} I recently became aware of the great \url{https://github.com/sasozivanovic/memoize}. While we aim to solve a similar goal, our approaches are quite different. While I focus on purity, and therefore create a different file for each picture, the above project puts all pictures in a single file and compile them all at once to avoid losing time to run the latex command for each picture (this mostly makes a difference for the first compilation). Our understanding of the main differences is the following:

Pros of \url{https://github.com/sasozivanovic/memoize}:
\begin{itemize}
\item The above library might be easier to setup as they do not need to specify the preamble of the file to compile (the preamble of the main file is used), and tikz pictures are automatically memoized (we provide |\cacheTikz| for that, but you still need to specify the preamble once).
\item This library allows to compile multiple pictures while loading \LaTeX{} only once, saving the time to start \LaTeX{} for each picture. However, it is not clear that it brings a faster compilation time compared to our library for multiple reasons:
  \begin{itemize}
  \item While during the first run all pictures are compiled with a single \LaTeX{} instance\footnote{We could also do something similar using some scripting to merge multiple cached files together, but this is not yet implemented.} (and inserted at the beginning of the document), they still need to extract each picture in a separate pdf file to include it at the right place in the document. This is done by default using |pdflatex|, which means that you need to call |pdflatex| once per picture, mitigating the advantage of compiling all pictures at once. Note that they also provide a python script to separate the pictures more efficiently, but this needs further manual interventions.
  \item We provide an option |compile in parallel| to compile all images in parallel, and the time is in my benchmark 1.5x faster than a normal compilation without any caching. Using this option, we should therefore compile faster than |memoized| that is not able to run jobs in parallel.
  \end{itemize}
\item Even if it does not officially support contexts, context might be automatically transmitted (while we need to do a bit of work to manually or semi-automatically define the context needed to compile a picture). However, this can break purity (see below).
\end{itemize}

Cons of \url{https://github.com/sasozivanovic/memoize}:
\begin{itemize}
\item The purity is not enforced so strongly since all images are in the same file. Notably, the hash only depends on the picture, but not on its context. So for instance if you define |\def\mycolor{blue}| before the picture, and use |\mycolor| inside the picture, if you change later the color to, say, |\def\mycolor{red}|, the picture will not be recompiled (so cleaning the cache and recompiling would produce a different result). In our case, the purity is always strictly enforced (unless you choose not to) since the context must be passed (explicitly or semi-automatically) to the compiled file.
\item As a result, the above library has poor support for contexts (in our case, you can easily, for instance, make a picture depend on the current page, and recompile the picture only if the current page changes: you can also do it the other way, and change some counters, say, depending on the cached file).
\item The above library only focuses on \LaTeX{} while our library works for any language.
\item The above library can only cache in pdf format, while we can generate any format (text, tex, jpg\dots{} and even videos that I include in my beamer presentations\dots{} but it is another topic!).
\item We have an arguably more complete documentation.
\end{itemize}

Note that |remember picture| is not supported in both libraries, but we can provide multiple methods to disable externalizations when |remember picture| is used.

\section{Tutorial}

\subsection{Installation}

To install the library, just copy the |robust-externalize.sty| file into the root of the project. Then, load the library using:\\

\begin{verbatim}
\usepackage{robust-externalize}
\end{verbatim}

If you want to display a message in the pdf on how to manually compile the file if |-shell-escape| is disabled, you can also load this configuration (only available on v2.0):

\begin{verbatim}
\robExtConfigure{
  enable fallback to manual mode,
}
\end{verbatim}

\noindent (otherwise, it will give you an error if |-shell-escape| is disabled and if some pictures are not yet cached)

If you forget/do not want to enable |-shell-escape|, this will give you this kind of message:

\noindent {
  \robExtConfigure{
    enable manual mode
  }
  \begin{tikzpictureC}[baseline=(A.base)]
    \node[fill=red, rounded corners](A){I must be manually compiled.};
    \node[fill=red, rounded corners, opacity=.3,overlay] at (A.north east){I am an overlay text};
  \end{tikzpictureC}
}


\subsection{Caching a tikz picture}

If you only care about \tikzname's picture, you have 3 options:
\begin{enumerate}
\item Call once |\cacheTikz| that will redefine |tikzpicture| and |\tikz| to use our library (if you use this solution, make sure to read how to disable externalization (\cref{sec:disableExternalization}) as we do not support for instance |remember picture|). Then, configure the default preamble for cached files as explained below by extending the |tikz| or |tikzpicture| presets (that first loads |tikz|).
\item Use the generic commands that allows you to wrap an arbitrary environment:
\begin{verbatim}
%  name environment ---v            v----- preset options
\cacheEnvironment{tikzpicture}{tikzpicture}
\end{verbatim}
  Note that you can also wrap commands (that you call with |\foo| instead of |\begin{foo}...\end{foo}|) using this:\\
  |\cacheCommand{yourcommand}[O{default val}m]{your preset options}|\\
  (|O{default val}m| means that the command accepts one optional argument with default value |default val| and one mandatory argument) but \textbf{we do recommend} to use |\cacheCommand| to cache |\tikz| since |\tikz| has a quite complicated parsing strategy (e.g.\ you can write |\tikz[options] \node{foo};| which has no mandatory argument enclosed in |{}|). |\cacheTikz| takes care of this already and caches both the environment |\begin{tikzpicture}| and the macro |\tikz|.
\item Use |tikzpictureC| instead of |tikzpicture| (this is mostly done to easily convert existing code to this library, but works only for |tikz| pictures).
\item Use the more general |CacheMe| environment, that can cache \tikzname, \LaTeX{}, python, and much more.
\end{enumerate}

These 4 options are illustrated below (note that all commands accept a first optional argument enclosed in |<...>| that contains the options to pass to |CacheMe| after loading the |tikzpicture| preset, that loads itself the |tikz| preset first):

Option 1:
\begin{codeexample}[width=0pt]
%% We override the default tikzpicture environment
%% to externalize all pictures
%% Warning: it will cause troubles with pictures relying on |remember pictures|
\cacheTikz

I am a cached picture: \begin{tikzpicture}[baseline=(A.base)]
  \node[draw,rounded corners,fill=pink!60](A){Hello World!};
\end{tikzpicture}. %

I am a non-cached picture: \begin{tikzpicture}<disable externalization>[baseline=(A.base)]
  \node[draw,rounded corners,fill=pink!60](A){Hello World!};
\end{tikzpicture}.
\end{codeexample}

Option 2:
\begin{codeexample}[width=0pt]
\cacheEnvironment{tikzpicture}{tikzpicture}
I am a cached picture: \begin{tikzpicture}[baseline=(A.base)]
  \node[draw,rounded corners,fill=pink!60](A){Hello World!};
\end{tikzpicture}.
\end{codeexample}

Option 3:
\begin{codeexample}[width=0pt]
I am a cached picture: \begin{tikzpictureC}[baseline=(A.base)]
  \node[draw,rounded corners,fill=pink!60](A){Hello World!};
\end{tikzpictureC}.
\end{codeexample}


Option 4:
\begin{codeexample}[width=0pt]
I am a cached picture: \begin{CacheMe}{tikzpicture}[baseline=(A.base)]
  \node[draw,rounded corners,fill=pink!60](A){Hello World!};
\end{CacheMe}.
\end{codeexample}

Since |CacheMe| is more general as it applies also to non-tikz pictures (just replace |tikzpicture| with the style of your choice), we will mostly use this syntax from now.

\subsection{Custom preamble}

Note that the pictures are compiled in a separate document, with a different preamble and class (we use the standalone class). This is interesting to reduce the compilation time of each picture (loading a large preamble is really time consuming) and to avoid unnecessary recompilation (do you want to recompile all your pictures when you add a single new macro?) without sacrificing the purity. But of course, you need to provide the preamble of the pictures. The easiest way is probably to modify the |tikz| preset (since |tikzpicture| loads this style first, it will also apply to tikzpictures. Note that you can also modify the |latex| preset if you want the change to apply to all \LaTeX{} documents):

\begin{codeexample}[width=0pt,vbox]
  \robExtConfigure{
    add to preset={tikz}{
      add to preamble={\usetikzlibrary{shadows}},
    },
  }

  See, tikz's style now packs the |shadows| library by default: %
  \begin{CacheMe}{tikzpicture}[even odd rule]
    \filldraw [drop shadow,fill=white] (0,0) circle (.5) (0.5,0) circle (.5);
  \end{CacheMe}
\end{codeexample}

and you can also create a new preset, and why not mix multiple presets together:
\begin{codeexample}[width=0pt,vbox]
  \robExtConfigure{
    new preset={load forest}{
      latex,
      add to preamble={\usepackage{forest}},
    },
    new preset={load hello}{
      add to preamble={\def\hello#1{Hello #1.}},
    },
  }

  \begin{CacheMe}{load forest, load hello}
    \begin{forest}
      [VP
        [DP[\hello{John}]]
        [V’
          [V[sent]]
          [DP[Mary]]
          [DP[D[a]][NP[letter]]]
        ]
      ]
    \end{forest}
  \end{CacheMe}
\end{codeexample}


\textbf{Note:} the |add to preset| and |new preset| directives have been added in v2.0, together with the |tikzpicture| preset. In v1.0, you would do |tikz/.append style={...}| (and you can still do this if you prefer), the difference is that |.append style| and |.style| require the user to double all hashes like |\def\mymacro##1{Hello ##1.}| which can lead to confusing errors.

You can also choose to overwrite the preset options for a single picture (or even a block of picture if you run the |\robExtConfigure| and |CacheMe| inside a group |{ ... }|):
\begin{codeexample}[width=0pt,vbox]
  See, you can add something to the preamble of a single picture: %
  \begin{CacheMe}{tikzpicture, add to preamble={\usetikzlibrary{shadows}}}[even odd rule]
    \filldraw [drop shadow,fill=white] (0,0) circle (.5) (0.5,0) circle (.5);
  \end{CacheMe}
\end{codeexample}

Note that if you use the |tikzpictureC| or |tikzpicture| syntax, you want to add the options right after the name of the command or environment, enclosed in |<>| (by default):
\begin{codeexample}[width=0pt,vbox]
  See, you can add something to the preamble of a single picture: %
  \begin{tikzpictureC}<add to preamble={\usetikzlibrary{shadows}}>[even odd rule]
    \filldraw [drop shadow,fill=white] (0,0) circle (.5) (0.5,0) circle (.5);
  \end{tikzpictureC}
\end{codeexample}

\textbf{Important}: Note that the preset options can be specified in a number of places: in |\robExtConfigure| (possibly in a group, or inside a preset), in the options of the picture, in the default options of |cacheEnvironment| and |cacheCommand| etc. Most options can be used in all these places, the only difference being the scope of the options.

\subsection{Dependencies}\label{sec:dependencies}

\textbf{Note: dependencies can sometimes be advantageously replaced with auto-forwarding of arguments, cf.\ \cref{sec:forwardIntro}.}

It might be handy to have a file that is loaded in both the main document and in the cached pictures. For instance, if you have a file |common_inputs.tex| that you want to input in both the main file and in the cached files, that contains, say:\\
|\def\myValueDefinedInCommonInputs{42}|\\
then you can add it as a dependency this way (here we use the |latex| preset that does not wrap the code inside a |tikzpicture| only to illustrate that we can also cache things that are not generated by tikz):
\begin{codeexample}[width=0pt,vbox]
\begin{CacheMe}{latex,
    add dependencies={common_inputs.tex},
    add to preamble={\input{__ROBEXT_WAY_BACK__common_inputs.tex}}}
  The answer is \myValueDefinedInCommonInputs.
\end{CacheMe}
\end{codeexample}
Note that the placeholder |__ROBEXT_WAY_BACK__| contains the path from the cache folder (containing the |.tex| that will be cached) to the root folder, and will be replaced when creating the file. This way, you can easily input files contained in the root folder. You can also create your own placeholders, read more below.

You can note that we used |add dependencies={common_inputs.tex}|: this allows us to recompile the files if |common_inputs.tex| changes. If you do not want this behavior (e.g. |common_inputs.tex| changes too often and you do not want to recompile everything at every change), you can remove this line, but beware: if you do a breaking changes in |common_inputs.tex| (e.g.\ redefine |42| to |43|), then the previously cached picture will not be recompiled! (So you will still read 42 instead of 43.)

\subsection{Wrap arbitrary environments}

You can wrap arbitrary environments using the already presented |cacheEnvironment| and |cacheCommand|, where the first mandatory argument is the name of the environment/macro, and the second mandatory argument contains the default preset. |cacheEnvironment| works for any environment while |cacheCommand| might need a bit of help to determine the signature of the macro if the function is defined via xparse. Long story short, |O{foo}| is an optional argument with default value foo, |m| is a mandatory argument. By default, you can pass an optional arguments via the first argument enclosed in |<>|:
\begin{codeexample}[width=0pt,vbox]
  \cacheCommand{zx}[O{}O{}O{}m]{latex, add to preamble={\usepackage{zx-calculus}\def\hello#1{Hello #1.}}}
  \zx<add to preamble={\usepackage{amsmath}\def\bye#1{Bye #1.}}>[mbr=1]{ % amsmath provides \text
    \zxX{\alpha} \rar \ar[d,C] & \zxZ{\text{\hello{Alice}}} \\
    \zxZ{\beta} \rar           & \zxZ{\text{\bye{Bob}}}
  }
\end{codeexample}

you can also disable externalization for some commands using this same command, here is for instance the code to use |todonotes|:
\begin{codeexample}[code only]
\cacheCommand{todo}[O{}m]{disable externalization}
\todo{Check how to use cacheCommand and cacheEnvironment}  
\end{codeexample}

Which gives you {
  \cacheCommand{todo}[O{}m]{disable externalization}
  \todo{Check how to use cacheCommand and cacheEnvironment}
}


See \cref{sec:wrapAutomatically} for more details.

\subsection{Disabling externalization}

You can use |disable externalization| to disable externalization (which is particularly practical if you set |\cacheTikz|). You can configure the exact command run in that case using |command if no externalization/.code={...}|, but most of the time it should work out of the box (see \cref{sec:disableExternalization} for details).

\begin{codeexample}[width=0pt,vbox]
  % In theory all pictures should be externalized (so remember picture should fail)
  \tikz[remember picture,baseline=(pointtome1.base)]
    \node[rounded corners, fill=orange](pointtome1){Point to me if you can};\\
  \cacheTikz
  % But we can disable it temporarily
  \begin{tikzpicture}<disable externalization>[remember picture]
    \node[rounded corners, fill=red](A){This figure is not externalized.
      This way, it can use remember picture.};
    \draw[->,overlay] (A) to[bend right] (pointtome1);
  \end{tikzpicture}\\

  % You can also disable it globally/in a group:
  {
    \robExtConfigure{disable externalization}
    
    \begin{tikzpicture}[remember picture]
      \node[rounded corners, fill=red](A){This figure is not externalized.
        This way, it can use remember picture.};
      \draw[->,overlay] (A.west) to[bend left] (pointtome1);
    \end{tikzpicture}\\

    \begin{tikzpicture}[remember picture]
      \node[rounded corners, fill=red](A){This figure is not externalized.
        This way, it can use remember picture.};
      \draw[->,overlay] (A.east) to[bend right] (pointtome1);
    \end{tikzpicture}\\
  }
  
  \begin{tikzpicture}
    \node[rounded corners, fill=green](A){This figure is externalized, but cannot use remember picture.};
  \end{tikzpicture}
\end{codeexample}

You can also disable externalization for some kinds of commands. For instance, the package |todonotes| requires |remember picture| and is therefore not compatible with externalization provided by this package. To disable externalization on all |\todo|, you can do:

\begin{codeexample}[code only]
\cacheCommand{todo}[O{}m]{disable externalization}
\todo{Check how to use cacheCommand and cacheEnvironment}  
\end{codeexample}

Which gives you {
  \cacheCommand{todo}[O{}m]{disable externalization}
  \todo{Check how to use cacheCommand and cacheEnvironment}
}

You can also disable externalization on elements that contain a specific string/regex. For instance, you can disable externalization on all elements containing |remember picture|:
\begin{codeexample}[vbox]
\cacheTikz
\robExtConfigure{
  add to preset={tikz}{
    if matches={remember picture}{disable externalization},
  },
}
\begin{tikzpicture}[remember picture]
  \node[fill=green](my node){Point to me};
\end{tikzpicture} %
Some text %
\begin{tikzpicture}[overlay, remember picture]
  \draw[->] (0,0) to[bend left] (my node);
\end{tikzpicture}
\end{codeexample}


\subsection{Feeding data from the main document to the cached documents}\label{sec:forwardIntro}

You can feed data from the main document to the cached file using placeholders, since |set placeholder eval={__foo__}{\bar}| will evaluate |\bar| and put the result in |__foo__|. For instance, if the picture depends on the current page, you can do:

\begin{codeexample}[width=0pt,vbox]
\begin{tikzpictureC}<set placeholder eval={__thepage__}{\thepage}>
  \node[rounded corners, fill=red]{The current page is __thepage__.};
\end{tikzpictureC}
\end{codeexample}

However, since v2.1, we also provide a method to specifically export a counter using |forward counter=my counter| and a macro using |forward=\macroToExport|, possibly by evaluating first using |forward eval=\macroToEval|:
\begin{codeexample}[width=0pt]
\cacheTikz
\NewDocumentCommand{\MyNode}{O{}m}{\node[rounded corners,fill=red,#1]{#2};}
\begin{tikzpicture}<forward=\MyNode>
  \MyNode{Alice}
  \MyNode[xshift=2cm]{Bob}
\end{tikzpicture}
\end{codeexample}

To avoid manually writing |forward=...| for each picture, we can instead load |auto forward| and define the macro for instance using |\newcommandAutoForward| (we also provide alternatives for xparse and def-based definitions):
\begin{codeexample}[vbox]
\cacheTikz  
\robExtConfigure{add to preset={tikz}{auto forward}}

\newcommandAutoForward{\MyNode}[2][draw,thick]{\node[rounded corners,fill=red,#1]{#2};}
\newcommandAutoForward{\MyGreenNode}[2][draw,thick]{\node[rounded corners,fill=green,#1]{#2};}

\begin{tikzpicture}
  \MyNode{Recompiled only if MyNode is changed}
  \MyNode[xshift=8cm]{but not if the (unused) MyGreenNode is changed.}
\end{tikzpicture}\\

\begin{tikzpicture}
  \MyGreenNode{Recompiled only if MyGreenNode is changed}
  \MyGreenNode[xshift=8cm]{but not if the (unused) MyNode is changed.}
\end{tikzpicture}
\end{codeexample}

If a macro depends on another macro/package, it is possible to load any additional style like using the last optional argument of |\newcommandAutoForward|:
\begin{codeexample}[vbox]
\cacheTikz  
\robExtConfigure{add to preset={tikz}{auto forward}}

\newcommandAutoForward{\MyName}{Alice}
\newcommandAutoForward{\MyNode}[2][draw,thick]{
  \node[rounded corners,fill=red,#1]{\ding{164} \MyName: #2};
}[forward=\MyName, add to preamble={\usepackage{pifont}}]

\begin{tikzpicture}
  \MyNode{Recompiled if MyNode or MyName are changed}
\end{tikzpicture}
\end{codeexample}

If instead of forwarding the macro you want to, say, load a package or any other style, you should use instead |\robExtConfigIfMacroPresent|:
\begin{codeexample}[vbox]
\cacheTikz  
\robExtConfigure{add to preset={tikz}{auto forward}}

\robExtConfigIfMacroPresent{\ding}{add to preamble={\usepackage{pifont}}}

\begin{tikzpicture}
  \node[fill=green, circle]{\ding{164}};
\end{tikzpicture}
\end{codeexample}

For more details and functions, see \cref{sec:forward}.

\subsection{Defining macros}

You can define macros in the preamble:
\begin{codeexample}[width=0pt,vbox]
\cacheMe[latex, add to preamble={\def\sayhello#1{Hello #1.}}]{
  \sayhello{my friend}
}
\end{codeexample}

\subsection{Feeding data back into the main document}

For more advanced usage, you might want to compute a data and cache the result in a macro that you could use later. This is possible if you write into the file |\jobname-out.tex| during the compilation of the cached file (by default, we already open |\writeRobExt| to write to this file). This file will be automatically loaded before loading the pdf (but you can customize all these operations, for instance if you do not want to load the pdf at all; the only requirement is that you should generate a |.pdf| file to specify that the compilation is finished).

For instance:

\begin{codeexample}[width=0pt,vbox]
\begin{CacheMe}{latex, add to preamble={\usepackage{tikz}}, do not include pdf}
We compute this data that is long to compute:
\pgfmathparse{(1 + sqrt(5))/2}% result is stored in \pgfmathresult
% We write the result to the -out file (\string\foo writes \foo to the file without evaluating it,
% so this will write "\gdef\myLongResult{1.61803}"):
% Note that CacheMe is evaluated in a group, so you want to use \gdef to define it
% outside of the group
\immediate\write\writeRobExt{%
  \string\gdef\string\myLongResult{\pgfmathresult}%
}
\end{CacheMe}

We computed the cached value \myLongResult.
\end{codeexample}

\subsection{Compile in parallel}
Make sure to have xargs installed (installed by default on most linux, on Windows you need to install the lightweight GNU On Windows (Gow) \url{https://github.com/bmatzelle/gow}), and type in your preamble:
\begin{codeexample}[code only]
\robExtConfigure{
  compile in parallel
}
\end{codeexample}
to compile all figures in parallel (you need to compile your document twice). Your document should build significantly faster, possibly faster than a normal run. If you want to compile in parallel
only if you have more than, say, 5 new elements to cache, do:
\begin{codeexample}[code only]
\robExtConfigure{
  compile in parallel after=5
}
\end{codeexample}

\subsection{Compile a preset}\label{sec:compilePreset}

Long story short: you can compile even faster (around 1.5x in our tests) by compiling presets, but beware that you will not be able to modify the placeholders except |add to preamble| with the default compiler we provide:

\begin{codeexample}[vbox]
 % We create a latex-based preset and compile it
\robExtConfigure{
  new preset={templateZX}{
    latex,
    add to preamble={
      \usepackage{tikz}
      \usepackage{tikz-cd}
      \usepackage{zx-calculus}
    },
    %% possibly add some dependencies
  },
  % We compile it into a new preset
  new compiled preset={compiled ZX}{templateZX, compile latex template}{},
}

% we use that preset automatically for ZX environments
\cacheEnvironment{ZX}{compiled ZX}
\cacheCommand{zx}{compiled ZX}

% Usage: (you can't use placeholders except for the preamble, trade-off of the compiled template)
\begin{ZX}<add to preamble={\def\sayHey#1{Hey #1!}}>
  \zxX{\sayHey{Bob}}
\end{ZX}
\end{codeexample}

For details, see \cref{sec:compileFaster}.

\subsection{For non-\LaTeX{} code}

Due to the way \LaTeX{} works, non-\LaTeX{} code can't be reliably read inside macros and some environments that parse their body (e.g. align) as some characters are removed (e.g. percent symbols are comments and are removed). For this reason, we sometimes need to separate the time where we define the code and where we insert it (this is done using placeholders, see |PlaceholderFromCode|), and we need to introduce new environments to populate the template (see \cref{sec:placeholders} for more details, to generate them from filename, to get the path of a file etc).

The environment |CacheMeCode| can be used for this purpose.

\subsubsection{Python code}

\paragraph{Generate an image}

For instance, you can use the default |python| template to generate an image with python. The following code:


\begin{codeexample}[code only]
\begin{CacheMeCode}{python, set includegraphics options={width=.8\linewidth}}
import matplotlib.pyplot as plt
year = [2014, 2015, 2016, 2017, 2018, 2019]  
tutorial_count = [39, 117, 111, 110, 67, 29]
plt.plot(year, tutorial_count, color="#6c3376", linewidth=3)  
plt.xlabel('Year')  
plt.ylabel('Number of futurestud.io Tutorials')   
plt.savefig("__ROBEXT_OUTPUT_PDF__")  
\end{CacheMeCode}
\end{codeexample}

will produce the image visible in \cref{fig:pythonGeneratedImage}. \textbf{Importantly: you do not want to indent the content of CacheMeCode, or the space will also appear in the final code.}
\begin{figure}
\centering
\begin{CacheMeCode}{python, set includegraphics options={width=.8\linewidth}}
import matplotlib.pyplot as plt
year = [2014, 2015, 2016, 2017, 2018, 2019]  
tutorial_count = [39, 117, 111, 110, 67, 29]
plt.plot(year, tutorial_count, color="#6c3376", linewidth=3)  
plt.xlabel('Year')  
plt.ylabel('Number of futurestud.io Tutorials')   
plt.savefig("__ROBEXT_OUTPUT_PDF__")
\end{CacheMeCode}
\caption{Image generated with python.}
\label{fig:pythonGeneratedImage}
\end{figure}

\paragraph{Compute a value}

We also provide by default a number of helper functions. For instance, |write_to_out(text)| will write |text| to the |*-out.tex| file that is loaded automatically by \LaTeX{}. This is useful to compute data that is not an image (note that |r"some string"| does not consider backslash as an escape string, which is handy to write \LaTeX{} code in python):

For instance:
\begin{codeAndResult}
\begin{CacheMeCode}{python, do not include pdf}
import math
write_to_out(r"\gdef\cosComputedInPython{" + str(math.cos(1)) + r"}")
\end{CacheMeCode}

$\rightarrow$ The cosinus of 1 is \cosComputedInPython.
\end{codeAndResult}

\paragraph{Improve an existing preset}

If you often use the same code (e.g.\ load matplotlib, save the file etc), you can directly modify the |__ROBEXT_MAIN_CONTENT__| placeholder to add the redundant information (or create a new template from scratch, see below). By default, it points to |__ROBEXT_MAIN_CONTENT_ORIG__| that contains directly the code typed by the user (this is true for all presets, as |CacheMe*| is in charge of setting this placeholder). When dealing with \LaTeX{} code, |__ROBEXT_MAIN_CONTENT__| should ideally contain a code that can be inserted as-it in the document in order to be compatible by default with |disable externalization|. So if you want to wrap the content of the user in an environment like |\begin{tikzpicture}...\end{tikzpicture}|, this is the placeholder to modify.

\begin{codeAndResult}
%% Create your style:
\begin{PlaceholderFromCode}{__MY_MATPLOTLIB_TEMPLATE_BEFORE__}
import matplotlib.pyplot as plt
import sys
\end{PlaceholderFromCode}

\begin{PlaceholderFromCode}{__MY_MATPLOTLIB_TEMPLATE_AFTER__}
plt.savefig("__ROBEXT_OUTPUT_PDF__")  
\end{PlaceholderFromCode}

\robExtConfigure{
  new preset={my matplotlib}{
    python,
    add before placeholder no space={__ROBEXT_MAIN_CONTENT__}{__MY_MATPLOTLIB_TEMPLATE_BEFORE__},
    add to placeholder no space={__ROBEXT_MAIN_CONTENT__}{__MY_MATPLOTLIB_TEMPLATE_AFTER__},
  },
}

%% Use your style:
%% See, you don't need to load matplotlib or save the file:
\begin{CacheMeCode}{my matplotlib, set includegraphics options={width=.5\linewidth}}
year = [2014, 2015, 2016, 2017, 2018, 2019]  
tutorial_count = [39, 117, 111, 110, 67, 29]
plt.plot(year, tutorial_count, color="#6c3376", linewidth=3)  
plt.xlabel('Year')  
plt.ylabel('Number of futurestud.io Tutorials')     
\end{CacheMeCode}
\end{codeAndResult}

\paragraph{Custom parameters and placeholders}

Let us say that you would like to define a default font size for your figure, but that you would like to allow the user to change this font size. Then, you should create a new placeholder with your default value, and use |set placeholder| to change this value later (see also the documentation of |CacheMeCode| to see how to create a new command to avoid typing |set placeholder|):

\begin{codeAndResult}
%% Create your style:

\begin{PlaceholderFromCode}{__MY_MATPLOTLIB_TEMPLATE_BEFORE__}
import matplotlib as mpl
import matplotlib.pyplot as plt
import sys
mpl.rcParams['font.size'] = __MY_MATPLOTLIB_FONT_SIZE__
\end{PlaceholderFromCode}

\begin{PlaceholderFromCode}{__MY_MATPLOTLIB_TEMPLATE_AFTER__}
plt.savefig("__ROBEXT_OUTPUT_PDF__")  
\end{PlaceholderFromCode}

\robExtConfigure{
  new preset={my matplotlib}{
    python,
    % We create a new placeholder (it is simple enough that you don't need to use PlaceholderFromCode)
    set placeholder={__MY_MATPLOTLIB_FONT_SIZE__}{12},
    add before placeholder no space={__ROBEXT_MAIN_CONTENT__}{__MY_MATPLOTLIB_TEMPLATE_BEFORE__},
    add to placeholder no space={__ROBEXT_MAIN_CONTENT__}{__MY_MATPLOTLIB_TEMPLATE_AFTER__},
  },
}

%% Use your style:
%% See, you don't need to load matplotlib or save the file:
Default font size: \begin{CacheMeCode}{my matplotlib, set includegraphics options={width=.5\linewidth}}
year = [2014, 2015, 2016, 2017, 2018, 2019]  
tutorial_count = [39, 117, 111, 110, 67, 29]
plt.plot(year, tutorial_count, color="#6c3376", linewidth=3)  
plt.xlabel('Year')  
plt.ylabel('Number of futurestud.io Tutorials')     
\end{CacheMeCode}

With font size 16:
\begin{CacheMeCode}{my matplotlib,
    set includegraphics options={width=.5\linewidth},
    set placeholder={__MY_MATPLOTLIB_FONT_SIZE__}{16}}
year = [2014, 2015, 2016, 2017, 2018, 2019]  
tutorial_count = [39, 117, 111, 110, 67, 29]
plt.plot(year, tutorial_count, color="#6c3376", linewidth=3)  
plt.xlabel('Year')  
plt.ylabel('Number of futurestud.io Tutorials')     
\end{CacheMeCode}
\end{codeAndResult}

Note that if you manage to move all the code in the template and that the user can configure everything using the options and an empty content, you can use |CacheMeNoContent| that takes no argument and that consider its body as the options.

\paragraph{Custom include command}

There may be some cases where you do not want to include a picture. We already saw the option |do not include pdf| if you do not want to include anything. But you can customize the include function, using notably:\\
|custom include command={your include command}|

For instance, let us say that you would like to display both the source code used to obtain a given code, together with the output of this code. Then, you can write this style:

\begin{codeexample}[code only]
{
%% Create your style:
\begin{PlaceholderFromCode}{__MY_PRINT_BOTH_TEMPLATE_BEFORE__}
# File where print("bla") should be redirected
# get_filename_from_extension("-foo.txt") will give you the path of the file
# in the cache that looks like robExt-somehash-foo.txt
print_file = open(get_filename_from_extension("-print.txt"),  "w")
sys.stdout = print_file
# This code will read the current code, and extract the lines between
# that starts with "### CODESTARTSHERE" and "### CODESTOPSHERE", and will write
# it into the *-code.text (we do not want to print all these functions in
# the final code)
with open(get_filename_from_extension("-code.txt"), "w") as f:
    # The current script has extension .tex
    with open(get_current_script(), "r") as script:
        should_write = False
        for line in script:
            if line.startswith("### CODESTARTSHERE"):
                should_write = True
            elif line.startswith("### CODESTOPSHERE"):
                should_write = False
            elif "HIDEME" in line:
                pass
            else:
                if should_write:
                    f.write(line)
### CODESTARTSHERE
\end{PlaceholderFromCode}


\begin{PlaceholderFromCode}{__MY_PRINT_BOTH_TEMPLATE_AFTER__}
### CODESTOPSHERE
print_file.close()
\end{PlaceholderFromCode}

\robExtConfigure{
  new preset={my python print both}{
    python,
    add before placeholder no space={__ROBEXT_MAIN_CONTENT__}{__MY_PRINT_BOTH_TEMPLATE_BEFORE__},
    add to placeholder no space={__ROBEXT_MAIN_CONTENT__}{__MY_PRINT_BOTH_TEMPLATE_AFTER__},
    set title/.style={
      set placeholder={__MY_TITLE__}{#1},
    },
    set title={Example},
    custom include command={
      % Useful to replace __MY_TITLE__:
      \evalPlaceholder{
        \begin{tcolorbox}[colback=red!5!white,colframe=red!75!black,title=__MY_TITLE__]
          \lstinputlisting[frame=single,
            breakindent=.5\textwidth,
            frame=single,
            breaklines=true,
            style=mypython]{\robExtAddCachePathAndName{\robExtFinalHash-code.txt}}
          Output: 
          \verbatiminput{\robExtAddCachePathAndName{\robExtFinalHash-print.txt}}
        \end{tcolorbox}
      }
    },
  },
}
\end{codeexample}

Once the style is defined (actually we already defined in the library under the name |python print code and result|), you can just write:
\begin{codeexample}[code only]
\begin{CacheMeCode}{my python print both, set title={The for loop}}
for name in ["Alice", "Bob"]:
    print(f"Hello {name}")
\end{CacheMeCode}
\end{codeexample}
to get:
{
%% Create your style:
\begin{PlaceholderFromCode}{__MY_PRINT_BOTH_TEMPLATE_BEFORE__}
# File where print("bla") should be redirected
# get_filename_from_extension("-foo.txt") will give you the path of the file
# in the cache that looks like robExt-somehash-foo.txt
print_file = open(get_filename_from_extension("-print.txt"),  "w")
sys.stdout = print_file
# This code will read the current code, and extract the lines between
# that starts with "### CODESTARTSHERE" and "### CODESTOPSHERE", and will write
# it into the *-code.text (we do not want to print all these functions in
# the final code)
with open(get_filename_from_extension("-code.txt"), "w") as f:
    # The current script has extension .tex
    with open(get_current_script(), "r") as script:
        should_write = False
        for line in script:
            if line.startswith("### CODESTARTSHERE"):
                should_write = True
            elif line.startswith("### CODESTOPSHERE"):
                should_write = False
            elif "HIDEME" in line:
                pass
            else:
                if should_write:
                    f.write(line)
### CODESTARTSHERE
\end{PlaceholderFromCode}


\begin{PlaceholderFromCode}{__MY_PRINT_BOTH_TEMPLATE_AFTER__}
### CODESTOPSHERE
print_file.close()
\end{PlaceholderFromCode}

\robExtConfigure{
  new preset={my python print both}{
    python,
    add before placeholder no space={__ROBEXT_MAIN_CONTENT__}{__MY_PRINT_BOTH_TEMPLATE_BEFORE__},
    add to placeholder no space={__ROBEXT_MAIN_CONTENT__}{__MY_PRINT_BOTH_TEMPLATE_AFTER__},
    set title/.style={
      set placeholder={__MY_TITLE__}{#1},
    },
    set title={Example},
    custom include command={
      % Useful to replace __MY_TITLE__:
      \evalPlaceholder{
        \begin{tcolorbox}[colback=red!5!white,colframe=red!75!black,title=__MY_TITLE__]
          % \noindent Code:
          \lstinputlisting[frame=single,breakindent=.5\textwidth,frame=single,breaklines=true,style=mypython]{\robExtAddCachePathAndName{\robExtFinalHash-code.txt}}
          Output: 
          \verbatiminput{\robExtAddCachePathAndName{\robExtFinalHash-print.txt}}
        \end{tcolorbox}
      }
    },
  },
}

%% Use your style:
\begin{CacheMeCode}{my python print both, set title={The for loop}}
for name in ["Alice", "Bob"]:
    print(f"Hello {name}")
\end{CacheMeCode}
}

\subsubsection{Other languages}\label{sec:otherLanguages}

We also provide support for other languages, notably |bash|, but it is relatively easy to add basic support for any new language. You only need to configure |set compilation command| to your command, |set template| to the file to compile (|__ROBEXT_MAIN_CONTENT__| contains the code typed by the user), and possibly a custom include command with |custom include command| if you do not want to do |\includegraphics| on the final pdf. For instance, to define a basic template for bash, you just need to use:

\begin{codeAndResult}
% Create your style
\begin{PlaceholderFromCode}{__MY_BASH_TEMPLATE__}
# Quit if there is an error
set -e  
__ROBEXT_MAIN_CONTENT__
# Create the pdf file to certify that no compilation error occured
touch "__ROBEXT_OUTPUT_PDF__"
\end{PlaceholderFromCode}

\robExtConfigure{
  new preset={my bash}{
    set compilation command={bash "__ROBEXT_SOURCE_FILE__"},
    set template={__MY_BASH_TEMPLATE__},
    %%% Version 1:
    % verbatim output,
    %%% Version 2:
    custom include command={%
      \evalPlaceholder{%
        \verbatiminput{__ROBEXT_CACHE_FOLDER____ROBEXT_OUTPUT_PREFIX__-out.txt}%
      }%
    },
    % Ensure that the code does not break when externalization is disabled
    print verbatim if no externalization,
  }
}

% Use your style
\begin{CacheMeCode}{my bash}
# Write the system conf to a file *-out.txt
uname -srv > "__ROBEXT_OUTPUT_PREFIX__-out.txt"
\end{CacheMeCode}
\end{codeAndResult}


\paragraph{Code inside a macro}

Due to fundamental \LaTeX{} restrictions, it is impossible to use |CacheMeCode| inside a macro or some environments as \LaTeX{} will strip all lines containing a percent character for instance. The solution here is to define our main content before, and then set it using |set main content| (that simply sets |__ROBEXT_MAIN_CONTENT_ORIG__|). In this example, we also show how |CacheMeNoContent| can be used when their is no content (the arguments to |CacheMe| are directly given in the body of |CacheMeNoContent|):

\begin{codeAndResult}
\begin{PlaceholderFromCode}{__TMP_MAIN_CONTENT__}
import matplotlib.pyplot as plt
year = [2014, 2015, 2016, 2017, 2018, 2019]  
tutorial_count = [39, 117, 111, 110, 67, 29]
plt.plot(year, tutorial_count, color="#6c3376", linewidth=3)  
plt.xlabel('Year')  
plt.ylabel('Number of futurestud.io Tutorials')   
plt.savefig("__ROBEXT_OUTPUT_PDF__")  
\end{PlaceholderFromCode}

\fbox{\begin{CacheMeNoContent}
  python,
  set includegraphics options={width=.8\linewidth},
  set main content={__TMP_MAIN_CONTENT__},
\end{CacheMeNoContent}}
\end{codeAndResult}

\subsection{Force to recompile or remove a cached item}

If you want to recompile a file (e.g. an untracked dependency was changed\dots{} and you do not want to track it to avoid recompilation when you change a single line in this file), you can use the |recompile| style since v2.1:

\begin{codeexample}[vbox]
\cacheEnvironment{tikzcd}{tikz, add to preamble={\usepackage{tikz-cd}}}
\begin{tikzcd}<recompile>
  A \rar & B
\end{tikzcd}    
\end{codeexample}

Note that this assumes that your compilation command is idempotent (so running it twice is like running it once) since the aux files are not cached. If you want to clean aux files or if you run an older version, see the documentation of |recompile| in \cref{sec:configureCompilationCommand}.

\section{Documentation}

Before starting this documentation, note that all commands are prefixed with |robExt| and all environments are prefixed with |RobExt|, but we also often define aliases without this prefix. The user is free to use any version, but we recommend to use the non-prefixed version unless a clash with a package forbids you from using it. In the following, we will only print the non-prefixed name when it exists. Note also that we follow the convention that environment names start with an upper case letter while commands start with a lower case letter.

\subsection{How it works}

This library must be able to generate 3 elements for any cached content:
\begin{itemize}
\item a source file, that will be compiled, and is obtained by expanding the placeholder |__ROBEXT_TEMPLATE__| (see \cref{sec:placeholders}),
\item a compilation command obtained by expanding the placeholder |__ROBEXT_COMPILATION_COMMAND__|,
\item a dependency file, that contains the hash of all the dependencies (see \cref{sec:dependencies} for details) and the compilation command,
\item an inclusion command (this one is not used during the caching process, it is only used when including the compiled document in the main document), that you can set using |custom include command={your command}|.
\end{itemize}

The hash of all these elements is computed in order to obtain a reference hash, denoted |somehash| that looks like a unique random value (note that |__ROBEXT_OUTPUT_PDF__| and alike are expanded after knowing the hash since they depend on the final hash value). This hash |somehash| will change whenever a dependency changes, or if the compilation command changes, ensuring purity. Then, the dependency file and the source file are written in the cache, by default in |robustExternalize/robExt-somehash.tex| and |robustExternalize/robExt-somehash.deps|. Then, the compilation command will be run from the cache folder. At the end, by default, we check if a file |robustExternalize/robExt-somehash.pdf| exists: if not we abort, otherwise we |\input| the file |robustExternalize/robExt-somehash-out.tex| and we run the include command (that includes the pdf by default). As we saw earlier, this command can be customized to use other files. \textbf{Importantly, all the files created during the compilation must be prefixed by} |robExt-somehash|, which can be obtained at runtime using |__ROBEXT_OUTPUT_PREFIX__|. This way, we can easily clean the cache while ensuring maximum purity.

In the following, we will denote by |*-foo.bar| the file in:\\
|robustExternalize/robExt-somehash-foo.bar|.

Note also that we usually define two names for each function, one normal and one prefixed with |robExt| (or |RobExt|) for environments. In this documentation, we only write the first form, but the second form is kept in case a conflicting package redefines some functions.

\subsection{Placeholders}\label{sec:placeholders}

Placeholders are the main concept allowing this library to generate the content of a source file based on a template (a template will itself be a placeholder containing other placeholders). A placeholder is a special strings like |__COLOR_IMAGE__| that should start and end with two underscores, ideally containing no space or double underscores inside the name directly. Placeholders are inserted for instance in a string and will be given a value later. This value will be used to replace (recursively) the placeholder in the template. For instance, if a placeholder |__LIKES__| contains |I like __FRUIT__ and __VEGETABLE__|, if the placeholder |__FRUIT__| contains |oranges| and if the placeholder |__VEGETABLE__| contains |salad|, then evaluating |__LIKES__| will output |I like oranges and salad.|

Note that you are not strictly forced to follow the above convention (it allows us to optimize the code to find and replace placeholders), but in that case you should enable |not all placeholders have underscores|.

Placeholders are local variables (internally just some \LaTeX{} 3 strings). You can therefore define a placeholder in a local group surrounded by brackets |{ ... }| if you want it to have a reduced scope.

\subsubsection{Reading a placeholder}

\begin{pgfmanualentry}
  \extractcommand\getPlaceholder\opt{\oarg{new placeholder name}}\marg{name placeholder or string}\@@
  \extractcommand\getPlaceholderInResult\opt{\oarg{new placeholder name}}\marg{name placeholder or string}\@@
  \extractcommand\getPlaceholderInResultFromList\marg{list,of,placeholders,to,replace}\opt{\oarg{new placeholder name}}\marg{name placeholder or string}\@@
  \pgfmanualbody

  Get the value of a placeholder after replacing (recursively) all the inner placeholders. |\getPlaceholderInResult| puts the resulting string in a \LaTeX{} 3 string |\l_robExt_result_str| and in |\robExtResult|, while |\getPlaceholder| directly outputs this string. You can also put inside the argument
  any arbitrary string, allowing you, for instance, to concatenate multiple placeholders, copy a placeholder etc. Note that you will get a string, but this string will not be evaluated by \LaTeX{} (see |\evalPlaceholder| for that), for instance math will not be interpreted:
\begin{codeexample}[width=0pt,vbox]
  \setPlaceholder{__MY_PLACEHOLDER__}{Hello __NAME__, I am a template $\delta_n$.}
  \setPlaceholder{__NAME__}{Alice __NICKNAME__}
  \setPlaceholder{__NICKNAME__}{the great}
  The placeholder evaluates to:\\
  \texttt{\getPlaceholder{__MY_PLACEHOLDER__}}\\
  Combining placeholders produces:\\
  \texttt{\getPlaceholder{In ``__MY_PLACEHOLDER__'', the name is __NAME__.}}
\end{codeexample}
You can also specify the optional argument in order to additionally define a new placeholder containing the resulting string (but you might prefer to use its alias |\setPlaceholderRec| described below):
\begin{codeexample}[width=0pt,vbox]
  \setPlaceholder{__MY_PLACEHOLDER__}{Hello __NAME__, I am a template $\delta_n$.}
  \setPlaceholder{__NAME__}{Alice __NICKNAME__}
  \setPlaceholder{__NICKNAME__}{the great}
  \getPlaceholderInResult[__NEW_PLACEHOLDER__]{In ``__MY_PLACEHOLDER__'', the name is __NAME__.}
  \printImportedPlaceholdersExceptDefaults
\end{codeexample}
The variation |\getPlaceholderInResultFromList| allows you to specify a set of placeholder to replace from:
\begin{codeexample}[width=0pt,vbox]
\setPlaceholder{__MY_PLACEHOLDER__}{Hello __NAME1__, __NAME2__ and __NAME3__, I am a template $\delta_n$.}
\setPlaceholder{__NAME1__}{Alice}
\setPlaceholder{__NAME2__}{Bob}
\setPlaceholder{__NAME3__}{Charlie}
\getPlaceholderInResultReplaceFromList{__MY_PLACEHOLDER__,__NAME1__,__NAME2__}[__NEW_PLACEHOLDER__]{
  Here we go: __MY_PLACEHOLDER__
}
\printImportedPlaceholdersExceptDefaults
\end{codeexample}
\end{pgfmanualentry}


\begin{pgfmanualentry}
  \extractcommand\evalPlaceholder\marg{name placeholder or string}\@@
  \pgfmanualbody

  Evaluate the value of a placeholder after replacing (recursively) all the inner placeholders. You can also put inside any arbitrary string.
\begin{codeexample}[width=0pt,vbox]
  \setPlaceholder{__MY_PLACEHOLDER__}{Hello __NAME__, I am a template $\delta_n$.}
  \setPlaceholder{__NAME__}{Alice __NICKNAME__}
  \setPlaceholder{__NICKNAME__}{the great}
  % The placeholder evaluates to \texttt{\getPlaceholder{__MY_PLACEHOLDER__}}.
  The placeholder evaluates to:\\
  \evalPlaceholder{__MY_PLACEHOLDER__}\\
  Combining placeholders produces:\\
  \evalPlaceholder{In ``__MY_PLACEHOLDER__'', the name is __NAME__.}
\end{codeexample}
\end{pgfmanualentry}

\subsubsection{List and debug placeholders}\label{sec:listAndDebugPlaceholders}

It can sometimes be handy to list all placeholders, print their contents etc (see also \cref{sec:debug}). We list here commands that are mostly useful for debugging purposes.

\begin{pgfmanualentry}
  \extractcommand\printImportedPlaceholdersExceptDefaults\opt{*}\@@
  \makeatletter%
  \def\extrakeytext{style, }
  \extractkey/robExt/print imported placeholders except default\@nil
  \makeatother%
  \pgfmanualbody

  Prints the verbatim content of all defined and imported placeholders (without performing any replacement of inner placeholders), except for the placeholders that are defined by default in this library (that we identify as they start with |__ROBEXT_|). The stared version does print the name of the placeholder defined in this library, but not their definition. This is mostly for debugging purposes.
\begin{codeexample}[width=0pt,vbox]
\placeholderFromContent{__LIKES__}{Hello __NAME__ I am a really basic template $\delta_n$.}
\placeholderFromContent{__NAME__}{Alice}    
\printImportedPlaceholdersExceptDefaults
\end{codeexample}
Compare with:
\begin{codeexample}[width=0pt,vbox]
\placeholderFromContent{__LIKES__}{Hello __NAME__ I am a really basic template $\delta_n$.}
\placeholderFromContent{__NAME__}{Alice}    
\printImportedPlaceholdersExceptDefaults*
\end{codeexample}

\end{pgfmanualentry}

\begin{pgfmanualentry}
  \extractcommand\printImportedPlaceholders\@@
  \makeatletter%
  \def\extrakeytext{style, }
  \extractkey/robExt/print imported placeholders\@nil
  \makeatother%
  \pgfmanualbody

  Prints the verbatim content of all defined and imported placeholders (without performing any replacement of inner placeholders), including the placeholders that are defined by default in this library (those starting with |__ROBEXT_|). This is mostly for debugging purposes. Here is the result of |\printImportedPlaceholders|:

  \printImportedPlaceholders
\end{pgfmanualentry}

\begin{pgfmanualentry}
  \extractcommand\printPlaceholderNoReplacement\marg{name placeholder}\@@
  \pgfmanualbody

  Prints the verbatim content of a given placeholder, without evaluating it and \textbf{without replacing inner placeholders: it is used mostly for debugging purposes} and will be used in this documentation to display the content of the placeholder for educational purposes. The stared version prints it inline.
\begin{codeexample}[width=0pt,vbox]
  \placeholderFromContent{__LIKES__}{Hello __NAME__ I am a really basic template $\delta_n$.}
  \placeholderFromContent{__NAME__}{Alice}    
  The (unexpanded) template contains \printPlaceholderNoReplacement{__LIKES__}.\\
  The (unexpanded) template contains \printPlaceholderNoReplacement*{__LIKES__}
\end{codeexample}
\end{pgfmanualentry}

\begin{pgfmanualentry}
  \extractcommand\printPlaceholder\marg{name placeholder}\@@
  \pgfmanualbody

  Like |\printPlaceholderNoReplacement| except that it first replaces the inner placeholders. The stared version prints it inline.
\begin{codeexample}[width=0pt,vbox]
  \placeholderFromContent{__LIKES__}{Hello __NAME__ I am a really basic template $\delta_n$.}
  \placeholderFromContent{__NAME__}{Alice}    
  The (unexpanded) template contains \printPlaceholder{__LIKES__}.\\
  The (unexpanded) template contains \printPlaceholder*{__LIKES__}
\end{codeexample}
\end{pgfmanualentry}

\begin{pgfmanualentry}
  \extractcommand\evalPlaceholderNoReplacement\marg{name placeholder}\@@
  \pgfmanualbody

  Evaluates the content of a given placeholder as a \LaTeX{} code, \textbf{without replacing the placeholders contained inside (mostly used for debugging purposes).}
\begin{codeexample}[width=0pt,vbox]
  \placeholderFromContent{__LIKES__}{Hello I am a really basic template $\delta_n$.}
  The (unexpanded) template evaluates to ``\evalPlaceholderNoReplacement{__LIKES__}''.
\end{codeexample}
\end{pgfmanualentry}

\begin{pgfmanualentry}
  \extractcommand\rescanPlaceholderInVariableNoReplacement\marg{name macro}\marg{name placeholder}\@@
  \pgfmanualbody
  (new v2.0) Create a new macro that executes the \LaTeX{} code in the placeholder.
\end{pgfmanualentry}

\begin{pgfmanualentry}
  \extractcommand\getPlaceholderNoReplacement\marg{name placeholder}\@@
  \pgfmanualbody

  Like |\evalPlaceholderNoReplacement| except that it only outputs the string without evaluating the macros inside.
\begin{codeexample}[width=0pt,vbox]
  \placeholderFromContent{__LIKES__}{Hello __NAME__ I am a really basic template $\delta_n$.}
  \placeholderFromContent{__NAME__}{Alice}
  The (unexpanded) template contains \texttt{\getPlaceholderNoReplacement{__LIKES__}}
\end{codeexample}
\end{pgfmanualentry}

\subsubsection{Setting a value to a placeholder}


\begin{pgfmanualentry}
  \extractcommand\placeholderFromContent\opt{*}\marg{name placeholder}\marg{content placeholder}\@@
  \extractcommand\setPlaceholder\opt{*}\marg{name placeholder}\marg{content placeholder}\@@
  \makeatletter%
  \def\extrakeytext{style, }
  \extractkey/robExt/set placeholder=\marg{name placeholder}\marg{content placeholder}\@nil
  \extractkey/robExt/set placeholder no import=\marg{name placeholder}\marg{content placeholder}\@nil
  \extractkey/robExt/set placeholder from content=\marg{name placeholder}\marg{content placeholder}\@nil
  \makeatother%
  \pgfmanualbody

  |\placeholderFromContent| (and its alias |\setPlaceholder| and its equivalent pgf styles |/robExt/set placeholder| and |/robExt/set placeholder from content|) is useful to set a value to a given placeholder.
\begin{codeexample}[width=0pt,vbox]
  \placeholderFromContent{__LIKES__}{Hello I am a basic template with math $\delta_n$ and macros \hello}
  The (unexpanded) template contains \printPlaceholderNoReplacement{__LIKES__} and %
  after evaluation and setting the value of hello,%
  \def\hello{Hello my friend!}%
  you get ``\evalPlaceholder{__LIKES__}''.
\end{codeexample}
  As you can see, \textbf{the precise content is not exactly identical to the original string}: \LaTeX{} comments are removed, spaces are added after macros, some newlines are removed etc. While this is usually not an issue when dealing with \LaTeX{} code, it causes some troubles when dealing with non-\LaTeX{} code. For this reason, we define \textbf{other commands} (see for instance |PlaceholderFromCode| below) that can accept verbatim content; the downside being that \LaTeX{} forbids usage of these verbatim commands inside other macros, so you should always define them at the top level (this seems to be fundamental to how \LaTeX{} works, as any input to a macro gets interpreted first as a \LaTeX{} string, losing all comments for instance). Note that this is not as restrictive as it may sound, as it is always possible to define the needed placeholders before any macro, while using them inside the macro, possibly combining them with other placeholders (defined either before or inside the macro).
\end{pgfmanualentry}


\begin{pgfmanualentry}
  \extractenvironement{PlaceholderFromCode}\opt{*}\marg{name placeholder}\@@
  \extractenvironement{SetPlaceholderCode}\opt{*}\marg{name placeholder}\@@
  \pgfmanualbody

  These two (aliased) environments are useful to set a verbatim value to a given placeholder: the advantage is that you can put inside any code, including \LaTeX{} comments, the downside is that you cannot use it inside macros and some environments (so you typically define it before the macros and call it inside).
% \begin{codeexample}[width=0pt,vbox]
%   \placeholderFromContent{__PYTHON_CODE__}{}
%   The (unexpanded) template contains \printPlaceholderNoReplacement{__LIKES__} and %
%   after evaluation (no replacement), you get ``\evalPlaceholderNoReplacement{__LIKES__}''.
% \end{codeexample}

\begin{codeexample}[width=0pt,vbox]
\begin{PlaceholderFromCode}{__PYTHON_CODE__}
def my_function(b): # this is a python code
    c = {}
    d[42] = 0
    return b
\end{PlaceholderFromCode}
\printImportedPlaceholdersExceptDefaults
\end{codeexample}


Note that |PlaceholderFromCode| should not be used inside other macros or inside some environments (notably the ones that need to evaluate the body of the environment, e.g. using |+b| argument or |environ|) as verbatim content is parsed first by the macro, meaning that some characters might be changed or removed. For instance, any percent character would be considered as a comment, removing the rest of the line. However, this should not be be problem if you use it outside of any macro or environment, or if you load it from a file. For instance this code:
\begin{verbatim}
\begin{PlaceholderFromCode}{__PYTHON_CODE__}
def my_function(b): # this is a python code
    c = {}
    d[42] = 0
    return b % 2
\end{PlaceholderFromCode}
\printImportedPlaceholdersExceptDefaults
\end{verbatim}
would produce:

{
\begin{PlaceholderFromCode}{__PYTHON_CODE__}
def my_function(b): # this is a python code
    c = {}
    d[42] = 0
    return b % 2
\end{PlaceholderFromCode}
\begin{codeexample}[width=0pt,vbox]
\printImportedPlaceholdersExceptDefaults
\end{codeexample}
}
Note that of course, you can define a placeholder before a macro and call it inside (explaining how we can generate this documentation).

Note that the star and no import version does NOT import the placeholder it the main group (unless you try to optimize the compilation time you should not need it, but see \cref{sec:importSystem} for details).
\end{pgfmanualentry}

\begin{pgfmanualentry}
  \extractcommand\placeholderPathFromFilename\opt{*}\marg{name placeholder}\marg{filename}\@@
  \makeatletter%
  \def\extrakeytext{style, }
  \extractkey/robExt/set placeholder path from filename=\marg{name placeholder}\marg{filename}\@nil
  \makeatother%
  \pgfmanualbody

  |\placeholderPathFromFilename{__MYLIB__}{mylib.py}| will copy |mylib.py| in the cache (setting its hash depending on its content), and set the content of the placeholder |__MYLIB__| to the \textbf{path} of the library in the cache. Note that the path is relative to the cache folder (it is easier to use for instance if you want to call this library from a code already in the cache).
\begin{codeexample}[width=0pt,vbox]
  \placeholderPathFromFilename{__MYLIB__}{mylib.py}
  \printImportedPlaceholdersExceptDefaults
  You can also get the path relative to the root folder:\\
  \robExtAddCachePath{\getPlaceholderNoReplacement{__MYLIB__}}
\end{codeexample}
Note that the star and no import version does NOT import the placeholder it the main group (unless you try to optimize the compilation time you should not need it, but see \cref{sec:importSystem} for details).
\end{pgfmanualentry}


\begin{pgfmanualentry}
  \extractcommand\placeholderFromFileContent\opt{*}\marg{name placeholder}\marg{filename}\@@
  \makeatletter%
  \def\extrakeytext{style, }
  \extractkey/robExt/set placeholder from file content=\marg{name placeholder}\marg{filename}\@nil
  \extractkey/robExt/set placeholder from file content no import=\marg{name placeholder}\marg{filename}\@nil
  \makeatother%
  \pgfmanualbody

  |\placeholderFromFileContent{__MYLIB__}{mylib.py}| will set the content of the placeholder |__MYLIB__| to the content of |mylib.py|.
\begin{codeexample}[width=0pt,vbox]
  \placeholderFromFileContent{__MYLIB__}{mylib.py}
  \printImportedPlaceholdersExceptDefaults
\end{codeexample}
Note that the star and no import version does NOT import the placeholder it the main group (unless you try to optimize the compilation time you should not need it, but see \cref{sec:importSystem} for details).
\end{pgfmanualentry}

\begin{pgfmanualentry}
  \extractcommand\placeholderPathFromContent\opt{*}\marg{name placeholder}\opt{\oarg{suffix}}\marg{content}\@@
  \makeatletter%
  \def\extrakeytext{style, }
  \extractkey/robExt/set placeholder path from content=\marg{name placeholder}\marg{suffix}\marg{content}\@nil
  \extractkey/robExt/set placeholder path from content no import=\marg{name placeholder}\marg{suffix}\marg{content}\@nil
  \makeatother%
  \pgfmanualbody

  |\placeholderPathFromContent{__MYLIB__}{some content}| will copy |some content| in a file in the cache (setting its hash depending on its content, the filename will end with |suffix| that defaults to |.tex|), and set the content of the placeholder |__MYLIB__| to the \textbf{path} of the file in the cache. Note that the path is relative to the cache folder (it is easier to use for instance if you want to call this library from a code already in the cache).
\begin{codeexample}[width=0pt,vbox]
  \placeholderPathFromContent{__MYLIB__}[.py]{some contents b}
  \printImportedPlaceholdersExceptDefaults
  You can also get the path relative to the root folder:\\
  \robExtAddCachePath{\getPlaceholderNoReplacement{__MYLIB__}}\\
  As a sanity check, this file contains
  \verbatiminput{\robExtAddCachePath{\getPlaceholderNoReplacement{__MYLIB__}}}
\end{codeexample}
Note that the star and no import version does NOT import the placeholder it the main group (unless you try to optimize the compilation time you should not need it, but see \cref{sec:importSystem} for details).
\end{pgfmanualentry}

\begin{pgfmanualentry}
  \extractenvironement{PlaceholderPathFromCode}\opt{*}\opt{\oarg{suffix}}\marg{name placeholder}\@@
  \pgfmanualbody

  This environment is similar to |\placeholderPathFromContent| except that it accepts verbatim code (therefore \LaTeX{} comments, newlines etc. will not be removed). However, due to \LaTeX{} limitations, this environment cannot be used inside macros or some environments, or this property will not be preserved.
  For instance, if you create your placeholder using:
\begin{verbatim}
\begin{PlaceholderPathFromCode}[.py]{__MYLIB__}
def my_function(b): # this is a python code
    c = {}
    d[42] = 0
    return b % 2
\end{PlaceholderPathFromCode}
\end{verbatim}
%% The code cannot be placed inside codeexample as it needs to parse the body:
\begin{PlaceholderPathFromCode}[.py]{__MYLIB__}
def my_function(b): # this is a python code
    c = {}
    d[42] = 0
    return b % 2
\end{PlaceholderPathFromCode}
You can then use it like:
\begin{codeexample}[width=0pt,vbox]
\printImportedPlaceholdersExceptDefaults
You can also get the path relative to the root folder:\\
\robExtAddCachePath{\getPlaceholderNoReplacement{__MYLIB__}}\\
As a sanity check, this file contains
\verbatiminput{\robExtAddCachePath{\getPlaceholderNoReplacement{__MYLIB__}}}
\end{codeexample}
Note that the star version does NOT import the placeholder it the main group (unless you try to optimize the compilation time you should not need it, but see \cref{sec:importSystem} for details).
\end{pgfmanualentry}


\begin{pgfmanualentry}
  \extractcommand\copyPlaceholder\opt{*}\marg{new placeholder}\marg{old placeholder}\@@
  \makeatletter%
  \def\extrakeytext{style, }
  \extractkey/robExt/copy placeholder=\marg{new placeholder}\marg{old placeholder}\@nil
  \extractkey/robExt/copy placeholder no import=\marg{new placeholder}\marg{old placeholder}\@nil
  \makeatother%
  \pgfmanualbody

  This creates a new placeholder with the content of |old placeholder|. Note that this is different from:\\
  |\setPlaceholder{new placeholder}{old placeholder}|\\
  because if we modify |old placeholder|, this will not affect |new placeholder|.
\begin{codeexample}[width=0pt,vbox]
  \setPlaceholder{__MY_CONTENT__}{Some content}
  \copyPlaceholder{__MY_OLD_CONTENT__}{__MY_CONTENT__}
  \setPlaceholder{__MY_CONTENT__}{The content used to be __MY_OLD_CONTENT__}
  \printImportedPlaceholdersExceptDefaults
\end{codeexample}
Note that the star and no import version does NOT import the placeholder it the main group (unless you try to optimize the compilation time you should not need it, but see \cref{sec:importSystem} for details).
\end{pgfmanualentry}

Now, we see how we can define a placeholder recursively, by giving it a value based on its previous value (useful for instance in order to add stuff to it).

\begin{pgfmanualentry}
  \extractcommand\setPlaceholderRec\marg{new placeholder}\marg{content with placeholder}\@@
  \extractcommand\setPlaceholderRecReplaceFromList\marg{list,of,placeholder,to,replace}\marg{new placeholder}\marg{content with placeholder}\@@
  \makeatletter%
  \def\extrakeytext{style, }
  \extractkey/robExt/set placeholder rec=\marg{name placeholder}\marg{content placeholder}\@nil
  \extractkey/robExt/set placeholder rec replace from list=\marg{list,of,placeholder,to,replace}\marg{name placeholder}\marg{content placeholder}\@nil
  \makeatother%
  \pgfmanualbody

  |\setPlaceholderRec{foo}{bar}| is actually an alias for |\getPlaceholderInResult[foo]{bar}|. Note that contrary to |\setPlaceholder|, it recursively replaces all inner placeholders. This is particularly useful to add stuff to an existing (or not) placeholder:
\begin{codeexample}[width=0pt,vbox]
\setPlaceholderRec{__MY_COMMAND__}{pdflatex}
\setPlaceholderRec{__MY_COMMAND__}{__MY_COMMAND__ myfile}
\printImportedPlaceholdersExceptDefaults
\end{codeexample}
Note that the if the placeholder content contains at the end the placeholder name, we will automatically remove it to avoid infinite recursion at evaluation time. This has the benefit that you can add something to a placeholder even if this placeholder does not exists yet (in which case it will be understood as the empty string):
\begin{codeexample}[width=0pt,vbox]
\setPlaceholderRec{__COMMAND_ARGS__}{__COMMAND_ARGS__ -l}
\setPlaceholderRec{__COMMAND_ARGS__}{__COMMAND_ARGS__ -s}
\printImportedPlaceholdersExceptDefaults
\end{codeexample}
The variation |\setPlaceholderRecReplaceFromList| allows us to specify a subset of placeholder that will be allowed to be expanded, and is an alias for |\getPlaceholderInResultReplaceFromList| (except that the optional argument is mandatory):
\begin{codeexample}[width=0pt,vbox]
\placeholderFromContent{__MY_PLACEHOLDER__}{Hello __NAME1__, __NAME2__ and __NAME3__,
  I am a template $\delta_n$.
}
\placeholderFromContent{__NAME1__}{Alice}
\placeholderFromContent{__NAME2__}{Bob}
\placeholderFromContent{__NAME3__}{Charlie}
\setPlaceholderRecReplaceFromList{__MY_PLACEHOLDER__,__NAME2__,__NAME3__}{__OTHER_PLACEHOLDER__}{
  Here we go: __MY_PLACEHOLDER__
}
\printImportedPlaceholdersExceptDefaults
\end{codeexample}
Note that the star and no import version does NOT import the placeholder it the main group (unless you try to optimize the compilation time you should not need it, but see \cref{sec:importSystem} for details).
\end{pgfmanualentry}

Note that sometimes, you might not want to use |\setPlaceholderRec| to simply append some data to the placeholder as it will also evaluate the inner placeholders (meaning that you will not be able to redefine them later). For this reason, we also provide functions to add something to the placeholder without evaluating it first:
\begin{pgfmanualentry}
  \extractcommand\addToPlaceholder\opt{*}\marg{placeholder}\marg{content to add}\@@
  \extractcommand\addToPlaceholderNoImport\opt{*}\marg{placeholder}\marg{content to add}\@@
  \extractcommand\addBeforePlaceholder\opt{*}\marg{placeholder}\marg{content to add}\@@
  \extractcommand\addBeforePlaceholderNoImport\opt{*}\marg{placeholder}\marg{content to add}\@@
  \makeatletter%
  \def\extrakeytext{style, }
  \extractkey/robExt/add to placeholder=\marg{name placeholder}\marg{content to add}\@nil
  \extractkey/robExt/add to placeholder no space=\marg{name placeholder}\marg{content to add}\@nil
  \extractkey/robExt/add before placeholder=\marg{name placeholder}\marg{content to add}\@nil
  \extractkey/robExt/add before placeholder no space=\marg{name placeholder}\marg{content to add}\@nil
  \extractkey/robExt/add before main content=\marg{name placeholder}\marg{content to add}\@nil
  \extractkey/robExt/add to placeholder no import=\marg{name placeholder}\marg{content to add}\@nil
  \extractkey/robExt/add to placeholder no space no import=\marg{name placeholder}\marg{content to add}\@nil
  \extractkey/robExt/add before placeholder no import=\marg{name placeholder}\marg{content to add}\@nil
  \extractkey/robExt/add before placeholder no space no import=\marg{name placeholder}\marg{content to add}\@nil
  \extractkey/robExt/add before main content no import=\marg{name placeholder}\marg{content to add}\@nil
  \makeatother%
  \pgfmanualbody

  |\addToPlaceholder{foo}{bar}| adds |bar| at the end of the placeholder |foo| (by default it also adds a space, unless you use the star version), creating it if it does not exist (the |before| variants add the content\dots{} before).
\begin{codeexample}[width=0pt,vbox]
\setPlaceholder{__ENGINE__}{pdflatex}
\setPlaceholder{__COMMAND__}{__ENGINE__ --option}
\addToPlaceholder{__COMMAND__}{--other}
\addToPlaceholder*{__COMMAND__}{-option}
\addBeforePlaceholder{__COMMAND__}{time}
\printImportedPlaceholdersExceptDefaults
\end{codeexample}
|add before main content| is a particular case where the placeholder is |__ROBEXT_MAIN_CONTENT__|. It is practical if you want to define for instance a macro, but in a way that even if you disable externalization, the command should still compile (if you define the macro in the preamble, it will not be added when disabling externalization). For instance:
\begin{codeexample}[width=0pt,vbox]
\robExtConfigure{
  new preset={my preset}{latex, add before main content={\def\hello#1{Hello #1.}}},
}
\cacheMe[my preset]{Here I am cached (\hello{Alice})} and \cacheMe[my preset, disable externalization]{Here I am not (\hello{Bob}).}
\end{codeexample}
The |no import| versions do not import the placeholder in the current group (only needed if you want to optimize the compilation time, see \cref{sec:importSystem} for details).
\end{pgfmanualentry}

\begin{pgfmanualentry}
  \extractcommand\placeholderFromString\opt{*}\marg{latex3 string}\@@
  \extractcommand\setPlaceholderFromString\opt{*}\marg{latex3 string}\@@
  \pgfmanualbody
  (new in v2.0) This allows you to assign an existing \LaTeX{}3 string to a placeholder (the star version does not import the placeholder, see \cref{sec:importSystem}).
\begin{codeexample}[width=0pt,vbox]
  \ExplSyntaxOn
  \setPlaceholderFromString{__my_percent_string__}{\c_percent_str}
  \printImportedPlaceholdersExceptDefaults
  \ExplSyntaxOff
\end{codeexample}
We provide a list of placeholders that are useful to escape parts of the strings (but you should not really need them, if you need weird characters like percent, most of the time you want to use placeholderFromCode):
\begin{codeexample}[width=0pt,vbox]
  \printPlaceholder{
    String containing
    __ROBEXT_LEFT_BRACE__,
    __ROBEXT_RIGHT_BRACE__,
    __ROBEXT_BACKSLASH__,
    __ROBEXT_HASH__,
    __ROBEXT_UNDERSCORE__,
    __ROBEXT_PERCENT__.
  }
\end{codeexample}
\end{pgfmanualentry}

\begin{pgfmanualentry}
  \extractcommand\placeholderReplaceInplace\marg{placeholder}\marg{from}\marg{to}\@@
  \extractcommand\placeholderReplaceInplaceEval\marg{placeholder}\marg{from}\marg{to}\@@
  \makeatletter%
  \def\extrakeytext{style, }
  \extractkey/robExt/placeholder replace in place=\marg{placeholder}\marg{from}\marg{to}\@nil
  \extractkey/robExt/placeholder replace in place eval=\marg{placeholder}\marg{from}\marg{to}\@nil
  \makeatother%
  \pgfmanualbody
  (new in v2.0) This allows you to replace a value in a placeholder. The |eval| variation first evaluates the string.
\begin{codeexample}[width=0pt,vbox]
\def\nameFrom{Bob}
\def\nameTo{Dylan}
\robExtConfigure{
  set placeholder={__NAMES__}{Hello Alice and Bob.},
  placeholder replace in place={__NAMES__}{Alice}{Charlie},
  placeholder replace in place eval={__NAMES__}{\nameFrom}{\nameTo},
}
\printImportedPlaceholdersExceptDefaults
\end{codeexample}
\end{pgfmanualentry}

\begin{pgfmanualentry}
  \extractcommand\placeholderHalveNumberHashesInplace\marg{placeholder}\@@
  \extractcommand\placeholderDoubleNumberHashesInplace\marg{placeholder}\@@
  \makeatletter%
  \def\extrakeytext{style, }
  \extractkey/robExt/placeholder halve number hashes in place=\marg{placeholder}\@nil
  \extractkey/robExt/placeholder double number hashes in place=\marg{placeholder}\@nil
  \makeatother%
  \pgfmanualbody
  (new in v2.0) This allows you to either turn any |##| into |#| or the other way around (may be practical when dealing with arguments to functions).
\begin{codeexample}[width=0pt,vbox]
\robExtConfigure{
  set placeholder={__DEMO__}{\def\hey##1{Hey ##1.}},
  placeholder halve number hashes in place={__DEMO__},
}
\printImportedPlaceholdersExceptDefaults
\end{codeexample}
\end{pgfmanualentry}

\begin{pgfmanualentry}
  \extractcommand\removePlaceholder\marg{placeholder}\@@
  \makeatletter%
  \def\extrakeytext{style, }
  \extractkey/robExt/remove placeholder=\marg{placeholder}\@nil
  \extractkey/robExt/remove placeholders=\marg{list,of,placeholder}\@nil
  \makeatother%
  \pgfmanualbody
  (new in v2.0) Remove placeholders.
\begin{codeexample}[width=0pt,vbox]
\robExtConfigure{
  set placeholder={__DEMOA__}{Alice},
  set placeholder={__DEMOB__}{Bob},
  set placeholder={__DEMOC__}{Charlie},
  remove placeholders={__DEMOA__,__DEMOC__},
}
\printImportedPlaceholdersExceptDefaults
\end{codeexample}
\end{pgfmanualentry}



\begin{pgfmanualentry}
  \extractcommand\evalPlaceholderInplace\marg{name placeholder}\@@
  \makeatletter%
  \def\extrakeytext{style, }
  \extractkey/robExt/eval placeholder inplace=\marg{name placeholder}\@nil
  \makeatother%
  \pgfmanualbody
  This command will update (inplace) the content of a macro by first replacing recursively the placeholders, and finally by expanding the  \LaTeX{} macros.
\begin{codeexample}[width=0pt,vbox]
\def\mymacro{Initial value}    
\placeholderFromContent{__MACRO_NOT_EVALUATED__}{\mymacro}
\placeholderFromContent{__MACRO_EVALUATED__}{\mymacro}
\evalPlaceholderInplace{__MACRO_EVALUATED__}
\printImportedPlaceholdersExceptDefaults
\def\mymacro{Final value}    
Compare \evalPlaceholder{__MACRO_EVALUATED__} and \evalPlaceholder{__MACRO_NOT_EVALUATED__}.
\end{codeexample}
\end{pgfmanualentry}

\begin{pgfmanualentry}
  \makeatletter%
  \def\extrakeytext{style, }
  \extractkey/robExt/set placeholder eval=\marg{name placeholder}\marg{content placeholder}\@nil
  \makeatother%
  \pgfmanualbody
  Alias for |\setPlaceholderRec{#1}{#2}\evalPlaceholderInplace{#1}|: set and evaluate recursively the placeholders and macros. This can be practical to pass the value of a counter/macro to the template (of course, if this value is fixed, you can also directly load it from the preambule):
\begin{codeexample}[width=0pt,vbox]
\begin{CacheMe}{tikzpicture, set placeholder eval={__thepage__}{\thepage}}
  \node[rounded corners, fill=red]{The current page is __thepage__.};
\end{CacheMe}
\end{codeexample}
Note that this works well for commands that expand completely, but some more complex commands might not expand properly (like |cref|). I need to investigate how to solve this issue, meanwhile you can still disable externalization for these pictures.
\end{pgfmanualentry}

\subsubsection{Groups: the import system}\label{sec:importSystem}

(This whole system was added v2.0.)

In order to replace placeholders, we need to know their list, but to get the best possible performance we do not maintain a single list of all placeholders since \LaTeX{} is quite slow when doing list manipulations. Therefore, we group them inside smaller groups (e.g.\ one group for |latex|, one group for |python| etc), and the user is free to import groups. By default, new placeholders are imported inside the |main| group, and this group will be used when doing placeholder replacements. Note that for most of the case, we define a star variant of functions like |\setPlaceholder*{__FOO__}{bar}| (or |set placeholder no import| for styles) in order to create a placeholder without importing it (but this is certainly not used a lot by end users unless they want to make them even faster).

Note that basically all commands to create a placeholder have a command/environment where you add a star like |\setPlaceholder*{}| or a style where you add |no import| like |set placeholder no import| in order to create a placeholder that is not imported at all.

\begin{pgfmanualentry}
  \extractcommand\importPlaceholder\marg{name placeholder}\@@
  \extractcommand\importPlaceholdersFromGroup\marg{name placeholder}\@@
  \extractcommand\importAllPlaceholders\@@
  \extractcommand\importPlaceholderFirst\@@
  \makeatletter%
  \def\extrakeytext{style, }
  \extractkey/robExt/import placeholder=\marg{name placeholder}\@nil
  \extractkey/robExt/import placeholders from group=\marg{name group}\@nil
  \extractkey/robExt/import all placeholders\@nil
  \extractkey/robExt/import placeholder first=\marg{name placeholders}\@nil
  \makeatother%
  \pgfmanualbody
  Import placeholders, either individually, from a group of placeholders (exists either in style form or command), or from all registered groups (warning: you should avoid importing all placeholders if care too much about efficiency, as this can significantly slow down the replacement procedure). The |first| variant inserts the placeholder at the beginning of the import list, which can speed up compilation when this placeholder must be replaced first (but not so useful for basic usage).
\begin{codeexample}[width=0pt,vbox]
\robExtConfigure{
  set placeholder no import={__name__}{Alice},
  set placeholder no import={__name2__}{Bob},
  print imported placeholders except default,
  import placeholder={__name__},
  print imported placeholders except default,
  print imported placeholders,
  import placeholders from group={latex},
  print imported placeholders,  
}  
\end{codeexample}

\begin{codeexample}[width=0pt,vbox]
\printPlaceholder{First try: __ROBEXT_LATEX_TRIM_LENGTH__}
\importAllPlaceholders
\printPlaceholder{Second try: __ROBEXT_LATEX_TRIM_LENGTH__}
\end{codeexample}
\end{pgfmanualentry}

\begin{pgfmanualentry}
  \extractcommand\clearImportedPlaceholders\marg{name placeholder}\@@
  \makeatletter%
  \def\extrakeytext{style, }
  \extractkey/robExt/clear imported placeholders\@nil
  \makeatother%
  \pgfmanualbody
  Clear all imported placeholders (comma separated list of placeholders)
\begin{codeexample}[width=0pt,vbox]
\robExtConfigure{
  set placeholder={__name__}{Alice},
  print imported placeholders except default,
  clear imported placeholders,
  print imported placeholders except default,
}  
\end{codeexample}
\end{pgfmanualentry}

\begin{pgfmanualentry}
  \extractcommand\removeImportedPlaceholder\marg{name placeholder}\@@
  \makeatletter%
  \def\extrakeytext{style, }
  \extractkey/robExt/remove imported placeholders\@nil
  \makeatother%
  \pgfmanualbody
  Comma separated list of placeholders to remove from the list of imported placeholders.
\begin{codeexample}[width=0pt,vbox]
\robExtConfigure{
  set placeholder={__name__}{Alice},
  set placeholder={__name2__}{Alice},
  print imported placeholders except default,
  remove imported placeholders={__name__,__name2__},
  print imported placeholders except default,
}  
\end{codeexample}
\end{pgfmanualentry}

\begin{pgfmanualentry}
  \extractcommand\printAllRegisteredGroups\@@
  \extractcommand\printAllRegisteredGroupsAndPlaceholders\opt{*}\@@
  \makeatletter%
  \def\extrakeytext{style, }
  \extractkey/robExt/print all registered groups\@nil
  \extractkey/robExt/print all registered groups and placeholders\@nil
  \extractkey/robExt/show all registered groups\@nil
  \extractkey/robExt/show all registered groups and placeholders\@nil
  \makeatother%
  \pgfmanualbody
  Print (or shows in the console for the alternative versions) all registered groups of placeholders (possibly with the placeholders they contain), mostly for debugging purpose, with the star prints also the content of the inner placeholders.
\begin{codeexample}[width=0pt,vbox]
\printAllRegisteredGroups
\end{codeexample}
\end{pgfmanualentry}

\begin{pgfmanualentry}
  \extractcommand\printGroupPlaceholders\opt{*}\@@
  \makeatletter%
  \def\extrakeytext{style, }
  \extractkey/robExt/print group placeholders=\marg{name group}\@nil
  \makeatother%
  \pgfmanualbody
  Print all placeholders (star = with content) of a given group.
\begin{codeexample}[width=0pt,vbox]
\printGroupPlaceholders*{bash}
\end{codeexample}
\end{pgfmanualentry}

\begin{pgfmanualentry}
  \extractcommand\printAllRegisteredGroupsAndPlaceholders\opt{*}\@@
  \pgfmanualbody
  Print all registered groups of placeholders with their inner content, mostly for debugging purpose, with the star prints only the content of the inner placeholders.
\begin{codeexample}[width=0pt,vbox]
\printAllRegisteredGroupsAndPlaceholders
\end{codeexample}
\end{pgfmanualentry}

\begin{pgfmanualentry}
  \extractcommand\newGroupPlaceholders\marg{name group}\@@
  \extractcommand\addPlaceholderToGroup\marg{name group}\marg{list,of,placeholders}\@@
  \extractcommand\addPlaceholdersToGroup\marg{name group}\marg{list,of,placeholders}\@@
  \extractcommand\removePlaceholderFromGroup\marg{name group}\marg{list,of,placeholders}\@@
  \extractcommand\removePlaceholdersFromGroup\marg{name group}\marg{list,of,placeholders}\@@
  \makeatletter%
  \def\extrakeytext{style, }
  \extractkey/robExt/new group placeholders=\marg{name group}\@nil
  \extractkey/robExt/add placeholder to group=\marg{name group}\marg{name placeholder}\@nil
  \extractkey/robExt/add placeholders to group=\marg{name group}\marg{list,of,placeholders}\@nil
  \extractkey/robExt/add placeholders to group=\marg{name group}\marg{list,of,placeholders}\@nil
  \extractkey/robExt/remove placeholders from group=\marg{name group}\marg{list,of,placeholders}\@nil
  \extractkey/robExt/remove placeholder from group=\marg{name group}\marg{list,of,placeholders}\@nil
  \makeatother%
  \pgfmanualbody
  Create a new group and add placeholders to it (you can put multiple placeholders separated by commas).
\begin{codeAndResult}
\robExtConfigure{
  new group placeholders={my dummy group},
  set placeholder no import={__name__}{Alice},
  add placeholders to group={my dummy group}{__name__},
}
\printGroupPlaceholders{my dummy group}
\removePlaceholdersFromGroup{my dummy group}{__name__}
\printGroupPlaceholders{my dummy group}
\end{codeAndResult}
\end{pgfmanualentry}

\begin{pgfmanualentry}
  \extractcommand\clearGroupPlaceholders\marg{name group}\@@
  \makeatletter%
  \def\extrakeytext{style, }
  % \extractkey/robExt/new group placeholders=\marg{name group}\@nil
  \makeatother%
  \pgfmanualbody
  Remove all elements in a group.
\begin{codeexample}[vbox]
\robExtConfigure{
  new group placeholders={my dummy group},
  set placeholder no import={__name__}{Alice},
  add placeholders to group={my dummy group}{__name__},
}
\clearGroupPlaceholders{my dummy group}
\printGroupPlaceholders{my dummy group}
\end{codeexample}
\end{pgfmanualentry}


\begin{pgfmanualentry}
  \extractcommand\copyGroupPlaceholders\marg{name group}\marg{name group to copy}\@@
  \makeatletter%
  \def\extrakeytext{style, }
  \extractkey/robExt/copy group placeholders=\marg{name group}\marg{name group to copy}\@nil
  \makeatother%
  \pgfmanualbody
  Remove all elements in a group.
\begin{codeexample}[vbox]
\robExtConfigure{
  new group placeholders={my dummy group},
  set placeholder no import={__name__}{Alice},
  add placeholders to group={my dummy group}{__name__},
}
\copyGroupPlaceholders{my copy}{my dummy group}
\printGroupPlaceholders{my copy}
\end{codeexample}
\end{pgfmanualentry}

\begin{pgfmanualentry}
  \extractcommand\appendGroupPlaceholders\marg{name group}\marg{name group to append}\@@
  \extractcommand\appendBeforeGroupPlaceholders\marg{name group}\marg{name group to append}\@@
  \makeatletter%
  \def\extrakeytext{style, }
  \extractkey/robExt/append group placeholders=\marg{name group}\marg{name group to copy}\@nil
  \extractkey/robExt/append before group placeholders=\marg{name group}\marg{name group to copy}\@nil
  \makeatother%
  \pgfmanualbody
  Append a group to another group (after or before).
\begin{codeexample}[vbox]
\robExtConfigure{
  new group placeholders={my dummy group},
  set placeholder no import={__name__}{Alice},
  add placeholders to group={my dummy group}{__name__},
  new group placeholders={my dummy group2},
  set placeholder no import={__name2__}{Bob},
  add placeholders to group={my dummy group2}{__name2__},
  append group placeholders={my dummy group}{my dummy group2},
}
\printGroupPlaceholders{my dummy group}
\end{codeexample}
\end{pgfmanualentry}

\begin{pgfmanualentry}
  \extractcommand\importPlaceholdersFromGroup\marg{name group}\@@
  \extractcommand\importAllPlaceholders\@@
  \makeatletter%
  \def\extrakeytext{style, }
  \extractkey/robExt/import placeholders from group=\marg{name group}\@nil
  \extractkey/robExt/import all placeholders\@nil
  \makeatother%
  \pgfmanualbody
  Import a group/all registered groups (note that this is equivalent to appending to the group called |main|). Note that for the later, you should be sure that the group is ``registered'', which is the case if you created it via |new group placeholders|.
\begin{codeexample}[vbox]
\robExtConfigure{
  new group placeholders={my dummy group},
  set placeholder no import={__name__}{Alice},
  add placeholder to group={my dummy group}{__name__},
  import all placeholders,
}
\printImportedPlaceholdersExceptDefaults
\end{codeexample}
\end{pgfmanualentry}


\subsection{Caching a content}

\subsubsection{Basics}

\begin{pgfmanualentry}
  \extractcommand\cacheMe\opt{\oarg{preset style}}\marg{content to cache}\@@
  \extractenvironement{CacheMe}\marg{preset style}\@@
  \pgfmanualbody
  This command (and its environment alias) is the main entry point if you want to cache the result of a file. The preset style is a pgfkeys-based style that is used to configure the template that is used, the compilation command, and more. You can either inline the style, or use some presets that configure the style automatically. After evaluating the style, the placeholders |__ROBEXT_TEMPLATE__| (containing the content of the file) and |__ROBEXT_COMPILATION_COMMAND__| (containing the compilation command run in the cache folder, that can use other placeholders internally like |__ROBEXT_SOURCE_FILE__| to get the path to the source file) should be set. Note that we provide some basic styles that allow settings these placeholders easily. See \cref{sec:placeholders} for a list of existing placeholders and presets. The placeholder |__ROBEXT_MAIN_CONTENT_ORIG__| will automatically be set by this command (or environment) so that it equals the content of the second argument (or the body of the environment). By default, |__ROBEXT_MAIN_CONTENT__| will point to |__ROBEXT_MAIN_CONTENT_ORIG__|, possibly wrapping it inside an environment like |\begin{tikzpicture}| (most of the time, you want to modify and display |__ROBEXT_MAIN_CONTENT__| rather than the |_ORIG_| to easily recover the input of the user). This style can also configure the command to use to include the file and more. By default it will insert the compiled PDF, making sure that the depth is respected (internally, we read the depth from an aux file created by our \LaTeX{} preset), but you can easily change it to anything you like.

  For an educational purpose, we write here an example that does not exploit any preset. In practice, we recommend however to use our presets, or to define new presets based on our presets (see below for examples).
\begin{codeexample}[width=0pt,vbox]
\begin{CacheMe}{set template={
      \documentclass{standalone}
      \begin{document}
      __ROBEXT_MAIN_CONTENT__
      \end{document}
    },
    set compilation command={pdflatex -shell-escape -halt-on-error "__ROBEXT_SOURCE_FILE__"},
    custom include command={%
      \includegraphics[width=4cm,angle=45]{\robExtAddCachePathAndName{\robExtFinalHash.pdf}}%
    },
  }
This content is cached $\delta$.    
\end{CacheMe}
\end{codeexample}
\end{pgfmanualentry}

\begin{pgfmanualentry}
  \extractcommand\robExtConfigure\marg{preset style}\@@
  \makeatletter%
  \def\extrakeytext{style, }
  \extractkey/robExt/new preset=\marg{name preset}\marg{preset options}\@nil
  \extractkey/robExt/add to preset=\marg{name preset}\marg{preset options}\@nil
  \makeatother%
  \pgfmanualbody
  You can then create your own style (or preset) in |\robExtConfigure| (that is basically an alias for |\pgfkeys{/robExt/.cd,#1}|) containing your template, add your own placeholders and commands to configure them etc. We provide two helper functions since v2.0:\\
  |new preset={your preset}{your configuration}|\\
  and\\
  |add to preset={your preset}{your configuration}|\\
  in order to create/modify the presets. You can also use |my preset/.style| or\\
  |my preset/.append style|\\
  to configure them instead, but in that case make sure to double the number of hashes like in |\def\mymacro##1{hello ##1}|, as the |#1| in |\def\mymacro#1{hello #1}.| would be understood as the (non-existent) argument of |my preset|.

\begin{codeexample}[width=0pt,vbox]
%% Define your presets once:
\robExtConfigure{%
  new preset={my latex preset}{
    %% Create a default value for my new placeholders:
    set placeholder={__MY_COLOR__}{red},
    set placeholder={__MY_ANGLE__}{45},
    % We can also create custom commands to "hide" the notion of placeholder
    set my angle/.style={
      set placeholder={__MY_ANGLE__}{#1}
    },
    set template={
      \documentclass{standalone}
      \usepackage{xcolor}
      \begin{document}
      \color{__MY_COLOR__}__ROBEXT_MAIN_CONTENT__
      \end{document}
    },
    set compilation command={pdflatex -shell-escape -halt-on-error "__ROBEXT_SOURCE_FILE__"},
    custom include command={%
      % The include command is a regular LaTeX command, but using
      % \evalPlaceholder avoids the need to play with expandafter, getPlaceholder etc...
      \evalPlaceholder{% 
        \includegraphics[width=4cm,angle=__MY_ANGLE__,origin=c]{%
          \robExtAddCachePathAndName{\robExtFinalHash.pdf}%
        }%
      }%
    },
  },
}

% Reuse them later...
\begin{CacheMe}{my latex preset}
This content is cached $\delta$.    
\end{CacheMe}
% And configure them at will
\begin{CacheMe}{my latex preset, set placeholder={__MY_COLOR__}{green}, set my angle=-45}
This content is cached $\delta$.    
\end{CacheMe}
\end{codeexample}
\end{pgfmanualentry}

\begin{pgfmanualentry}
  \extractenvironement{CacheMeCode}\marg{preset style}\@@
  \pgfmanualbody
  Like |CacheMe|, except that the code is read verbatim by \LaTeX{}. This way, you can put non-\LaTeX{} code inside safely, but you will not be able to use it inside a macro or some environments that read their body. Here is an example where we define an environment that automatically import matplotlib, save the figure, and insert it into a figure. Note that we define in this example new commands to type |set caption=foo| instead of |set placeholder={__FIG_CAPTION__}{foo}|.
%% codeexample cannot deal with verbatim content

\begin{codeexample}[code only]
%% Define the python code to use as a template
%% (impossible to define it in \robExtConfigure directly since
%% it is a verbatim environment)
\begin{PlaceholderFromCode}{__MY_MATPLOTLIB_TEMPLATE__}
import matplotlib.pyplot as plt
import sys
__ROBEXT_MAIN_CONTENT__
plt.savefig("__ROBEXT_OUTPUT_PDF__")  
\end{PlaceholderFromCode}

% Create a new preset called matplotlib
\robExtConfigure{
  new preset={matplotlib figure}{
    set template={__MY_MATPLOTLIB_TEMPLATE__},
    set compilation command={python "__ROBEXT_SOURCE_FILE__"},
    set caption/.style={
      set placeholder={__FIG_CAPTION__}{#1}
    },
    set label/.style={
      set placeholder={__FIG_LABEL__}{#1}
    },
    set includegraphics options/.style={
      set placeholder={__INCLUDEGRAPHICS_OPTIONS__}{#1}
    },
    set caption={},
    set label={},
    set includegraphics options={width=1cm},
    custom include command={%
      \evalPlaceholder{%
        \begin{figure}
          \centering
          \includegraphics[__INCLUDEGRAPHICS_OPTIONS__]{\robExtAddCachePathAndName{\robExtFinalHash.pdf}}%
          \caption{__FIG_CAPTION__a}
          \label{__FIG_LABEL__}
        \end{figure}%
      }%
    },
  },
}

%% Use it
\begin{CacheMeCode}{matplotlib figure, set includegraphics options={width=.8\linewidth}, set caption={Hello}}
year = [2014, 2015, 2016, 2017, 2018, 2019]  
tutorial_count = [39, 117, 111, 110, 67, 29]
plt.plot(year, tutorial_count, color="#6c3376", linewidth=3)  
plt.xlabel('Year')  
plt.ylabel('Number of futurestud.io Tutorials') 
\end{CacheMeCode}    
\end{codeexample}

%% Define the python code to use as a template
%% (impossible to define it in \robExtConfigure directly since
%% it is a verbatim environment)
\begin{PlaceholderFromCode}{__MY_MATPLOTLIB_TEMPLATE__}
import matplotlib.pyplot as plt
import sys
__ROBEXT_MAIN_CONTENT__
plt.savefig("__ROBEXT_OUTPUT_PDF__")  
\end{PlaceholderFromCode}

% Create a new preset called matplotlib
\robExtConfigure{
  new preset={matplotlib figure}{
    set template={__MY_MATPLOTLIB_TEMPLATE__},
    set compilation command={python "__ROBEXT_SOURCE_FILE__"},
    set caption/.style={
      set placeholder={__FIG_CAPTION__}{#1}
    },
    set label/.style={
      set placeholder={__FIG_LABEL__}{#1}
    },
    set includegraphics options/.style={
      set placeholder={__INCLUDEGRAPHICS_OPTIONS__}{#1}
    },
    set caption={},
    set label={},
    set includegraphics options={width=1cm},
    custom include command={%
      \evalPlaceholder{%
        \begin{figure}
          \centering
          \includegraphics[__INCLUDEGRAPHICS_OPTIONS__]{\robExtAddCachePathAndName{\robExtFinalHash.pdf}}%
          \caption{__FIG_CAPTION__}
          \label{__FIG_LABEL__}
        \end{figure}%
      }%
    },
  },
}

%% Use it
\begin{CacheMeCode}{matplotlib figure, set includegraphics options={width=.8\linewidth}, set caption={An example to show how matplotlib pictures can be inserted}}
year = [2014, 2015, 2016, 2017, 2018, 2019]  
tutorial_count = [39, 117, 111, 110, 67, 29]
plt.plot(year, tutorial_count, color="#6c3376", linewidth=3)  
plt.xlabel('Year')  
plt.ylabel('Number of futurestud.io Tutorials') 
\end{CacheMeCode}

Note that as we explained it before, due to \LaTeX{} limitations, it is impossible to call |CacheMeCode| inside macros and inside some environments that evaluate their body. To avoid that issue, it is always possible to define the macro before and call it inside. We will exemplify this on the previous example, but note that \textbf{this example is only for educational purposes} since the environment |figure| does not evaluate its body, and |CacheMeCode| can therefore safely be used inside without using this trickery:
\begin{codeexample}[code only]
%% Define the python code to use as a template
%% (impossible to define it in \robExtConfigure directly since
%% it is a verbatim environment)
\begin{PlaceholderFromCode}{__MY_MATPLOTLIB_TEMPLATE__}
import matplotlib.pyplot as plt
import sys
__ROBEXT_MAIN_CONTENT__
plt.savefig("__ROBEXT_OUTPUT_PDF__")  
\end{PlaceholderFromCode}

% Create a new preset called matplotlib
\robExtConfigure{
  new preset={matplotlib}{
    set template={__MY_MATPLOTLIB_TEMPLATE__},
    set compilation command={python "__ROBEXT_SOURCE_FILE__"},
    set includegraphics options/.style={
      set placeholder={__INCLUDEGRAPHICS_OPTIONS__}{#1}
    },
    set includegraphics options={width=1cm},
    custom include command={%
      \evalPlaceholder{%
        \includegraphics[__INCLUDEGRAPHICS_OPTIONS__]{\robExtAddCachePathAndName{\robExtFinalHash.pdf}}%
      }%
    },
  },
}


%% You cannot use CacheMeCode inside some macros or environments due to fundamental LaTeX limitations.
%% But you can always define them before, and call them inside:
\begin{SetPlaceholderCode}{__TMP__}
year = [2014, 2015, 2016, 2017, 2018, 2019]  
tutorial_count = [39, 117, 111, 110, 67, 29]
plt.plot(year, tutorial_count, color="#6c3376", linewidth=3)  
plt.xlabel('Year')  
plt.ylabel('Number of futurestud.io Tutorials') 
\end{SetPlaceholderCode}    

\begin{figure}
  \centering
  \cacheMe[matplotlib, set includegraphics options={width=.8\linewidth}, set caption={Hello}]{__TMP__}
  \caption{An example to show how code can be inserted into macros or environments that evaluate their contents (this trick is actually not needed for figures)}
\end{figure}
\end{codeexample}
\end{pgfmanualentry}


\subsubsection{Options to configure the template}


\begin{pgfmanualentry}
  \makeatletter%
  \def\extrakeytext{style, }
  \extractkey/robExt/set template=\marg{content template}\@nil
  \makeatother%
  \pgfmanualbody
  Style that alias to |set placeholder={__ROBEXT_TEMPLATE__}{#1}|, in order to define the placeholder that will hold the template of the final file.
\end{pgfmanualentry}

\begin{pgfmanualentry}
  \makeatletter%
  \def\extrakeytext{style, }
  \extractkey/robExt/set template=\marg{content template}\@nil
  \makeatother%
  \pgfmanualbody
  Style that alias to |set placeholder={__ROBEXT_TEMPLATE__}{#1}|, in order to define the placeholder that will hold the template of the final file.
\end{pgfmanualentry}

\subsubsection{Options to configure the compilation command}\label{sec:configureCompilationCommand}

\begin{pgfmanualentry}
  \makeatletter%
  \def\extrakeytext{style, }
  \extractkey/robExt/set compilation command=\marg{compilation command}\@nil
  \makeatother%
  \pgfmanualbody
  Style that alias to |set placeholder={__ROBEXT_COMPILATION_COMMAND__}{#1}|, in order to define the placeholder that will hold the compilation command.
\end{pgfmanualentry}

\begin{pgfmanualentry}
  \makeatletter%
  \def\extrakeytext{style, }
  \extractkey/robExt/add argument to compilation command=\marg{argument}\@nil
  \extractkey/robExt/add arguments to compilation command=\marg{argument}\@nil
  \makeatother%
  \pgfmanualbody
  |add argument to compilation command| is a style that alias to:\\
  |set placeholder={__ROBEXT_COMPILATION_COMMAND__}{__ROBEXT_COMPILATION_COMMAND__ "#1"}|
  in order to add an argument to the compilation command. |add arguments to compilation command| (note the |s|) accepts multiple arguments separated by a comma.
\end{pgfmanualentry}

\begin{pgfmanualentry}
  \makeatletter%
  \def\extrakeytext{style, }
  \extractkey/robExt/add key value argument to compilation command=\marg{key=value}\@nil
  \makeatother%
  \pgfmanualbody
  Adds to the command line two arguments |key| and |value|. This is a way to quickly pass arguments to a script: the script just needs to loop over the arguments and consider the odd elements as keys and the next elements as the value. Another option is to insert some placeholders directly in the script.
\end{pgfmanualentry}

\begin{pgfmanualentry}
  \makeatletter%
  \def\extrakeytext{style, }
  \extractkey/robExt/add key and file argument to compilation command=\marg{key=filename}\@nil
  \makeatother%
  \pgfmanualbody
  |filename| is the path to a file in the root folder. This adds, as:\\
  |add key value argument to compilation command|\\
  two arguments, where the first argument is the key, but this time the second argument is the path of |filename| relative to the cache folder (useful since scripts run from this folder). Moreover, it automatically ensures that when |filename| changes, the file gets recompiled. Note that contrary to some other commands, this does not copy the file in the cache, which is practical notably for large files like videos.
\end{pgfmanualentry}

\begin{pgfmanualentry}
  \makeatletter%
  \def\extrakeytext{style, }
  \extractkey/robExt/force compilation\@nil
  \extractkey/robExt/do not force compilation\@nil
  \makeatother%
  \pgfmanualbody
  By default, we compile cached documents only if |-shell-escape| is enabled. However, if the user allowed |pdflatex| (needed to compile latex documents), |cd| (not needed when using |no cache folder|), and |mkdir| (not needed when using |no cache folder| or if the cache folder that defaults to |robustExternalize| is already created) to run in restricted mode (so without enabling |-shell-escape|), then there is no need for |-shell-escape|. In that case, set |force compilation| and this library will compile even if |-shell-escape| is disabled.
\end{pgfmanualentry}

\begin{pgfmanualentry}
  \makeatletter%
  \def\extrakeytext{style, }
  \extractkey/robExt/recompile\@nil
  \extractkey/robExt/do not recompile\@nil
  \makeatother%
  \pgfmanualbody
  (new to v2.1) |recompile| will force a compilation even if the image is already cached (mostly useful if you made a change to a non-tracked dependency). Note that since this library does not want to remove any file on the hard drive (we do not want to risk to remove important files in case of a bug), this command will NOT clear the auxiliary files already present, so for maximum purity, ensure that your compiling command is idempotent (so that compiling twice gives the same outcome as compiling once).

\begin{codeexample}[vbox]
\cacheEnvironment{tikzcd}{tikz, add to preamble={\usepackage{tikz-cd}}}
\begin{tikzcd}<recompile>
  A \rar & B
\end{tikzcd}    
\end{codeexample}
  
If you want to actually remove the aux files, you can either change the compilation command, or add something like this to remove the file (note that it will prompt a message before running the actual command), using the hook |robust-externalize/just-before-writing-files| that we introduced in v2.1:
\begin{codeexample}[code only]
\cacheTikz

\robExtConfigure{
  clean and recompile/.code={%
    \AddToHook{robust-externalize/just-before-writing-files}{%
      \edef\command{rm -f "\robExtAddCachePathAndName{\robExtFinalHash.pdf}" && %
        rm -f "\robExtAddCachePathAndName{\robExtFinalHash.aux}"}%
      \message{You will run the command ``\command''}%
      \message{Press X if you do NOT want to run it, otherwise press ENTER.}%
      \show\def% this is just used to wait a user input
      \immediate\write18{\command}%
    }%
  },%
}

\begin{tikzpicture}<clean and recompile>
  \node[fill=red] {B};
\end{tikzpicture}
\end{codeexample}
Note that if you do not want to run a command from \LaTeX{}, or if you are on a version smaller than 2.1, you can also simply print the name of the file in the pdf and remove it manually:

\begin{codeexample}[width=0cm,vbox]
\begin{CacheMe}{tikzpicture, name output=filetodelete}[baseline=(A.base)]
  \node[draw,rounded corners,fill=pink!60](A){Hello World!};
\end{CacheMe}\\

You want to delete the file \texttt{\filetodeleteInCache.pdf}. %
\end{codeexample}

\end{pgfmanualentry}


\subsubsection{Options to configure the inclusion command}

The inclusion command is the command that is run to include the cached file back in the pdf (e.g. based on |\includegraphics|). We describe now how to configure this command.

\begin{pgfmanualentry}
  \makeatletter%
  \def\extrakeytext{style, }
  \extractkey/robExt/custom include command advanced=\marg{include command}\@nil
  \makeatother%
  \pgfmanualbody
  Sets the command to run to include the compiled file. You can use:\\
  |\robExtAddCachePathAndName{\robExtFinalHash.pdf}|\\
  in order to get the path of the compiled pdf file. Note that we recommend rather to use |custom include command| that automatically checks if the file compiled correctly and that load the |*-out.tex| file if it exists (useful to pass information back to the pdf).
\end{pgfmanualentry}

\begin{pgfmanualentry}
  \makeatletter%
  \def\extrakeytext{style, }
  \extractkey/robExt/custom include command=\marg{include command}\@nil
  \makeatother%
  \pgfmanualbody
  Sets the command to run to include the compiled file, after checking if the file has been correctly compiled and loading |*-out.tex| (useful to pass information back to the pdf).
\end{pgfmanualentry}

\begin{pgfmanualentry}
  \makeatletter%
  \def\extrakeytext{style, }
  \extractkey/robExt/do not include pdf\@nil
  \makeatother%
  \pgfmanualbody
  Do not include the pdf. Useful if you only want to compile the file but use it later (note that you should still generate a |.pdf| file, possibly empty, to indicate that the compilation runs smootly). Equivalent to:\\
  |custom include command={}|
\end{pgfmanualentry}

\begin{pgfmanualentry}
  \makeatletter%
  \def\extrakeytext{style, }
  \extractkey/robExt/enable manual mode\@nil
  \extractkey/robExt/disable manual mode\@nil
  \extractkey/robExt/enable fallback to manual mode\@nil
  \extractkey/robExt/disable fallback to manual mode\@nil
  \makeatother%
  \pgfmanualbody
  If you do or do not want to ask latex to run the compilation commands itself (for instance for security
  reasons), you can use these commands and run the command manually later:
  \begin{codeexample}[width=0pt,vbox]
    \robExtConfigure{
      enable manual mode
    }

    The next picture must be manually compiled %
    (see JOBNAME-robExt-compile-missing-figures.sh):\\ %
    \begin{tikzpictureC}[baseline=(A.base)]
      \node[fill=red, rounded corners](A){I must be manually compiled.};
      \node[fill=red, rounded corners, opacity=.3,overlay] at (A.north east){I am an overlay text};
    \end{tikzpictureC}
  \end{codeexample}
  The main difference between |manual mode| and |fallback to manual mode| is that |fallback to manual mode| will try to compile the file if |-shell-escape| is enabled, while |manual mode| will never run any command, even if |-shell-escape| is enabled. Note that |enable fallback to manual mode| is available starting from v2.0.
  
  See \cref{sec:operationsCache} for more details.
\end{pgfmanualentry}

\begin{pgfmanualentry}
  \makeatletter%
  \def\extrakeytext{style, }
  \extractkey/robExt/include graphics args\@nil
  \makeatother%
  \pgfmanualbody
  By default, the include commands runs |\includegraphics| on the pdf, and possibly raises it if needed. You can customize the arguments passed to |\includegraphics| here.
\end{pgfmanualentry}

\subsubsection{Configuration of the cache}

If needed, you can configure the cache:

\begin{pgfmanualentry}
  \makeatletter%
  \def\extrakeytext{style, }
  \extractkey/robExt/set filename prefix=\marg{prefix}\@nil
  \makeatother%
  \pgfmanualbody
  By default, the files in the cache starts with |robExt-|. If needed you can change this here, or by manually defining |\def\robExtPrefixFilename{yourPrefix-}|.
\end{pgfmanualentry}

\begin{pgfmanualentry}
  \makeatletter%
  \def\extrakeytext{style, }
  \extractkey/robExt/set subfolder and way back=\marg{cache folder}\marg{path to project from cache}\@nil
  \extractkey/robExt/set cache folder and way back=\marg{cache folder}\marg{path to project from cache}\@nil
  \makeatother%
  \pgfmanualbody
  By default, the cache is located in |robustExternalize/|, using:\\
  |set cache folder and way back={robustExternalize/}{../},|\\
  You can customize it the way you want, just be make sure that going to the second arguments after going to the first argument leads you back to the original position, and make sure to terminate the path with a |/| so that |__ROBEXT_WAY_BACK__common_inputs.tex| gives the path of the file |common_inputs.tex| in the root folder (do not write |__ROBEXT_WAY_BACK__/common_inputs.tex| as this would expand to the absolute path |/common_inputs.tex| if you disable the cache folder). Note that both:\\
  |set cache folder and way back|\\
  and\\
  |set subfolder and way back|\\
  are alias, but the first one was introduced in v2.0 as it is clearer, the second one being kept for backward compatibility.
\end{pgfmanualentry}

\begin{pgfmanualentry}
  \makeatletter%
  \def\extrakeytext{style, }
  \extractkey/robExt/no cache folder\@nil
  \makeatother%
  \pgfmanualbody
  Do not put the cache in a separate subfolder.
\end{pgfmanualentry}


\newpage
\subsubsection{Customize or disable externalization}\label{sec:disableExternalization}

You might want (sometimes or always) to disable externalization, for instance to use |remember picture| \tikz[remember picture,baseline=(pointtome.base)] \node[rounded corners, fill=orange](pointtome){Point to me if you can};, even if you used |\cacheTikz|:

\begin{pgfmanualentry}
  \makeatletter%
  \def\extrakeytext{style, }
  \extractkey/robExt/disable externalization\@nil
  \extractkey/robExt/de\@nil
  \extractkey/robExt/disable externalization now\@nil
  \extractkey/robExt/enable externalization\@nil
  \makeatother%
  \pgfmanualbody
  Enable or disable externalization.
  \begin{codeexample}[width=0pt,vbox]
    % In theory all pictures should be externalized (so remember picture should fail)
    \cacheTikz
    % But we can disable it temporarily
    \begin{tikzpicture}<disable externalization>[remember picture]
      \node[rounded corners, fill=red](A){This figure is not externalized.
        This way, it can use remember picture.};
      \draw[->,overlay] (A) to[bend right] (pointtome);
    \end{tikzpicture}\\

    % You can also disable it globally/in a group:
    {
      \robExtConfigure{disable externalization}
      
      \begin{tikzpicture}[remember picture]
        \node[rounded corners, fill=red](A){This figure is not externalized.
          This way, it can use remember picture.};
        \draw[->,overlay] (A.west) to[bend left] (pointtome);
      \end{tikzpicture}\\

      \begin{tikzpicture}[remember picture]
        \node[rounded corners, fill=red](A){This figure is not externalized.
          This way, it can use remember picture.};
        \draw[->,overlay] (A.east) to[bend right] (pointtome);
      \end{tikzpicture}
    }
  \end{codeexample}
  |disable externalization now| additionally redefines all automatically cached commands and environments to their default value right now (|disable externalization| would only do so when running the |command if no externalization| is run, which should be preferred if possible). This is mostly useful when an automatically cached command cannot be parsed (e.g.\ you specified the signature |O{}m| but in fact the command expects a more complicated parsing algorithm. For instance, in |tikz| we can omit the brackets, and it might confuse the system if we defined it as:
  \begin{codeexample}[width=0pt,vbox]
    %% Not the recommended way to proceed, but this is for the example
    \cacheCommand{tikz}[O{}m]{tikz}
    {
      % Needed, or the parser would be confused by the next command
      \robExtConfigure{disable externalization now}
      \tikz\node[fill=green]{The short version of tikz can be confusing.};
    }
  \end{codeexample}
  Since v2.1 you can also automatically disable externalization if the input contains a given string, for instance |remember picture| (that is not cachable with this library):
\begin{codeexample}[vbox]
\cacheTikz
\robExtConfigure{
  add to preset={tikzpicture}{
    % \b means "word boundary", and spaces must be escaped
    if matches={remember picture}{disable externalization},
  },
}
\begin{tikzpicture}[remember picture]
  \node[fill=green](my original node){Point to me};
\end{tikzpicture} %
Some text %
\begin{tikzpicture}[overlay, remember picture]
  \draw[->] (0,0) to[bend left] (my original node);
\end{tikzpicture}
\end{codeexample}  
\end{pgfmanualentry}

\begin{pgfmanualentry}
  \makeatletter%
  \def\extrakeytext{style, }
  \extractkey/robExt/command if no externalization\@nil
  \extractkey/robExt/command if externalization\@nil
  \makeatother%
  \pgfmanualbody
  You can easily change the command to run if externalization is disabled using by setting the \textbf{.code} of this key. By default, it is configured as:\\
\begin{verbatim}
command if no externalization/.code={%
  \robExtDisableTikzpictureOverwrite\evalPlaceholder{__ROBEXT_MAIN_CONTENT__}%
}
\end{verbatim}
  Unless you know what you are doing, you should include |\robExtDisableTikzpictureOverwrite| as it is often necessary to avoid infinite recursion when externalization is disabled and the original command has been replaced with a cached version (for instance done by |\cacheTikz|). Note that if you write your own style, try to modify |__ROBEXT_MAIN_CONTENT__| so that it can be included as-it in a document: this way you do not need to change this command.
  You can also use |command if externalization/.append code={...}| if you want to run a code only if externalization is enabled (this is used for instance by the |forward| functionality).
\end{pgfmanualentry}

\begin{pgfmanualentry}
  \makeatletter%
  \def\extrakeytext{style, }
  \extractkey/robExt/print verbatim if no externalization\@nil
  \makeatother%
  \pgfmanualbody
  Sets |command if no externalization| to print the verbatim content of |__ROBEXT_MAIN_CONTENT__| if externalization is disabled. Internally, it just sets it to:\\
  |\printPlaceholder{__ROBEXT_MAIN_CONTENT__}|\\
  This is mostly useful when typesetting |__ROBEXT_MAIN_CONTENT__| directly does not make sense (e.g. in python code). This style is used for instance in the |python| preset, allowing us to get:
  
\begin{codeAndResult}
\begin{CacheMeCode}{python,
    verbatim output,
    set placeholder eval={__thepage__}{\thepage},
    %% We disable externalization
    disable externalization}
with open("__ROBEXT_OUTPUT_PREFIX__-out.txt", "w") as f:
    for i in range(5):
        f.write(f"Hello {i}, we are on page __thepage__\n")
\end{CacheMeCode}
\end{codeAndResult}

You can also disable the externalization on all elements that use a common preset, for instance you can disable externalization on all |bash| instances (useful if you are on Windows for instance):
\begin{codeAndResult}
\robExtConfigure{
  % bash code will not be compiled (useful on windows for instance)
  add to preset={bash}{
    disable externalization
  },
}
\begin{CacheMeCode}{bash, verbatim output}
# $outputTxt contains the path of the file that will be printed via \verbatiminput
uname -srv > "${outputTxt}"
\end{CacheMeCode}
\end{codeAndResult}

\end{pgfmanualentry}

\begin{pgfmanualentry}
  \makeatletter%
  \def\extrakeytext{style, }
  \extractkey/robExt/execute after each externalization\@nil
  \extractkey/robExt/execute before each externalization\@nil
  \makeatother%
  \pgfmanualbody
  By doing |execute after each externalization={some code}|, you will run some code after the externalization. This might be practical for instance to update a counter (e.g. the number of pages\dots) based on the result of the compiled file.
\end{pgfmanualentry}

\subsubsection{Dependencies}\label{sec:dependenciesDoc}

In order to enforce reproducibility, you should tell what are the files that your code depends on, by adding this file as a dependency. This has the advantage that if this file is changed, your code is automatically recompiled. On the other hand, you might not want this behavior (e.g. if this file often changes in a non-important way): in that case, just don't add the file as a dependency (but keep that in mind as you might not be able to recompile your file if you clear the cache if you introduced breaking changes).

\begin{pgfmanualentry}
  \makeatletter%
  \def\extrakeytext{style, }
  \extractkey/robExt/dependencies=\marg{list,of,dependencies}\@nil
  \extractkey/robExt/add dependencies=\marg{list,of,dependencies}\@nil
  \extractkey/robExt/reset dependencies\@nil
  \makeatother%
  \pgfmanualbody
  Set/add/reset the dependencies (you can put multiple files separated by commas). These files should be relative to the main compiled file. For instance, if you have a file |common_inputs.tex| that you want to input in both the main file and in the cached files, that contains, say:\\
  |\def\myValueDefinedInCommonInputs{42}|\\
  then you can add it as a dependency using:
\begin{codeexample}[width=0pt,vbox]
\begin{CacheMe}{latex,
    add dependencies={common_inputs.tex},
    add to preamble={\input{__ROBEXT_WAY_BACK__common_inputs.tex}}}
  The answer is \myValueDefinedInCommonInputs.
\end{CacheMe}
\end{codeexample}
Note that the placeholder |__ROBEXT_WAY_BACK__| contains the path from the cache folder (containing the |.tex| that will be cached) to the root folder.\\ This way, you can easily input files contained in the root folder.
\end{pgfmanualentry}

\subsubsection{Forward macros}\label{sec:forward}

(since v2.1)

You can also see the tutorial (\cref{sec:forwardIntro}) for some examples. So far, we have seen multiple approaches to use a macro in both the main document and in cached pictures, having different advantages:
\begin{itemize}
\item if we define the macro in a file say |common_inputs.tex|, if we include that file in both files \textbf{and} if we add it to the list of dependencies (\cref{sec:dependenciesDoc}), then you will have great purity (if |common_inputs.tex| changes, all files are recompiled), but this might be a problem if you often change the file |common_inputs.tex| since you will need to recompile all pictures every time you add a new macro
\item If you follow the same procedure, but \textbf{without} adding the file in the list of dependencies, then you will save compilation time (no need to recompile everything if you add a macro in |common_inputs.tex|), but you will lose purity (if you change the definition of a macro in that file, you will need to manually specify the list of figures to recompile, e.g. using |recompile|).
\item You can also create multiple files like
\begin{verbatim}
common_inputs_functionality_A.tex
\end{verbatim}
  and
\begin{verbatim}
common_inputs_functionality_B.tex
\end{verbatim}
  and use the first solution (maybe by creating a preset that automatically calls |add to preamble| and |add dependencies|), and only input the appropriate file depending on the set of macro that you need. This gives a tradeoff of above approaches (good purity, with less recompilation), but you will still compile often if you put many macros in a single file (and nobody wants to create one macro per file!).
\end{itemize}

In this section, we provide a third solution that tries to solve the above issue (purity without frequent recompilations), by allowing the user to forward only a specific macro, with something like |forward=\myMacro|. In order to avoid to manually forward all macros used for each picture, we also provide some functions to automatically detect the macros to forward.

\begin{pgfmanualentry}
  \makeatletter%
  \def\extrakeytext{style, }
  \extractkey/robExt/forward=macro to forward\@nil
  \extractkey/robExt/fw=macro to forward\@nil
  \extractkey/robExt/forward at letter=macro to forward\@nil
  \makeatother%
  \pgfmanualbody
  |forward| (alias |fw|) will forward the definition of a macro to the picture if externalization is enabled:
\begin{codeexample}[width=0pt]
\cacheTikz
\def\myName{Alice}
\begin{tikzpicture}<forward=\myName>
  \node[fill=red]{\myName};
\end{tikzpicture}
\end{codeexample}
Note that it works for macro defined with |\def|, |\newcommand| (or alike) and |\NewDocumentCommand| (or other xparse commands). For instance with xparse:
\begin{codeexample}[width=0pt]
\cacheTikz
\NewDocumentCommand{\MyNode}{O{}m}{\node[rounded corners,fill=red,#1]{#2};}
\begin{tikzpicture}<forward=\MyNode>
  \MyNode{Alice}
  \MyNode[xshift=2cm]{Bob}
\end{tikzpicture}
\end{codeexample}
and with |\newcommand|:
\begin{codeexample}[width=0pt]
\cacheTikz
\newcommand{\MyNodeWithNewCommand}[2][draw,thick]{\node[rounded corners,fill=red,#1]{#2};}
\begin{tikzpicture}<forward=\MyNodeWithNewCommand>
  \MyNodeWithNewCommand{Alice}
  \MyNodeWithNewCommand[xshift=2cm]{Bob}
\end{tikzpicture}
\end{codeexample}
Note that this might not work with quite involved macros. Notably, if the macro is defined by calling other macros, you need to also forward these macros.
\begin{codeexample}[width=0pt]
\cacheTikz
\def\myName{Alice}
\NewDocumentCommand{\MyNodeWithInnerMacros}{O{}m}{\node[rounded corners,fill=red,#1]{\myName: #2};}
\begin{tikzpicture}<forward=\MyNodeWithInnerMacros,forward=\myName>
  \MyNodeWithInnerMacros{she is tall}
\end{tikzpicture}
\end{codeexample}
Really involved macros (e.g.\ defined inside libraries) might call many other macros, possibly using the @ letter (used only in module code). If you REALLY want, you can forward such macros if you forward \textbf{all} macros that are used internally, using |forward at letter| that will automatically wrap the macro inside |\makeatletter...\makeatother|, but you should certainly use |add to preamble={\usepackage{yourlib}}| if you need to do something like that:
\begin{codeexample}[width=0pt]
\cacheTikz
\makeatletter
\def\module@name{Alice}
\newcommand{\MyNodeWithAtCommands}[2][draw,thick]{\node[rounded corners,fill=red,#1]{\module@name};}
\begin{tikzpicture}<forward at letter=\MyNodeWithAtCommands,forward at letter=\module@name>
  \MyNodeWithAtCommands{Alice}
\end{tikzpicture}
\makeatother
\end{codeexample}
\end{pgfmanualentry}

\begin{pgfmanualentry}
  \makeatletter%
  \def\extrakeytext{style, }
  \extractkey/robExt/forward eval=macro to eval and forward\@nil
  \makeatother%
  \pgfmanualbody
  Evaluate a macro, and defines it using this new value when compiling the function. This might be useful for instance if your macro depends on other macros that you do not want to export:
\begin{codeexample}[width=0pt]
\def\name{Alice}
\def\fullName{\name}
\begin{tikzpictureC}<forward eval=\fullName>
  \node[rounded corners, fill=red]{\fullName.};
\end{tikzpictureC}
\end{codeexample}
\end{pgfmanualentry}

\begin{pgfmanualentry}
  \makeatletter%
  \def\extrakeytext{style, }
  \extractkey/robExt/forward counter=counter to forward\@nil
  \makeatother%
  \pgfmanualbody
  Forward a counter:
\begin{codeexample}[width=0pt]
\begin{tikzpictureC}<forward counter=page>
  \node[rounded corners, fill=red]{The current page is \thepage.};
\end{tikzpictureC}    
\end{codeexample}
\end{pgfmanualentry}

\begin{pgfmanualentry}
  \makeatletter%
  \def\extrakeytext{style, }
  \extractkey/robExt/forward color=color to forward\@nil
  \extractkey/robExt/auto forward color=\marg{preset}\marg{color to forward}\@nil
  \makeatother%
  \pgfmanualbody
  |forward color| forwards a color (defined with xcolor). It works with colors defined with |\definecolor|:
\begin{codeexample}[width=0pt]
\definecolor{myred}{HTML}{F01122}
\begin{tikzpictureC}<forward color=myred>
  \node[fill=myred]{A};
\end{tikzpictureC}
\end{codeexample}
but also with colors defined with |\colorlet|:
\begin{codeexample}[width=0pt]
\definecolor{myred}{HTML}{F01122}
\colorlet{myviolet}{blue!50!myred}
\begin{tikzpictureC}<forward color=myviolet>
  \node[fill=myviolet,yshift=-1cm]{B};
\end{tikzpictureC}
\end{codeexample}
Note that using |if matches|, you can also automatically forward colors:
\begin{codeexample}[width=0pt]
\definecolor{myred}{HTML}{F01122}
\colorlet{myviolet}{blue!50!myred}
\colorlet{mypink}{pink!90!orange}
\robExtConfigure{
  add to preset={tikz}{
    if matches={mypink}{forward color=mypink},
  },
}
\begin{tikzpictureC}
  \node[fill=mypink,yshift=-1cm]{B};
\end{tikzpictureC}
\end{codeexample}
but since this is a bit long to type, we provide a shortcut:
\begin{codeexample}[width=0pt]
\definecolor{myred}{HTML}{F01122}
\colorlet{myviolet}{blue!50!myred}
\colorlet{mypink}{pink!90!orange}
\robExtConfigure{
  auto forward color={tikz}{mypink},
}
\begin{tikzpictureC}
  \node[fill=mypink,yshift=-1cm]{B};
\end{tikzpictureC}
\end{codeexample}
\end{pgfmanualentry}

\begin{pgfmanualentry}
  \makeatletter%
  \def\extrakeytext{style, }
  \extractkey/robExt/auto forward\@nil
  \makeatother%
  \pgfmanualbody
  Enable automatic forwarding of macros marked with |*AutoForward| commands or |\configIfMacroPresent|. This command naively extracts all used macro via a simple regex match and forwards them/execute the style configured in |\configIfMacroPresent| if it exists.
\begin{codeexample}[vbox]
\cacheTikz  
\robExtConfigure{add to preset={tikz}{auto forward}}

\newcommandAutoForward{\MyNode}[2][draw,thick]{\node[rounded corners,fill=red,#1]{#2};}
\newcommandAutoForward{\MyGreenNode}[2][draw,thick]{\node[rounded corners,fill=green,#1]{#2};}

\begin{tikzpicture}
  \MyNode{Recompiled only if MyNode is changed}
  \MyNode[xshift=8cm]{but not if the (unused) MyGreenNode is changed.}
\end{tikzpicture}\\

\begin{tikzpicture}
  \MyGreenNode{Recompiled only if MyGreenNode is changed}
  \MyGreenNode[xshift=8cm]{but not if the (unused) MyNode is changed.}
\end{tikzpicture}
\end{codeexample}
\end{pgfmanualentry}

\begin{pgfmanualentry}
  \extractcommand\configIfMacroPresent\marg{macro}\marg{style to run if macro is present}\@@
  \pgfmanualbody
  Runs the corresponding style if the macro |macro| is present (make sure to enable |auto forward|).
\begin{codeexample}[vbox]
\cacheTikz  
\robExtConfigure{add to preset={tikz}{auto forward}}

\configIfMacroPresent{\ding}{add to preamble={\usepackage{pifont}}}

\begin{tikzpicture}
  \node[fill=green, circle]{\ding{164}};
\end{tikzpicture}  
\end{codeexample}
\end{pgfmanualentry}

\begin{pgfmanualentry}
  \extractcommand\newcommandAutoForward\marg{macro}\opt{\oarg{nb args}}\opt{\oarg{optional default value}}\marg{definition}\opt{\oarg{additional style}}\@@
  \extractcommand\renewcommandAutoForward\marg{macro}\opt{\oarg{nb args}}\opt{\oarg{optional default value}}\marg{definition}\opt{\oarg{additional style}}\@@
  \extractcommand\providecommandAutoForward\marg{macro}\opt{\oarg{nb args}}\opt{\oarg{optional default value}}\marg{definition}\opt{\oarg{additional style}}\@@
  \extractcommand\NewDocumentCommandAutoForward\marg{macro}\marg{arg spec}\opt{\oarg{additional style}}\marg{definition}\@@
  \extractcommand\RenewDocumentCommandAutoForward\marg{macro}\marg{arg spec}\opt{\oarg{additional style}}\marg{definition}\@@
  \extractcommand\ProvideDocumentCommandAutoForward\marg{macro}\marg{arg spec}\opt{\oarg{additional style}}\marg{definition}\@@
  \extractcommand\DeclareDocumentCommandAutoForward\marg{macro}\marg{arg spec}\opt{\oarg{additional style}}\marg{definition}\@@
  \extractcommand\defAutoForward\marg{macro}\opt{\oarg{def-style arg spec}}\marg{macro}\opt{\oarg{additional style}}\@@
  \pgfmanualbody
\begin{verbatim}
\newcommandAutoForward{\macro}[nb args][default value]{def}[STYLE]
\end{verbatim}
  is like
\begin{verbatim}
\newcommand{\macro}[nb args][default value]{def}
\end{verbatim}
  but it additionally runs
\begin{verbatim}
\configIfMacroPresent{\macro}{forward=\macro,STYLE}
\end{verbatim}
  to automatically forward the macro if it is used. This works similarly for other commands (note that for |\defAutoForward|, you need to put the arguments in a bracket, like |\defAutoForward\myMacro[#1 and #2]{Hey #1, #2}| for |\def\myMacro#1 and #2{Hey #1, #2}|).
\end{pgfmanualentry}

\begin{pgfmanualentry}
  \makeatletter%
  \def\extrakeytext{style, }
  \extractkey/robExt/load auto forward macro config=\marg{macro}\@nil
  \makeatother%
  \pgfmanualbody
  Like |forward| except that it loads the configuration that would have been loaded if the macro was present in the file. This is mostly useful to say that a macro depends on the style of another macro without copy/pasting the style of the second macro inside the first one.
\begin{codeexample}[vbox]
\cacheTikz  
\robExtConfigure{add to preset={tikz}{auto forward}}

\newcommandAutoForward{\MyName}{\ding{164} Alice}[add to preamble={\usepackage{pifont}}]
\newcommandAutoForward{\MyNode}[2][draw,thick]{\node[rounded corners,fill=green,#1]{\MyName: #2};}[
  load auto forward macro config=\MyName
]

\begin{tikzpicture}
  \MyNode{Recompiled only if MyNode is changed}
  \MyNode[yshift=-1cm]{or if MyName is changed}
\end{tikzpicture}
\end{codeexample}
\end{pgfmanualentry}

\begin{pgfmanualentry}
  \makeatletter%
  \def\extrakeytext{style, }
  \extractkey/robExt/if matches=\marg{string}\marg{style to apply}\@nil
  \extractkey/robExt/if matches regex=\marg{latex3 regex}\marg{style to apply}\@nil
  \makeatother%
  \pgfmanualbody
  Apply the corresponding style if the string (resp.\ regex in \LaTeX{}3 format) matches is matched in the content. The regex version can be more expressive, but is wayyy slower.
\begin{codeexample}[vbox]
\robExtConfigure{
  add to preset={my python}{
    python print code and result,
    % \b is for word border, and ( needs to be escaped in regex
    if matches={cos(}{add import={from math import cos}},
    if matches={sin(}{add import={from math import sin}},
  },
}
\begin{CacheMeCode}{my python}
print(cos(1)+sin(2))
\end{CacheMeCode}
\end{codeexample}
With regex:
\begin{codeexample}[vbox]
\robExtConfigure{
  add to preset={my python}{
    python print code and result,
    % \b is for word border, and ( needs to be escaped in regex
    if matches regex={\b cos\(}{add import={from math import cos}},
    if matches regex={\b sin\(}{add import={from math import sin}},
  },
}
\begin{CacheMeCode}{my python}
print(cos(1)+sin(2))
\end{CacheMeCode}
\end{codeexample}
This can also be practical to disable caching if some pictures uses |remember picture| (which is not supported by this library):
\begin{codeexample}[vbox]
\cacheTikz
\robExtConfigure{
  add to preset={tikzpicture}{
    % \b means "word boundary", and spaces must be escaped
    if matches={remember picture}{disable externalization},
  },
}
\begin{tikzpicture}[remember picture]
  \node[fill=green](my original node){Point to me};
\end{tikzpicture} %
Some text %
\begin{tikzpicture}[overlay, remember picture]
  \draw[->] (0,0) to[bend left] (my original node);
\end{tikzpicture}
\end{codeexample}
\end{pgfmanualentry}

\subsubsection{Pass compiled file to another template}\label{sec:passOutputToOtherTemplate}

It can sometimes be handy to use the result of a previous cached file to cache another file, or to do anything else (e.g.\ it can also be practical to debug an issue). |name output| can be used to do that
\begin{pgfmanualentry}
  \makeatletter%
  \def\extrakeytext{style, }
  \extractkey/robExt/name output=\marg{macro name}\@nil
  \makeatother%
  \pgfmanualbody
  |name output=foo| will create two global macros |\foo| and |\fooInCache|: |\foo| expands to the prefix of the files created in the template like |robExt-somehash|, and |\fooInCache| also adds the cache folder like |robustExternalize/robExt-somehash|. You can then use |set placeholder eval| to send it to another cached file. It is then your role to add the extension, usually |.tex| to get the source (even if the source is a python file), |.pdf| to get the pdf, |-out.tex| to get the file that is loaded before the import, |-out.txt| if you wanted to make it compatible with |verbatim output| (this list is not exhaustive as each script might decide to create a different output file). Here is a demo:
  
\begin{codeexample}[width=0cm,vbox]
\begin{CacheMe}{tikzpicture, do not add margins, name output=mycode}[baseline=(A.base)]
  \node[draw,rounded corners,fill=pink!60](A){Hello World!};
\end{CacheMe}\\[3mm]

The prefix is \texttt{\mycode} and with the cache folder it is in:\\
\texttt{\mycodeInCache}.\\
It this can be helpful for instance to debug, as you can inspect the source:
\verbatiminput{\mycodeInCache.tex}
but it is also practical to define a template based on the previously cached files:\\

\begin{CacheMe}{tikzpicture, set placeholder eval={__previous__}{\mycode.pdf}}
  \node[rounded corners, fill=green!50]{A cached file can use result from another cached file:
    \includegraphics[width=2cm]{__previous__}\includegraphics[width=2cm]{__previous__}};
\end{CacheMe}  
\end{codeexample}

Note that if you do not want to display the first cached file, you can use |do not include pdf| to hide it.

\end{pgfmanualentry}

\subsubsection{Compile in parallel}\label{sec:compileParallel}

(introduced in v2.1, read below on how to do it manually on v2.0)

\begin{pgfmanualentry}
  \makeatletter%
  \def\extrakeytext{style, }
  \extractkey/robExt/compile in parallel=\opt{nb of pictures to compile normally} (default 0)\@nil
  \extractkey/robExt/compile in parallel after=\opt{nb pictures to compile normally} (default 0)\@nil
  \makeatother%
  \pgfmanualbody
  These two commands are alias. Typing in the preamble:
\begin{codeexample}[code only]
\robExtConfigure{
  compile in parallel
}
\end{codeexample}
will cause the cached elements to be compiled in parallel (this requires two compilations of the main project). For this to work out of the box, you need to have GNU |parallel| installed (on linux or Mac) or rush (\url{https://github.com/shenwei356/rush/}) on Windows. Typing |compile in parallel after=3| will start the compilation in parallel only if you have more than 3 new elements to compile, and compile the first 3 elements normally (useful when you work on one picture at a time since you do not need to compile twice the document).
\end{pgfmanualentry}

\begin{pgfmanualentry}
  \makeatletter%
  \def\extrakeytext{style, }
  \extractkey/robExt/compile in parallel command=\marg{command to run}\@nil
  \extractkey/robExt/compile in parallel with xargs=\opt{number threads} (default 16\%)\@nil
  \extractkey/robExt/compile in parallel with gnu parallel=\opt{--jobs option} (default 200\%)\@nil
  \extractkey/robExt/compile in parallel with rush=\opt{additional options} (default )\@nil
  \makeatother%
  \pgfmanualbody
  You can customize the compilation command to run via something like:
\begin{codeexample}[code only]
\robExtConfigure{
  compile in parallel command={
    parallel --jobs 200\% :::: '\jobname-\robExtAddPrefixName{compile-missing-figures.sh}
  },
}
\end{codeexample}
The other styles are predefined commands to compile with xargs (by default, already installed in linux, on Windows you can get it by installing the lightweight GNU On Windows (Gow) \url{https://github.com/bmatzelle/gow}), gnu parallel, or rush (\url{https://github.com/shenwei356/rush/}). You can use these commands directly like:
\begin{codeexample}[code only]
\robExtConfigure{
  compile in parallel with gnu parallel
}
\end{codeexample}
or use the optional parameter to configure the number of threads (xargs starts with 16 threads by default, and gnu parallel takes twice the number of CPU threads available), like in:
\begin{codeexample}[code only]
\robExtConfigure{
  compile in parallel with xargs=64,% Compiles with 64 processes at a time
}
\end{codeexample}
\end{pgfmanualentry}

NB: in version smaller than v2.1 (or if you prefer the manual method), you can instead run it in manual mode and use for instance GNU |parallel| (or any tool of you choice) to run all commands in the file |\jobname-robExt-compile-missing-figures.sh| in parallel. To do that:
\begin{itemize}
\item First add |\robExtConfigure{enable manual mode}| in your file
\item Create the cache folder, e.g.\ using |mkdir robustExternalize|
\item Compile your file, with, e.g.\ |pdflatex yourfile.tex| (in my benchmark: 4.2s)
\item This should create a file |yourfile-robExt-compile-missing-figures.sh|, with each line containing a compilation command like this (the command should be compatible with Windows, Mac and Linux, if not, let me know on github):
\begin{verbatim}
cd robustExternalize/ && pdflatex -halt-on-error "robExt-4A806941E86C6B657A0CB3D160CFFF3E.tex
\end{verbatim}
  \textbf{IMPORTANT}: \LaTeX{} might render \textbf{multiple times} the same picture if it is inside some particular environments (align, tables\dots), so this file can contain duplicates if you inserted cached data inside. In version 2.1 this is solved, but on older versions, to avoid running the same command twice, \textbf{be sure to remove duplicates!} On Linux, this can be done by piping the file into |sort| and |uniq| (if your files can depend on previously compiled elements as explained in \cref{sec:passOutputToOtherTemplate}, you should additionally preserve \mylink{https://unix.stackexchange.com/questions/194780/remove-duplicate-lines-while-keeping-the-order-of-the-lines}{the order of the lines}). Since this is anyway a rather advanced use case, we will not consider that case for simplicity.
\item Run all these command in parallel, for instance on Linux you can install |parallel|, and run the following command (|--bar| displays a progress bar):
\begin{verbatim}
cat yourfile-robExt-compile-missing-figures.sh | sort | uniq | parallel --bar --jobs 200%
\end{verbatim}
(note that since v2.1, it is not necessary to remove duplicates using |sort| and |uniq| since the file contains no more duplicates) In my benchmark, it ran during 52s, so 1.6x faster than the original compilation without caching)
\item Recompile the original document with |pdflatex yourfile.tex|
\end{itemize}
In my benchmark, the total time is 2x4.2s + 52s = 60s, so 1.5 times faster than a normal command (of course, this depends a lot on the preamble of the file that you compile, since the loading time of the file is the main bottleneck for the first compilation; the advantage here is that it is easy to include only the necessary things in the preamble of cached pictures, possibly creating different presets if it is easier to manage it this way).

\subsubsection{Compile a template to compile even faster}\label{sec:compileFaster}

Long story short: you can compile even faster (at least 1.5x in our tests, but we expect this to be more visible on larger tests due to the loading time of the preamble of the main document that takes most of the time of the compilation) by compiling presets, but beware that you will not be able to modify the placeholders except |add to preamble| with the default compiler we provide. The gain comes from the fact that instead of trying to find all placeholders to replace in the string, we can directly replace a few hard-coded placeholders to save time (yes, \LaTeX{} is not really efficient so it can make a difference\dots).


\begin{pgfmanualentry}
  \makeatletter%
  \def\extrakeytext{style, }
  \extractkey/robExt/new compiled preset=\marg{preset options to compile}\marg{runtime options}\@nil
  \extractkey/robExt/compile latex template\@nil
  \makeatother%
  \pgfmanualbody
  Compile a preset by creating a new placeholders that removes all placeholders except |__ROBEXT_MAIN_CONTENT__|, |__ROBEXT_MAIN_CONTENT_ORIG__| and |__ROBEXT_LATEX_PREAMBLE__|. |preset options to compile| should be sure that the template, compilation command and include command placeholders contain the final version: this can be easily done for latex presets by adding |compile latex template| at the end of |preset options to compile|:.
\begin{codeexample}[vbox]
 % We create a latex-based preset and compile it
\robExtConfigure{
  new preset={templateZX}{
    latex,
    add to preamble={
      \usepackage{tikz}
      \usepackage{tikz-cd}
      \usepackage{zx-calculus}
    },
    %% possibly add some dependencies
  },
  % We compile it into a new preset
  new compiled preset={compiled ZX}{templateZX, compile latex template}{},
}

% we use that preset automatically for ZX environments
\cacheEnvironment{ZX}{compiled ZX}
\cacheCommand{zx}{compiled ZX}

% Usage: (you can't use placeholders except for the preamble, trade-off of the compiled template)
\begin{ZX}<add to preamble={\def\sayHey#1{Hey #1!}}>
  \zxX{\sayHey{Bob}}
\end{ZX}
\end{codeexample}
\end{pgfmanualentry}


Explanations: in v2.0, we made substantial improvements in order to improve significantly the compilation time (our benchmark went from 20s (v1.0) to only 5.77s (v2.0)), but you might want to make this even faster: with a compiled preset, you can improve it further, experimentation showed an additional improvement factor of 1.4x (final compilation time was 4s in our tests, reminder: without the library it would take 1mn25s). The main bottleneck in term of time is the expansion of the placeholders (that allows great flexibility, but can add a significant time). At a high level, we keep track in a list of all declared placeholders, and loop over them to replace each of them until the string is left unmodified by a full turn. This is simple to implement, does not need to assume any shape for the placeholder\dots{} but not extremely efficient in \LaTeX{} where string and list manipulations are costly.

We can play on multiple parameters to speed up the process:
\begin{itemize}
\item Disable completely the placeholders and only replace a small amount of fixed placeholders in a fixed order and stop expanding the placeholder. This is what we do by default when we compile a preset: it is the quickest solution, but you cannot use arbitrary placeholders.
\item Start the expansion using a fixed (preset-dependent) list of placeholders, check if some templates are still present (assuming that templates must contain at least two consecutive underscores |__|), if not stop, otherwise go to the normal (less efficient) placeholder replacement. It is what we do for instance in the latex preset, and gives a great balance between efficiency and flexibility. Details can be found in \cref{sec:advancedOperations}.
\item Stop iterating over all placeholders, only choose the one meaningful (why would I care about python placeholders in latex?). For this, we introduce an import system, where you can create placeholders that are not added to the ``global'' list of placeholders, and import them where you want. In practice, you should not really need to use it unless you do more advanced stuff as we already take care of setting it properly, but details are in \cref{sec:importSystem}.
\end{itemize}
Another option could be to rewrite the code to search directly the list of placeholders in the string, but it is certainly not trivial to do in latex, would require even more constraints on the shape of placeholders, and would certainly still no more efficient than the compiled stuff. But anyway the already implemented solutions already give fairly good performances.

\subsection{Default presets}\label{sec:presets}

We provide by default some presets for famous languages (for now \LaTeX{} and python).

\subsubsection{All languages}

First, here are a few options that are available irrespective of the used language.

\begin{pgfmanualentry}
  \makeatletter%
  \def\extrakeytext{style, }
  \extractkey/robExt/set includegraphics options=\marg{options}\@nil
  \extractkey/robExt/add to includegraphics options=\marg{options}\@nil
  \makeatother%
  \pgfmanualbody
  Set/add options to the |\includegraphics| run when inserting the pdf (by the default include command). By default it is empty, but the |latex| preset sets it to:\\
  |trim=__ROBEXT_LATEX_TRIM_LENGTH__ __ROBEXT_LATEX_TRIM_LENGTH__|\\
  |__ROBEXT_LATEX_TRIM_LENGTH__ __ROBEXT_LATEX_TRIM_LENGTH__|\\
  in order to remove the margin added in the standalone package options, which is needed to display overlay texts.
\end{pgfmanualentry}

\begin{pgfmanualentry}
  \makeatletter%
  \def\extrakeytext{style, }
  \extractkey/robExt/verbatim output\@nil
  \makeatother%
  \pgfmanualbody
  Shortcut for:
\begin{verbatim}
custom include command={%
  \evalPlaceholder{%
    __ROBEXT_VERBATIM_COMMAND__{%
      __ROBEXT_CACHE_FOLDER____ROBEXT_OUTPUT_PREFIX__-out.txt}%
  }%
},
\end{verbatim}
  i.e.\ instead of printing the pdf we print the content of the file |__ROBEXT_OUTPUT_PREFIX__-out.txt| using the command in |__ROBEXT_VERBATIM_COMMAND__|, that defaults to |\verbatiminput|:
\begin{codeAndResult}
\begin{CacheMeCode}{python, verbatim output}
with open("__ROBEXT_OUTPUT_PREFIX__-out.txt", "w") as f:
    for i in range(5):
        f.write(f"Hello {i}\n")
\end{CacheMeCode}
\end{codeAndResult}
\end{pgfmanualentry}

\begin{pgfmanualentry}
  \makeatletter%
  \def\extrakeytext{style, }
  \extractkey/robExt/debug\@nil
  \extractkey/robExt/print command and source\@nil
  \makeatother%
  \pgfmanualbody
  (new in v2.0) These two (alias) commands allow you to print the compilation command, the dependency file (useful to be sure you tracked all dependencies), and the source file:
\begin{codeexample}[width=0pt,vbox]
\begin{CacheMe}{latex,
    add dependencies={common_inputs.tex},
    add to preamble={\input{__ROBEXT_WAY_BACK__common_inputs.tex}},
    debug
  }
  The answer is \myValueDefinedInCommonInputs.
\end{CacheMe}
\end{codeexample}

\end{pgfmanualentry}


\subsubsection{\LaTeX{} and \tikzname{}}

The |latex| preset is used to cache any \LaTeX{} content, like tikz pictures. Note that as of today, it supports overlay content out of the box (if the overlay is more than 30cm long, you might want to customize a placeholder), but not images that need to use |remember picture|.

\begin{pgfmanualentry}
  \makeatletter%
  \def\extrakeytext{style, }
  \extractkey/robExt/latex\@nil
  \makeatother%
  \pgfmanualbody
  This style sets the template |__ROBEXT_LATEX__| and the compilation command:\\ |__ROBEXT_LATEX_COMPILATION_COMMAND__|\\
  (cf \cref{sec:placeholders} for details), and adds a number of styles described below, to easily configure the most common options. You can use it as follows:
  \begin{codeexample}[width=0pt,vbox]
    The next picture is cached %
    \begin{CacheMe}{latex, add to preamble={\usepackage{tikz}}}
      \begin{tikzpicture}[baseline=(A.base)]
        \node[fill=red, rounded corners](A){My node that respects baseline.};
        \node[fill=red, rounded corners, opacity=.3,overlay] at (A.north east){I am an overlay text};
      \end{tikzpicture}
    \end{CacheMe} and you can see that overlay and depth works.
  \end{codeexample}
\end{pgfmanualentry}

To see how to create your own preset or automatically load a library, see \cref{sec:customize}.

The next options can be used after calling the |latex| style:

\begin{pgfmanualentry}
  \makeatletter%
  \def\extrakeytext{style, }
  \extractkey/robExt/latex/use latexmk\@nil
  \extractkey/robExt/latex/use lualatex\@nil
  \extractkey/robExt/latex/use xelatex\@nil
  \makeatother%
  \pgfmanualbody
  Use latexmk/lualatex/xelatex to compile. It is a shortcut for:\\
  |set placeholder={__ROBEXT_LATEX_ENGINE__}{yourfavoriteengine}|
\end{pgfmanualentry}

\begin{pgfmanualentry}
  \makeatletter%
  \def\extrakeytext{style, }
  \extractkey/robExt/latex/set latex options=\marg{latex options}\@nil
  \extractkey/robExt/latex/add to latex options=\marg{latex options}\@nil
  \makeatother%
  \pgfmanualbody
  Set/add elements to the set of latex options of the |\documentclass| (it will automatically add a comma before if you add an element). Internally it sets |__ROBEXT_LATEX_OPTIONS__|. By default, it sets:\\
  |margin=__ROBEXT_LATEX_TRIM_LENGTH__|
  (where |__ROBEXT_LATEX_TRIM_LENGTH__| is defined as 30cm by default) in order to add a margin that will be trimmed  later in the |\includegraphics|. This is useful not to cut stuff displayed outside of the bounding box (overlays). 
\end{pgfmanualentry}

\begin{pgfmanualentry}
  \makeatletter%
  \def\extrakeytext{style, }
  \extractkey/robExt/latex/set documentclass=\marg{documentclass}\@nil
  \makeatother%
  \pgfmanualbody
  Set the documentclass of the document (defaults to |standalone|). Internally, it sets the placeholder |__ROBEXT_LATEX_DOCUMENT_CLASS__|.
\end{pgfmanualentry}

\begin{pgfmanualentry}
  \makeatletter%
  \def\extrakeytext{style, }
  \extractkey/robExt/latex/set preamble=\marg{code of preamble}\@nil
  \extractkey/robExt/latex/add to preamble=\marg{code of preamble}\@nil
  \extractkey/robExt/latex/set preamble hyperref=\marg{code of preamble}\@nil
  \extractkey/robExt/latex/add to preamble hyperref=\marg{code of preamble}\@nil
  \extractkey/robExt/latex/set preamble after hyperref=\marg{code of preamble}\@nil
  \extractkey/robExt/latex/add to preamble after hyperref=\marg{code of preamble}\@nil
  \makeatother%
  \pgfmanualbody
  Set/add element to the preamble (defaults to |standalone|). Internally, it sets the placeholder |__ROBEXT_LATEX_PREAMBLE__|. The variations |hyperref| and |after hyperref| are used to put stuff after the preambule, as |hyperref| typically needs to be loaded last (or nearly): packages that must be loaded after |hyperref| like |cref| can be added via |add to preamble after hyperref|.
\end{pgfmanualentry}

\begin{pgfmanualentry}
  \makeatletter%
  \def\extrakeytext{style, }
  \extractkey/robExt/latex/do not wrap code\@nil
  \makeatother%
  \pgfmanualbody
  By default, the main content is wrapped into a box in order to measure its depth to properly set the baseline. If you do not want to do this wrapping, you can set this option. Internally, it is a shortcut for:\\
  |set placeholder={__ROBEXT_LATEX_MAIN_CONTENT_WRAPPED__}{__ROBEXT_MAIN_CONTENT__}|
  \textbf{IMPORTANT}: note that it means that you might need to adapt your code to take into account the fact that they are inside a box (I don't know of any other solution to compute the depth, but it does not mean that there is none).
\end{pgfmanualentry}

\begin{pgfmanualentry}
  \makeatletter%
  \def\extrakeytext{style, }
  \extractkey/robExt/latex/in command\@nil
  \makeatother%
  \pgfmanualbody
  Sets |__ROBEXT_MAIN_CONTENT| to point to |__ROBEXT_MAIN_CONTENT_ORIG__| to remove any wrapping of the user input, for instance made by tikzpicture.
\end{pgfmanualentry}

\begin{pgfmanualentry}
  \makeatletter%
  \def\extrakeytext{style, }
  \extractkey/robExt/tikz\@nil
  \extractkey/robExt/tikzpicture\@nil
  \makeatother%
  \pgfmanualbody
  |tikz| loads |latex| and then adds |tikz| to the preamble. |tikzpicture| first load |tikz| and wraps the main content within |\begin{tikzpicture}| and |\end{tikzpicture}| using:\\
\begin{verbatim}
set placeholder={__ROBEXT_MAIN_CONTENT__}{%
  \begin{tikzpicture}__ROBEXT_MAIN_CONTENT_ORIG__\end{tikzpicture}%
}
\end{verbatim}
  so that the user does not need to type it. See for instance the introduction for examples of use.
\end{pgfmanualentry}


\subsubsection{Python}

We provide support for python:

\begin{pgfmanualentry}
  \makeatletter%
  \def\extrakeytext{style, }
  \extractkey/robExt/python\@nil
  \makeatother%
  \pgfmanualbody
  Load the |python| preset (inspect |__ROBEXT_PYTHON__|) for details on the exact template, but note that this template might be subject to changes. We also provide a few helper functions:
  \begin{itemize}
  \item |write_to_out(text)| writes |text| to the |*-out.tex| file that will be loaded automatically before running the include function
  \item |parse_args()| is a function that returns a dictionary mapping some keys to values depending on the called arguments: for instance, if you call the python file with |python script key1 value1 key2 value2|, then the dictionary will map |key1| to |value1| and |key2| to |value2|. You might like this in conjunction with commands presented in \cref{sec:configureCompilationCommand}. Note that if you place placeholders in your code, you might not need this, but this is used if you plan to use your script outside of this library.
  \item |get_cache_folder()| outputs the cache folder.
  \item |get_file_base()| outputs the prefix of all files that should be created by this script, that looks like |robExt-somehash|.
  \item |get_current_script()| returns the current script.
  \item |get_filename_from_extension(extension)| outputs the prefix |robExt-somehash| concatenated with the extension. You often need this function to get the path of a file that your script is creating, for instance, |get_filename_from_extension("-out.txt")| is the path |*-out.txt| of the file that is read by |verbatim output|.
  \item |get_verbatim_output()| returns |get_filename_from_extension("-out.txt")|
  \item |finished_with_no_error()| creates the pdf file if it does not exists (to certify that the compilation ran without issues). The template automatically runs this function at the end. 
  \end{itemize}
  We demonstrate its usage on a few examples:
\begin{codeAndResult}
\begin{CacheMeCode}{python, verbatim output}
with open(get_verbatim_output(), "w") as f:
    for i in range(5):
        f.write(f"Hello {i}\n")
\end{CacheMeCode}
\end{codeAndResult}

\textbf{Importantly: you do not want to indent the whole content of CacheMeCode, or the spaces will also appear in the final code.}

You can also generate some images. This code will produce the image in \cref{fig:pythonGeneratedImage2}:
\begin{codeexample}[code only]
\begin{CacheMeCode}{python, set includegraphics options={width=.8\linewidth}}
import matplotlib.pyplot as plt
year = [2014, 2015, 2016, 2017, 2018, 2019]  
tutorial_count = [39, 117, 111, 110, 67, 29]
plt.plot(year, tutorial_count, color="#6c3376", linewidth=3)  
plt.xlabel('Year')  
plt.ylabel('Number of futurestud.io Tutorials')   
plt.savefig("__ROBEXT_OUTPUT_PDF__")  
\end{CacheMeCode}
\end{codeexample}
{\begin{figure}
\centering
\begin{CacheMeCode}{python, set includegraphics options={width=.8\linewidth}}
import matplotlib.pyplot as plt
year = [2014, 2015, 2016, 2017, 2018, 2019]  
tutorial_count = [39, 117, 111, 110, 67, 29]
plt.plot(year, tutorial_count, color="#6c3376", linewidth=3)  
plt.xlabel('Year')  
plt.ylabel('Number of futurestud.io Tutorials')   
plt.savefig("__ROBEXT_OUTPUT_PDF__")
\end{CacheMeCode}
\caption{Image generated with python.}
\label{fig:pythonGeneratedImage2}
\end{figure}
}

Note that by default, the executable called |python| is run. It seems like on windows |python3| is not created and only |python| exists, while on linux the user can choose whether |python| should point to |python3| or |python2| (on NixOs, I directly have |python| pointing to |python3|, and in ubuntu, you might need to install |python-is-python3| or create a symlink, as explained \href{https://askubuntu.com/questions/1296790/python-is-python3-package-in-ubuntu-20-04-what-is-it-and-what-does-it-actually}{here}). In any case, you can customize the name of the executable by setting something like:
\begin{verbatim}
\setPlaceholder{__ROBEXT_PYTHON_EXEC__}{python3}
\end{verbatim}
or using the style |force python3| that forces |python3|.
\end{pgfmanualentry}


\begin{pgfmanualentry}
  \makeatletter%
  \def\extrakeytext{style, }
  \extractkey/robExt/python print code and result\@nil
  \makeatother%
  \pgfmanualbody
  This is a demo style that can print a python code and its result.
\begin{codeAndResult}
\begin{CacheMeCode}{python print code and result, set title={The for loop}}
for name in ["Alice", "Bob"]:
    print(f"Hello {name}")
\end{CacheMeCode}
\end{codeAndResult}
You can set |__ROBEXT_PYTHON_TCOLORBOX_PROPS__| the options of the tcolorbox,\\ |__ROBEXT_PYTHON_CODE_MESSAGE__| and |__ROBEXT_PYTHON_RESULT_MESSAGE__| which are displayed before the corresponding block, |__ROBEXT_PYTHON_LSTINPUT_STYLE__| which contains the default lstinput style and |__MY_TITLE__| (cf |set title|) that contains the title of the box. Make sure to have the following packages to use the default styling:
\begin{verbatim}
\usepackage{pythonhighlight}
\usepackage{tcolorbox}
\end{verbatim}
\end{pgfmanualentry}

\begin{pgfmanualentry}
  \makeatletter%
  \def\extrakeytext{style, }
  \extractkey/robExt/python/add import\@nil
  \makeatother%
  \pgfmanualbody
  (since v2.1) Add an import statement in the first part of the file.
\begin{codeexample}[vbox]
\robExtConfigure{
  add to preset={my python}{
    python print code and result,
    add import={from math import cos},
    add import={from math import sin},
  },
}
\begin{CacheMeCode}{my python}
print(cos(1)+sin(2))
\end{CacheMeCode}
\end{codeexample}

\end{pgfmanualentry}

\subsubsection{Bash}

We provide a basic bash template, that sets:
\begin{verbatim}
set -e
outputTxt="__ROBEXT_OUTPUT_PREFIX__-out.txt"
outputTex="__ROBEXT_OUTPUT_PREFIX__-out.tex"
outputPdf="__ROBEXT_OUTPUT_PDF__"
\end{verbatim}
in order to quit when an error occurs, and to define two variables containing the path to the |pdf| file and to the file that is read by the |verbatim output| setting (that just apply a |\verbatiminput| on that file). Finally, it also creates the file |outputPdf| with |touch| in order to notify that the compilation succeeded.

In practice:
\begin{codeAndResult}
\begin{CacheMeCode}{bash, verbatim output}
# $outputTxt contains the path of the file that will be printed via \verbatiminput
uname -srv > "${outputTxt}"
\end{CacheMeCode}
\end{codeAndResult}

\subsubsection{Verbatim text}

Sometimes, it might be handy to write the text to a file and use it somehow. This is possible using |verbatim text|, that defaults to calling |\verbatiminput| on that file:

\begin{codeAndResult}
\begin{CacheMeCode}{verbatim text}
def some_verbatim_fct(a):
    # See this is a verbatim code where I can use the % symbol
    return a % b
\end{CacheMeCode}  
\end{codeAndResult}

You can also call |verbatim text no include|: it will not include the text, but it sets a macro |\robExtPathToInput| containing the path to the input file. Use it the way you like! For instance, we define here a macro |codeAndResult| that prints the code and runs it (we use a pretty printer from pgf, so you need to load |\usepackage{tikz}\input{pgfmanual-en-macros.tex}| to use it). It is what we use right now in this documentation for verbatim blocks like here. You can obtain a simpler version using:

\begin{codeAndResult}
\begin{CacheMeCode}{verbatim text no include}
\NewDocumentCommand{\testVerbatim}{+v}{
\begin{flushleft}\ttfamily%
#1
\end{flushleft}}
\testVerbatim{Demo % with percent}
\end{CacheMeCode}
We will input the file \robExtPathToInput{}:
\input{\robExtPathToInput}
This file contains:
\verbatiminput{\robExtPathToInput}
\end{codeAndResult}

You might also like to use |name output=yourfile| that will create two macros |\yourfile| and |\yourfileInCache|, equal respectively to the prefix |robExt-somehash| and |pathOfCache/robExt-somehash|. 

\subsubsection{Custom languages}

To add support to new languages, see \cref{sec:otherLanguages} for an example.


\subsection{List of special placeholders and presets}\label{sec:placeholders}

This library defines a number of pre-existing placeholders, or placeholders playing a special role. We list some of them in this section. All placeholders created by this library start with |__ROBEXT_|. Note that you can list all predefined placeholders (at least those globally defined) using |\printImportedPlaceholdersExceptDefaults| (note that some other placeholders might be created directly in the style set right before the command, and may not appear in this list if you call it before setting the style).

\subsubsection{Generic placeholders}

We define two special placeholders that should be defined by the user (possibly indirectly, using presets offered by this library):
\begin{itemize}
\item |__ROBEXT_TEMPLATE__| is a placeholder that should contain the code of the file to compile.
\item |__ROBEXT_MAIN_CONTENT_ORIG__| is a placeholder containing the text typed by the user, automatically set by |CacheMe|, |CacheMeCode| etc. You rarely need to deal with this placeholder directly since |__ROBEXT_MAIN_CONTENT__| will typically point to it and add some necessary wrapping.
\item |__ROBEXT_MAIN_CONTENT__| is a placeholder that might be used inside |__ROBEXT_TEMPLATE__|, that points by default to |__ROBEXT_MAIN_CONTENT_ORIG__| and that contains the content that should be typed inside the document. For instance, this might be a tikz picture, a python function without the import etc. Note that it is often used to wrap the text of the user |__ROBEXT_MAIN_CONTENT_ORIG__|: for instance, the |tikzpicture| preset adds the |\begin{tikzpicture}| around the user code automatically: this way we do not need to edit the command to disable externalization.
\item |__ROBEXT_COMPILATION_COMMAND__| contains the compilation command to run to compile the file (assuming we are in the cache folder).
\end{itemize}

We also provide a number of predefined placeholders in order to get the name of the source file etc... Note that most of these placeholders are defined (and/or expanded inplace) late during the compilation stage as one needs first to obtain the hash of the file, and therefore all dependencies, the content of the template etc.
\begin{itemize}
\item |__ROBEXT_SOURCE_FILE__| contains the path of the file to compile (containing the content of |__ROBEXT_TEMPLATE__|) like |robExt-somehash.tex|, relative to the cache folder (since we always go to this folder before doing any action, you most likely want to use this directly in the compilation command).
\item |__ROBEXT_OUTPUT_PDF__| contains the path of the pdf file produced after the compilation command relative to the cache folder (like |robExt-somehash.pdf|). Even if you do not plan to output a pdf file, you should still create that file at the end of the compilation so that this library can know whether the compilation succeeded. 
\item |__ROBEXT_OUTPUT_PREFIX__| contains the prefix that all newly created file should follow, like |robExt-somehash|. If you want to create additional files (e.g. a picture, a video, a console output etc...) make sure to make it start with this string. It will not only help to ensure purity, but it also allows us to garbage collect useless files easily.
\item |__ROBEXT_WAY_BACK__| contains the path to go back to the main project from the cache folder, like |../| (internally it is equals to the expanded value of |\robExtPrefixPathWayBack|).
\item |__ROBEXT_CACHE_FOLDER__| contains the path to the cache folder. Since most commands are run from the cache folder, this should not be really useful to the user.
\end{itemize}

You can also use these placeholders to customize the default include function:
\begin{itemize}
\item |__ROBEXT_INCLUDEGRAPHICS_OPTIONS__| contains the options given to |\includegraphics| when loading the pdf
\item |__ROBEXT_INCLUDEGRAPHICS_FILE__| contains the file loaded by |\includegraphics|, defaults to |\robExtAddCachePathAndName{\robExtFinalHash.pdf}|, that is itself equivalent to |__ROBEXT_CACHE_FOLDER____ROBEXT_OUTPUT_PDF__| or\\|__ROBEXT_CACHE_FOLDER____ROBEXT_OUTPUT_PREFIX__.pdf|.
\end{itemize}

\subsubsection{Placeholders related to \LaTeX{}}
Some placeholders are reserved only when dealing with \LaTeX{} code:
\begin{itemize}
\item |__ROBEXT_LATEX__| is the main entrypoint, containing all the latex template. It internally calls other placeholders listed below.
\item |__ROBEXT_LATEX_OPTIONS__|: contains the options to compile the document, like |a4paper|. Empty by default.
\item |__ROBEXT_LATEX_DOCUMENT_CLASS__|: contains the class of the document. Defaults to |standalone|.
\item |__ROBEXT_LATEX_PREAMBLE__|: contains the preamble. Is empty by default.
\item |__ROBEXT_LATEX_MAIN_CONTENT_WRAPPED__|: content inside the |document| environment. It will wrap the actual content typed by the user |__ROBEXT_MAIN_CONTENT__| around a box to compute its depth. If you do not want this behavior, you can set |__ROBEXT_LATEX_MAIN_CONTENT_WRAPPED__| to be equal to |__ROBEXT_MAIN_CONTENT__|. It calls internally |__ROBEXT_LATEX_CREATE_OUT_FILE__| and |__ROBEXT_LATEX_WRITE_DEPTH_TO_OUT_FILE__| to do this computation.
\item |__ROBEXT_LATEX_CREATE_OUT_FILE__| creates a new file called |\jobname-out.tex| and open it in the handle called |\writeRobExt|
\item |__ROBEXT_LATEX_WRITE_DEPTH_TO_OUT_FILE__| writes the height, depth and width of the box |\boxRobExt| into the filed opened in |\writeRobExt|.
\item |__ROBEXT_LATEX_COMPILATION_COMMAND__| is the command used to compile a \LaTeX{} document. It uses internally other placeholders:
\item |__ROBEXT_LATEX_ENGINE__| is the engine used to compile the document (defaults to |pdflatex|)
\item |__ROBEXT_LATEX_COMPILATION_COMMAND_OPTIONS__| contains the options used to compile the document (defaults to |-shell-escape -halt-on-error|)
\end{itemize}

\subsubsection{Placeholders related to python}

\begin{itemize}
\item |__ROBEXT_PYTHON_EXEC__| contains the python executable (defaults to |python|) used to compile
\item |__ROBEXT_PYTHON__| contains the python template
\item |__ROBEXT_PYTHON_IMPORT__| can contain import statements
\item |__ROBEXT_PYTHON_MAIN_CONTENT_WRAPPED__| is used to add all the above functions. You can set it to |__ROBEXT_MAIN_CONTENT__| if you do not want them
\item |__ROBEXT_PYTHON_FINISHED_WITH_NO_ERROR__| is called at the end to create the pdf file even if it is not created, you can set it to the empty string if you do not want to do that.
\end{itemize}

\subsubsection{Placeholders related to bash}

\begin{itemize}
\item |__ROBEXT_BASH_TEMPLATE__| contains the bash template. By default, it sets |set -e|, creates |outputTxt|, |outputTex| and |outputPdf| pointing to the corresponding files, and it created the pdf file at the end.
\item |__ROBEXT_SHELL__| contains the shell (defaults to |bash|).
\end{itemize}


\subsection{Customize presets and create your own style}\label{sec:customize}

Note that you can define your own presets simply by creating a new pgf style (please refer to tikz-pgf's documentation for more details). For instance, we defined the |tikz| and |tikzpicture| presets using:
\begin{codeexample}[code only]
\robExtConfigure{
  new preset={tikz}{
    latex,
    add to preamble={\usepackage{tikz}},
  },
  new preset={tikzpicture}{
    tikz,
    set placeholder={__ROBEXT_MAIN_CONTENT__}{\begin{tikzpicture}__ROBEXT_MAIN_CONTENT_ORIG__\end{tikzpicture}},
  },
}
\end{codeexample}
in order to automatically load |tikz| and add the surrounding |tikzpicture| when needed (note that the style is always loaded \textbf{after} the definition of |__ROBEXT_MAIN_CONTENT_ORIG__|, so in theory you could also modify it directly even if it is not recommended). You can also customize an existing style by adding stuff to it using |add to preset| (or |.append style| but make sure to double the hashes). For instance, here, we add the |shadows| library to the |tikz| preset by default:
\begin{codeexample}[width=0pt,vbox]
  \robExtConfigure{
    add to preset={tikz}{
      add to preamble={\usetikzlibrary{shadows}},
    },
  }
  See, tikz's style now packs the |shadows| library by default: %
  \begin{CacheMe}{tikzpicture}[even odd rule]
    \filldraw [drop shadow,fill=white] (0,0) circle (.5) (0.5,0) circle (.5);
  \end{CacheMe}
\end{codeexample}


\subsection{Cache automatically a given environment}\label{sec:wrapAutomatically}

It might be handy to cache automatically a given environment: we already provide:\\ |\cacheTikz|\\
to cache all tikz pictures (unless externalization is disabled), but we also provide tools to handle arbitrary environments.

\begin{pgfmanualentry}
  \extractcommand\robExtExternalizeAllTikzpicture\opt{\oarg{preset for tikz}}\opt{\oarg{preset for tikzpicture}}\opt{\oarg{delimiters}}\@@
  \extractcommand\cacheTikz\opt{\oarg{preset for tikz}}\opt{\oarg{preset for tikzpicture}}\opt{\oarg{delimiters}}\@@
  \pgfmanualbody
  This will automatically cache all tikz pictures (both are alias, |\cacheTizk| is available from v2.0). Since v2.0, we added the delimiters options and we allow to specify custom presets for |tikz| and |tikzpicture|, and we parse |\tikz| as well (the |tikz| preset is now used by |\tikz| by default while the |tikzpicture| preset is used by |tikzpicture|). Note that we add an additional optional argument to the tikz picture via its first argument delimited by |delimiters| (defaults to |<>|) to specify preset options, which can for instance be practical to disable externalization on individual pictures. Cf.\ \cref{sec:disableExternalization} to see an example of use.
\begin{codeexample}[width=0pt,vbox]
\cacheTikz

\def\whereIAm{(not cached)}
\robExtConfigure{
  add to preset={tikz}{
    add to preamble={
      \def\whereIAm{(cached)}
    },
  },
}

\tikz \node[fill=pink, rounded corners]{See that this syntax can be safely used. \whereIAm};

\tikz<add to preamble={\def\sayHello#1{Hello #1!}}> \node[fill=green, rounded corners]{\sayHello{Alice} \whereIAm};

\tikz<disable externalization> \node[fill=green, rounded corners]{Foo \whereIAm};
\end{codeexample}
Here is an example to specify arguments via parenthesis:
\begin{codeexample}[width=0pt,vbox]
% the first two arguments are the default presets used by \tikz and \tikzpicture
\cacheTikz[tikz][tikzpicture][()]

\def\whereIAm{(not cached)}
\robExtConfigure{
  add to preset={tikz}{
    add to preamble={
      \def\whereIAm{(cached)}
    },
  },
}

\tikz \node[fill=pink, rounded corners]{See that this syntax can be safely used. \whereIAm};

\tikz(add to preamble={\def\sayHello#1{Hello #1!}}) \node[fill=green, rounded corners]{\sayHello{Alice} \whereIAm};

\tikz(disable externalization) \node[fill=green, rounded corners]{Foo \whereIAm};
\end{codeexample}
You can also change the default style loaded for tikz and tikzpicture (note that you might prefer to modify |tikz| directly using |add to preset|), but you need v2.1 (buggy in v2.0):

\begin{codeexample}[vbox]
% You might prefer to modify tikz directly using "add to preset"
\cacheTikz[tikz, add to preamble={\def\hello{H}},][tikzpicture,
  add to preamble={\def\hello{HH}},
]

\tikz[baseline=(A.base)] \node(A){In tikz \hello}; %
\begin{tikzpicture}[baseline=(A.base)]
  \node(A){In tikzpicture \hello};
\end{tikzpicture}
\end{codeexample}

\end{pgfmanualentry}

\begin{pgfmanualentry}
  \extractcommand\robExtDisableTikzpictureOverwrite\@@
  \pgfmanualbody
  This is useful to temporarily reset the current environment to their original value (since v2.0 it also resets other environments, not just |tikzpicture|). This must typically be called at the beginning of |command if no externalization| to avoid infinite recursion if you redefine it, but expect for this case the user is not expected to use this option.
\end{pgfmanualentry}

Let us say that you want to cache all elements of a given environment, like |minipage| or |zx-calculus| pictures (another package of mine):


\begin{pgfmanualentry}
  \extractcommand\cacheEnvironment\opt{\marg{delimiters}}\marg{name environment}\marg{default preset options}\@@
  \pgfmanualbody
  This will automatically cache the corresponding environment (but note that you still need to define it in the preamble of the cached files, for instance by loading the appropriate package):
  \begin{codeexample}[width=0pt,vbox]
    \cacheEnvironment{ZX}{latex, add to preamble={\usepackage{zx-calculus}}}
    A ZX figure %
    \begin{ZX}
      \zxX{\alpha} \rar \ar[d,C] & \zxZ*{a\pi} \\
      \zxZ{\beta} \rar           & \zxX*{b\pi}
    \end{ZX} and the same but aligned on the second line %
    \begin{ZX}[mbr=2]
      \zxX{\alpha} \rar \ar[d,C] & \zxZ*{a\pi} \\
      \zxZ{\beta} \rar           & \zxX*{b\pi}
    \end{ZX}.
  \end{codeexample}  
  You can of course configure them using globally defined configuration, but you can also provide arguments to a single picture using the delimiters |delimiters| that default to |<...>| as the optional argument, like: %
  \begin{codeexample}[width=0pt,vbox]
    \cacheEnvironment{ZX}{latex, add to preamble={\usepackage{zx-calculus}}}
    \begin{ZX}<add to preamble={\usepackage{amsmath}}> % amsmath provides \text
      \zxX{\alpha} \rar \ar[d,C] & \zxZ{\text{yes}} \\
      \zxZ{\beta} \rar           & \zxX*{b\pi}
    \end{ZX}
  \end{codeexample}
  If you do not like |<>| or if your command already have this parameter, you can change it, for instance to get parens as delimiters, use:
  \begin{codeexample}[width=0pt,vbox]
    \cacheEnvironment[()]{ZX}{latex, add to preamble={\usepackage{zx-calculus}}}
    \begin{ZX}(add to preamble={\usepackage{amsmath}}) % amsmath provides \text
      \zxX{\alpha} \rar \ar[d,C] & \zxZ{\text{yes}} \\
      \zxZ{\beta} \rar           & \zxX*{b\pi}
    \end{ZX}
  \end{codeexample}
  otherwise you must provide some presets arguments, possibly empty:\\
  |\begin{someweirdenv}<><args to someweirdenv>|

    You can also use this argument to disable externalisation on a per-picture basis:
  \begin{codeexample}[width=0pt,vbox]
    \cacheEnvironment{ZX}{latex, add to preamble={\usepackage{zx-calculus}}}
    \begin{ZX}<disable externalization> % amsmath provides \text
      \zxX{\alpha} \rar \ar[d,C] & \zxZ*{a\pi} \\
      \zxZ{\beta} \rar           & \zxX*{b\pi}
    \end{ZX}
  \end{codeexample}
  
\end{pgfmanualentry}

\begin{pgfmanualentry}
  \extractcommand\cacheCommand\opt{\oarg{delimiter}}\marg{name command}\opt{\oarg{xparse signature}}\marg{default preset options}\@@
  \pgfmanualbody
  This will automatically cache the corresponding command, in a similar way to environment. It work exactly the same, except that we (usually) need to specify the signature of the command or it is impossible to know where the command stops. If the command is defined using xparse-compatible commands like |\NewDocumentCommand|, this detection is automatically done. Otherwise, you need to specify it in the xparse format: long story short, |O{foo}| is an optional argument with default value foo, |m| is a mandatory argument.

  First, let us see an example with a command defined using xparse:
  \begin{codeexample}[width=0pt,vbox]
    % We cache the command \zx that is defined with NewDocumentCommand: no need to specify the signature
    \cacheCommand{zx}{latex, add to preamble={\usepackage{zx-calculus}}}
    A ZX figure %
    \zx{
      \zxX{\alpha} \rar \ar[d,C] & \zxZ*{a\pi} \\
      \zxZ{\beta} \rar           & \zxX*{b\pi}
    } and the same but aligned on the second line %
    \zx[mbr=2]{
      \zxX{\alpha} \rar \ar[d,C] & \zxZ*{a\pi} \\
      \zxZ{\beta} \rar           & \zxX*{b\pi}
    }.
  \end{codeexample}
  
  You can of course configure them using globally defined configuration, but you can also provide arguments to a single picture using the delimiters that default to |<>| as the optional argument, like: %
  \begin{codeexample}[width=0pt,vbox]
    \cacheCommand{zx}{latex, add to preamble={\usepackage{zx-calculus}\def\hello#1{Hello #1.}}}
    \zx<add to preamble={\usepackage{amsmath}\def\bye#1{Bye #1.}}>[mbr=1]{ % amsmath provides \text
      \zxX{\alpha} \rar \ar[d,C] & \zxZ{\text{\hello{Alice}}} \\
      \zxZ{\beta} \rar           & \zxZ{\text{\bye{Bob}}}
    }
  \end{codeexample}
  or with parens as delimiters:
  \begin{codeexample}[width=0pt,vbox]
    \cacheCommand[()]{zx}{latex, add to preamble={\usepackage{zx-calculus}}}
    \zx(add to preamble={\usepackage{amsmath}})[mbr=1]{ % amsmath provides \text
      \zxX{\alpha} \rar \ar[d,C] & \zxZ{\text{yes}} \\
      \zxZ{\beta} \rar           & \zxX*{b\pi}
    }
  \end{codeexample} 

  Here is an example with a command not defined with |\NewDocumentCommand| (we also show how you can use the optional argument to disable externalisation on a per-picture basis):
\begin{codeexample}[width=0pt,vbox]
\def\fromWhere{I am not cached}
\newcommand{\myMacroNotDefinedWithNewDocumentCommand}[3][default value]{
  Optional argument 1: #1,
  Argument 2: #2,
  Argument 3: #3 (\fromWhere).
}
\cacheCommand{myMacroNotDefinedWithNewDocumentCommand}[O{default value}mm]{
  latex,
  add to preamble={%
    \def\fromWhere{I am cached}
    \newcommand{\myMacroNotDefinedWithNewDocumentCommand}[3][default value]{
      Optional argument 1: #1,
      Argument 2: #2,
      Argument 3: #3 (\fromWhere).
    }
  }
}

\myMacroNotDefinedWithNewDocumentCommand{second arg}{last arg}\\
\myMacroNotDefinedWithNewDocumentCommand<disable externalization>{second arg}{last arg}
\end{codeexample}
  
\end{pgfmanualentry}

\paragraph{Defining its own macro without repeating its definition.}
Note that when you define yourself a macro function, the above structure might not be optimal as you need to define the macro in the main document (in case you disable the externalization) and in the template. One option to avoid repeated code is to write the definition in a common input file (see \cref{sec:dependencies}), but you might prefer to keep everything in a single file. In that case, you can simply wrap your function into |cacheMe| yourself. Say you want your function to draw a circle like, |\tikz \draw[fill=#2] (0,0) circle [radius=#1];|, then wrap in inside |CacheMe|:

\begin{codeexample}[width=0mm,vbox]
% D<>{} is an optional argument enclosed in <>, we use this to pass arguments to CacheMe
\NewDocumentCommand{\drawMyCircle}{D<>{}O{2mm}m}{
  \begin{CacheMe}{tikz, #1}
    \tikz \draw[fill=#3] (0,0) circle [radius=#2];
  \end{CacheMe}
}

You can externalize it \drawMyCircle{pink} %
or disable externalization \drawMyCircle<disable externalization>[4mm]{red}.
\end{codeexample}

\paragraph{Manually wrap a command.} In some more advanced cases, if you do not know the definition of the command and if |\cacheCommand| does not work (e.g. the number of mandatory argument depends on the first argument), you can wrap it manually. To do so, first copy first your command using |\DeclareCommandCopy{\robExtCommandOrigNAMEOFCOMMAND}{\NAMEOFCOMMAND}| and run |\robExtAddToCommandResetList{NAMEOFCOMMAND}| (this is needed to disable externalization), and override the command by wrapping it into a cache. Make sure to add |\def\robExtCommandOrigName{NAMEOFCOMMAND}| at the beginning of the function or |disable externalization| will not work:

\begin{codeAndResult}
Original command: \zx{\zxZ{\alpha} \rar & \zxX{\beta}}. %
\DeclareCommandCopy{\robExtCommandOrigzx}{\zx}% used to recover the original function
\robExtAddToCommandResetList{zx}% let us know that this function should be reset when externalization is disabled
\DeclareDocumentCommand{\zx}{D<>{}O{}m}{%
  \def\robExtCommandOrigName{zx}%
  \begin{CacheMe}{latex, add to preamble={\usepackage{zx-calculus}}, #1}%
    \zx[#2]{#3}%
  \end{CacheMe}%
}
This is cached \zx<>{\zxZ{\alpha} \rar & \zxX{\beta}}, you can add options like %
\zx<add to preamble={\usepackage{amsmath}}>{\zxZ{\alpha} \rar & \zxX{\text{Hey}}} %
and we can disable the externalization: %
\zx<disable externalization>{\zxZ{\alpha} \rar & \zxX{\text{Hey}}}.
\end{codeAndResult}

You can do a similar trick for environment, by using instead
\begin{verbatim}
\NewEnvironmentCopy{robExtEnvironmentOrigNAMEENV}{NAMEENV}
\robExtAddToEnvironmentResetList{NAMEENV}
\def\robExtEnvironmentOrigName{NAMEENV}
\end{verbatim}

\subsection{Operations on the cache}\label{sec:operationsCache}

Every time we compile a document, we create automatically a bunch of files:
\begin{itemize}
\item the cache is located by default in the |robustExternalize| folder. Feel free to remove this folder if you want to completely clear the cache (but then you need to recompile everything). See below if you want to clean it in a better way.
\item |\jobname-robExt-all-figures.txt| contains the list of all figures contained in the document. Mostly useful to help the script that remove other figures.
\item |robExt-remove-old-figures.py| is a python script that will remove all cached files that are not used anymore. Just run |python robExt-remove-old-figures.py| to clean it. You will then see the list of files that the script wants to remove: make sure it does not remove any important data, and press ``y''. Note that it will search for all files that look like |*robExt-all-figures.txt| to see the list of pictures that are still in use, and by default it will only remove the images in the |robustExternalize| folder that start with |robExt-|. If you change the path of the cache or the prefix, edit the script (should not be hard to do).
\item |\jobname-robExt-compile-missing-figures.sh| contains a list of commands that you need to run to compile the images not yet compiled in the cache (this list will only be created if you enable the manual compilation mode).
\item |\jobname-robExt-tmp-file-you-can-remove.tmp| is a temporary file. Feel free to remove it.
\end{itemize}

We go over some of these scripts.

\subsubsection{Cleaning the cache}

You might want to clean the cache. Of course you can remove all generated files, but if you want to keep the picture in use in the latest version of the document, we provide a python script (automatically generated in the root folder) to do this. Just install python3 and run:\\

|python3 robExt-remove-old-figures.py|\\

(on windows, the executable might be called |python|) You will then be prompted for a confirmation after providing the list of files that will be removed.

\subsubsection{Listing all figures in use}

After the compilation of the document, a file |robExt-all-figures.txt| is created with the list of the |.tex| file of all figures used in the current document.

\subsubsection{Manually compiling the figures}

When enabling the manual mode (useful if we don't want to enable |-shell-escape|):

\begin{verbatim}
\robExtConfigure{
  enable manual mode
}
\end{verbatim}

or the fallback to manual mode (it will only enable the manual mode if |-shell-escape| is disabled):
\begin{verbatim}
\robExtConfigure{
  enable fallback to manual mode,
}
\end{verbatim}

the library creates a file |JOBNAME-robExt-compile-missing-figures.sh| that contains the instructions to build the figures that are not yet in the cache (each line contains the compilation command to run). On Linux (or on Windows with bash/cygwin/… installed, it possibly even work out of the box without) you can easily execute them using:

\begin{verbatim}
bash JOBNAME-robExt-compile-missing-figures.sh
\end{verbatim}


\subsection{How to debug}\label{sec:debug}

If for some reasons you are unable to understand why a build fails, first check if you compiled your document with |-shell-escape| (not that this must appear \textbf{before} the filename). Then, you can look at the log file to get more advices: when a cached document is compiled, we always write the full compilation command before compiling the file in the log file. This way, you can easily check the content of the file and see why it fails to compile. The compilation errors are also displayed directly in the output, and you can read the log file for details.

You might often get an error |! Missing $ inserted.|: this is typically when a placeholder was left unreplaced (e.g.\ you forgot to define it, import it, or you forgot to wrap a command in |\evalPlaceholder{}|): since \LaTeX{} is asked to typeset |__something__|, it thinks that you are trying to write a subscript, and asks you to start first the math mode.

To get more advanced info during the compilation command, set the style |more logs| (that does |\def\robExtDebugMessage#1{\message{^^J[robExt] #1}}|) in order to print some messages on the replacement process (there is also |\robExtDebugWarning| for more important messages, but it prints the logs by default.

Then, it is often handy to check the content of the generated template. You can either get the name in the log, via |name output|, or print it directly in the file:

\begin{pgfmanualentry}
  \makeatletter%
  \def\extrakeytext{style, }
  \extractkey/robExt/debug\@nil
  \extractkey/robExt/print command and source\@nil
  \makeatother%
  \pgfmanualbody
You can also use |debug| (or its alias |print command and source|) in order to disable the compilation and print instead the compilation command and the source file:

\begin{codeexample}[width=0pt,vbox]
  \begin{tikzpictureC}<debug>[baseline=(A.base)]
    \node[fill=red, rounded corners](A){My node that respects baseline \ding{164}.};
    \node[fill=red, rounded corners, opacity=.3,overlay] at (A.north east){I am an overlay text};
  \end{tikzpictureC}
\end{codeexample}
\begin{codeexample}[width=0pt,vbox]
\end{codeexample}
\end{pgfmanualentry}

\begin{pgfmanualentry}
  \extractcommand\robExtShowPlaceholder\opt{*}\marg{placeholder name}\@@
  \makeatletter%
  \def\extrakeytext{style, }
  \extractkey/robExt/show placeholder=\marg{placeholder name}\@nil
  \makeatother%
  \pgfmanualbody
  |\robExtShowPlaceholder{__ROBEXT_MAIN_CONTENT__}| will show in the terminal the content of the placeholder. Use the start not to pause.
\end{pgfmanualentry}

\begin{pgfmanualentry}
  \extractcommand\robExtShowPlaceholders\opt{*}\@@
  \extractcommand\robExtShowPlaceholdersContents\opt{*}\@@
  \makeatletter%
  \def\extrakeytext{style, }
  \extractkey/robExt/show placeholders\@nil
  \extractkey/robExt/show placeholders contents\@nil
  \makeatother%
  \pgfmanualbody
  Show all imported placeholders and their content. Use the start not to pause.
\end{pgfmanualentry}


\subsection{Advanced optimizations}\label{sec:advancedOperations}

We made a number of optimizations to get near instant feedback on documents having hundreds of pictures, basically by saying which placeholder should be replaced, in which order, when to stop replacing placeholders, and possibly by compiling templates (see \cref{sec:compilePreset}). Most of them are invisible to the end user and pre-configured for presets like |latex| or when compiling a preset, but you might want to also use them on your own.

First note that the order used to import placeholders is important, as it will first replace the first imported placeholders: so if you imported the placeholders in the right order, a single loop should be enough to replace them all, otherwise you might need $n$ loops if you need to replace $n$ placeholders (the basic evaluation procedure just loops over placeholders and replace all occurrences in the string until there is either no remaining placeholders (see below), or if you obtain the same result after two loops).


\begin{pgfmanualentry}
  \extractcommand\onlyPlaceholdersInCompilationCommand\marg{list,of,placeholders}\@@
  \extractcommand\firstPlaceholdersInCompilationCommand\marg{list,of,placeholders}\@@
  \extractcommand\onlyPlaceholdersInTemplate\marg{list,of,placeholders}\@@
  \extractcommand\firstPlaceholdersInTemplate\marg{list,of,placeholders}\@@
  \pgfmanualbody
  Using these commands (possibly inside |/utils/exec={\yourcommand{foo}}| in a style), you will tell to the system to evaluate the compilation/template placeholders (depending on the name of the command) only using the placeholders in the argument, or (for the commands starting with |first|), by starting with the placeholders in argument before continuing with others placeholders.
\end{pgfmanualentry}

\begin{pgfmanualentry}
  \makeatletter%
  \def\extrakeytext{style, }
  \extractkey/robExt/all placeholders have underscores\@nil
  \extractkey/robExt/not all placeholders have underscores\@nil
  \makeatother%
  \pgfmanualbody
In order to know when to stop replacing placeholders earlier (i.e.\ without running until it converges), it is practical to know how placeholders look like in order to stop before. This (enabled by default) says to the library that all placeholders should start with two underscores |__|, allowing us to easily check if a string contains a placeholder. If you want to use placeholders with fancy names and no underscores, disable this feature (but it might be a bit slower).
\end{pgfmanualentry}


\begin{pgfmanualentry}
  \makeatletter%
  \def\extrakeytext{style, }
  \extractkey/robExt/do not load special placeholders\@nil
  \extractkey/robExt/load special placeholders\@nil
  \makeatother%
  \pgfmanualbody
  Similarly, to save some time, we do not import some special placeholders and only replace them manually at the end, the list being: |__ROBEXT_OUTPUT_PREFIX__|, |__ROBEXT_SOURCE_FILE__|, |__ROBEXT_OUTPUT_PDF__|, |__ROBEXT_WAY_BACK__|, |__ROBEXT_CACHE_FOLDER__|. If for some reasons you do want to import them earlier, you can do it this way.
\end{pgfmanualentry}

\begin{pgfmanualentry}
  \makeatletter%
  \def\extrakeytext{style, }
  \extractkey/robExt/disable placeholders\@nil
  \extractkey/robExt/enable placeholders\@nil
  \makeatother%
  \pgfmanualbody
  To save even more time, you can disable the placeholder system completely, and only replace the following placeholders: |__ROBEXT_MAIN_CONTENT__|, |__ROBEXT_LATEX_PREAMBLE__| and |__ROBEXT_MAIN_CONTENT_ORIG__|. This is done notably inside |new compiled preset| in order to speed up the compilation by maybe a factor of 1.5 or 2, but then you cannot use any other placeholders. Note that the default presets like |latex| enable placeholders, so you should be able to call |latex| inside a compiled template to come back to the uncompiled version.  
\end{pgfmanualentry}

\begin{pgfmanualentry}
  \makeatletter%
  \def\extrakeytext{style, }
  \extractkey/robExt/disable import mechanism\@nil
  \extractkey/robExt/enable optimizations\@nil
  \makeatother%
  \pgfmanualbody
  If you disable the import mechanism (enabled by default), it will do an |\importAllPlaceholders| while evaluating the template. This is not recommended as it is quite inefficient, it is better to selectively import the group you need.
\end{pgfmanualentry}

\begin{pgfmanualentry}
  \makeatletter%
  \def\extrakeytext{style, }
  \extractkey/robExt/disable optimizations\@nil
  \extractkey/robExt/enable optimizations\@nil
  \makeatother%
  \pgfmanualbody
  
\end{pgfmanualentry}



\section{TODO and known bugs:}

\begin{itemize}
\item See how to deal with label, links and cite inside pictures (without disabling externalization). For label and links, I have a proof of concept, I ``just'' need to write it down. \url{https://tex.stackexchange.com/questions/695277/clickable-includegraphics-for-cross-reference-data}
\item Deal with remember picture
\item Since externalizing is all about speed, it would be nice to do more benchmarks.
\end{itemize}

\section{Acknowledgments}

I am deeply indebted to many users on \url{tex.stackexchange.com} that made the writing of this library possible. I can't list you all, but thank you so much! A big thanks also to the project \url{https://github.com/sasozivanovic/memoize/} from which I borrowed most of the code to automatically wrap commands. Thanks to kiryph for providing useful feedbacks.

\section{Changelog}

\begin{itemize}
\item future v2.1 (master on github, not released yet on CTAN):
  \begin{itemize}
  \item Fix a bug when compiling the main document with lualatex or xelatex
  \item Fix a bug when compiling a template that was not exporting dependencies
  \item Add |recompile| to recompile a file even if no dependencies changed.
  \item Add a hook |robust-externalize/just-before-writing-files|.
  \item Fixed a bug in |\cacheTikz| that was not working with a custom argument for tikzpicture
  \item Added options |forward*| to forward macros and counters
  \item Added |auto forward| and co
  \item Added |compile in parallel| and co
  \end{itemize}
\item v2.0:
  \begin{itemize}
  \item \textbf{not backward compatible}: the preset |tikzpicture| should be used instead of |tikz|: |tikz| does not wrap anymore its content into |\begin{tikzpicture}| in order to be usable inside the command version |\tikz| (but |tikzpicture| does load |tikz| first).  Moreover, all arguments to |tikzpictureC| or |tikzpicture| (if you loaded |cacheTikz|) should be specified in the first argument, using |<your options>|. This is used to provide a more uniform interface with the new |\cacheEnvironment| command. Hopefully this should \textbf{not cause too much trouble} since the first version of the library has been published on CTAN only a few days ago.
  \item not really compatible: Renamed some latex placeholders to prefix all of them with |__ROBEXT_LATEX_|. Should not be a problem if you used the style made to change them.
  \item Added an alias to |\robExtExternalizeAllTikzpictures| called |\cacheTikz|. Now, this also configures |\tikz|.
  \item You can now use ampersands etc freely thanks to lrbox.
  \item You can now automatically cache environments using |\cacheEnvironment| and commands using |\cacheCommand|.
  \item We provide |new preset| and |add to preset| to avoid doubling the number of hashes.
  \item We provide new functions like |\setPlaceholderRecReplaceFromList| to only expand a subset of placeholders.
  \item An error message is given if you forget to set a template.
  \item Add |enable fallback to manual mode| to fallback to manual mode only if shell escape is missing.
  \item Allow restricted mode
  \item |debug| mode available
  \item Rename |set subfolder and way back| to |set cache folder and way back| (in a backward compatible way)
  \item Waaaaaaay more efficient. Also added a method to compile presets.
  \item I added the import system for efficiency reasons (not present in v1.0 at all). Note that I also added some other functions that I forgot to notify as new in the doc...
  \end{itemize}
\end{itemize}

\printindex

\end{document}
% Local Variables:
% TeX-command-extra-options: "-shell-escape -halt-on-error"
% End: